\documentclass[]{article}
\usepackage{lmodern}
\usepackage{amssymb,amsmath}
\usepackage{ifxetex,ifluatex}
\usepackage{fixltx2e} % provides \textsubscript
\ifnum 0\ifxetex 1\fi\ifluatex 1\fi=0 % if pdftex
  \usepackage[T1]{fontenc}
  \usepackage[utf8]{inputenc}
\else % if luatex or xelatex
  \ifxetex
    \usepackage{mathspec}
  \else
    \usepackage{fontspec}
  \fi
  \defaultfontfeatures{Ligatures=TeX,Scale=MatchLowercase}
\fi
% use upquote if available, for straight quotes in verbatim environments
\IfFileExists{upquote.sty}{\usepackage{upquote}}{}
% use microtype if available
\IfFileExists{microtype.sty}{%
\usepackage{microtype}
\UseMicrotypeSet[protrusion]{basicmath} % disable protrusion for tt fonts
}{}
\usepackage[margin=1in]{geometry}
\usepackage{hyperref}
\hypersetup{unicode=true,
            pdftitle={Ciência de Dados para Todos (Data Science For All) - 2018.1 - Análise da Produção Científica e Acadêmica da Universidade de Brasília - Modelo de Relatório Final da Disciplina - Departamento de Ciência da Computação da UnB},
            pdfauthor={Antônio Henrique de Moura Rodrigues, Breno Felipe Nunes Gomes, Ícaro Nery Rezende},
            pdfborder={0 0 0},
            breaklinks=true}
\urlstyle{same}  % don't use monospace font for urls
\usepackage{color}
\usepackage{fancyvrb}
\newcommand{\VerbBar}{|}
\newcommand{\VERB}{\Verb[commandchars=\\\{\}]}
\DefineVerbatimEnvironment{Highlighting}{Verbatim}{commandchars=\\\{\}}
% Add ',fontsize=\small' for more characters per line
\usepackage{framed}
\definecolor{shadecolor}{RGB}{248,248,248}
\newenvironment{Shaded}{\begin{snugshade}}{\end{snugshade}}
\newcommand{\KeywordTok}[1]{\textcolor[rgb]{0.13,0.29,0.53}{\textbf{#1}}}
\newcommand{\DataTypeTok}[1]{\textcolor[rgb]{0.13,0.29,0.53}{#1}}
\newcommand{\DecValTok}[1]{\textcolor[rgb]{0.00,0.00,0.81}{#1}}
\newcommand{\BaseNTok}[1]{\textcolor[rgb]{0.00,0.00,0.81}{#1}}
\newcommand{\FloatTok}[1]{\textcolor[rgb]{0.00,0.00,0.81}{#1}}
\newcommand{\ConstantTok}[1]{\textcolor[rgb]{0.00,0.00,0.00}{#1}}
\newcommand{\CharTok}[1]{\textcolor[rgb]{0.31,0.60,0.02}{#1}}
\newcommand{\SpecialCharTok}[1]{\textcolor[rgb]{0.00,0.00,0.00}{#1}}
\newcommand{\StringTok}[1]{\textcolor[rgb]{0.31,0.60,0.02}{#1}}
\newcommand{\VerbatimStringTok}[1]{\textcolor[rgb]{0.31,0.60,0.02}{#1}}
\newcommand{\SpecialStringTok}[1]{\textcolor[rgb]{0.31,0.60,0.02}{#1}}
\newcommand{\ImportTok}[1]{#1}
\newcommand{\CommentTok}[1]{\textcolor[rgb]{0.56,0.35,0.01}{\textit{#1}}}
\newcommand{\DocumentationTok}[1]{\textcolor[rgb]{0.56,0.35,0.01}{\textbf{\textit{#1}}}}
\newcommand{\AnnotationTok}[1]{\textcolor[rgb]{0.56,0.35,0.01}{\textbf{\textit{#1}}}}
\newcommand{\CommentVarTok}[1]{\textcolor[rgb]{0.56,0.35,0.01}{\textbf{\textit{#1}}}}
\newcommand{\OtherTok}[1]{\textcolor[rgb]{0.56,0.35,0.01}{#1}}
\newcommand{\FunctionTok}[1]{\textcolor[rgb]{0.00,0.00,0.00}{#1}}
\newcommand{\VariableTok}[1]{\textcolor[rgb]{0.00,0.00,0.00}{#1}}
\newcommand{\ControlFlowTok}[1]{\textcolor[rgb]{0.13,0.29,0.53}{\textbf{#1}}}
\newcommand{\OperatorTok}[1]{\textcolor[rgb]{0.81,0.36,0.00}{\textbf{#1}}}
\newcommand{\BuiltInTok}[1]{#1}
\newcommand{\ExtensionTok}[1]{#1}
\newcommand{\PreprocessorTok}[1]{\textcolor[rgb]{0.56,0.35,0.01}{\textit{#1}}}
\newcommand{\AttributeTok}[1]{\textcolor[rgb]{0.77,0.63,0.00}{#1}}
\newcommand{\RegionMarkerTok}[1]{#1}
\newcommand{\InformationTok}[1]{\textcolor[rgb]{0.56,0.35,0.01}{\textbf{\textit{#1}}}}
\newcommand{\WarningTok}[1]{\textcolor[rgb]{0.56,0.35,0.01}{\textbf{\textit{#1}}}}
\newcommand{\AlertTok}[1]{\textcolor[rgb]{0.94,0.16,0.16}{#1}}
\newcommand{\ErrorTok}[1]{\textcolor[rgb]{0.64,0.00,0.00}{\textbf{#1}}}
\newcommand{\NormalTok}[1]{#1}
\usepackage{longtable,booktabs}
\usepackage{graphicx,grffile}
\makeatletter
\def\maxwidth{\ifdim\Gin@nat@width>\linewidth\linewidth\else\Gin@nat@width\fi}
\def\maxheight{\ifdim\Gin@nat@height>\textheight\textheight\else\Gin@nat@height\fi}
\makeatother
% Scale images if necessary, so that they will not overflow the page
% margins by default, and it is still possible to overwrite the defaults
% using explicit options in \includegraphics[width, height, ...]{}
\setkeys{Gin}{width=\maxwidth,height=\maxheight,keepaspectratio}
\IfFileExists{parskip.sty}{%
\usepackage{parskip}
}{% else
\setlength{\parindent}{0pt}
\setlength{\parskip}{6pt plus 2pt minus 1pt}
}
\setlength{\emergencystretch}{3em}  % prevent overfull lines
\providecommand{\tightlist}{%
  \setlength{\itemsep}{0pt}\setlength{\parskip}{0pt}}
\setcounter{secnumdepth}{0}
% Redefines (sub)paragraphs to behave more like sections
\ifx\paragraph\undefined\else
\let\oldparagraph\paragraph
\renewcommand{\paragraph}[1]{\oldparagraph{#1}\mbox{}}
\fi
\ifx\subparagraph\undefined\else
\let\oldsubparagraph\subparagraph
\renewcommand{\subparagraph}[1]{\oldsubparagraph{#1}\mbox{}}
\fi

%%% Use protect on footnotes to avoid problems with footnotes in titles
\let\rmarkdownfootnote\footnote%
\def\footnote{\protect\rmarkdownfootnote}

%%% Change title format to be more compact
\usepackage{titling}

% Create subtitle command for use in maketitle
\newcommand{\subtitle}[1]{
  \posttitle{
    \begin{center}\large#1\end{center}
    }
}

\setlength{\droptitle}{-2em}

  \title{Ciência de Dados para Todos (Data Science For All) - 2018.1 - Análise da
Produção Científica e Acadêmica da Universidade de Brasília - Modelo de
Relatório Final da Disciplina - Departamento de Ciência da Computação da
UnB}
    \pretitle{\vspace{\droptitle}\centering\huge}
  \posttitle{\par}
    \author{Antônio Henrique de Moura Rodrigues, Breno Felipe Nunes Gomes, Ícaro
Nery Rezende}
    \preauthor{\centering\large\emph}
  \postauthor{\par}
      \predate{\centering\large\emph}
  \postdate{\par}
    \date{17/11/2019}


\begin{document}
\maketitle

\section{Introdução}\label{introducao}

O documento a seguir apresenta o material desenvolvido pelo grupo 11 da
disciplica de Tópicos avançados em Computadores - Turma D - 2018.2, do
Departamento de Ciência da Computação da Univerdade de Brasília. A
proposta do relatório final da disciplina é a verificação e tratamento
dos dados dos programas de pós-graduação da Universidade de Brasília,
com a finalidade de se obter resultados quanto ao progresso de cada
área, através da realização de análises das respectivas produções
científicas e acadêmicas.

A metodologia utilizada durante a execução do relatório foi a CRISP-DM
(um acrônimo para Cross Industry Standard Process for Data Mining),
muito utilizada nas áreas relacionadas a ciência de dados e afins.

\section{CRISP-DM (Corresponderia à seção de
Metodologia)}\label{crisp-dm-corresponderia-a-secao-de-metodologia}

Para desenvolvimento do trabalho devem ser seguidos, da forma mais
simplificada e coerente possível, as fases e atividades genéricas do
ciclo de vida de um projeto executado em aderência ao CRISP-DM, conforme
ilustra de forma geral a figura 1. Em outras palavras, a produção do
relatorio deve seguir a metodologia CRISP-DM.

Fases do Ciclo de Vida do CRISM-DM. Fonte: (Chapman et al., 2000).

Perceba que a Figura 1 sugere haver grande flexibilidade na execução das
fases, de modo que se pode retornar a fases anteriores em muitos pontos.

``A widely used methodology for data mining is the CRoss-Industry
Standard Process for Data Mining (CRISP-DM) which mas initiated in 1996
(\ldots{}) with the intent of providing a process that is
\textbf{reliable and repeatable} by people with little data-mining
background, with a framework within which experience can be recorded, to
support the replication of projects, to support planning and management,
as well as to demonstrate data mining as a mature discipline
(\ldots{})'' {[}Sullivan, Rob. Introduction to Data Mining for the Life
Sciences. Springer Science \& Business Media. 2012{]}

O seu trabalho deve conter uma seção metodologia, onde você faz uma
breve descrição da metodologia que seu grupo adotou para realização do
trabalho, que pode ser baseada no texto dessa seção, desde que citado
adequadamente.

\subsection{Delimitações iniciais}\label{delimitacoes-iniciais}

Em aderência à estrutura do CRISP-DM, algumas delimitações de contexto
para o trabalho são apresentadas a seguir:

\subsubsection{Domínio de Aplicação do
projeto}\label{dominio-de-aplicacao-do-projeto}

O domínio de aplicação do projeto é o da produção científica e acadêmica
brasileira, mais específicamente a produção científica ou produção
acadêmica de um subgrupo de pesquisadores vinculados à Universidade de
Brasília. O domínio de aplicação do projeto foi do Colégio de
Humanidades, na grande área de Ciências Humanas, a área de Sociologia.
Na grande área de Ciências Sociais Aplicadas, as áreas de Comunicação e
Informação, e de Serviço Social.

\subsubsection{Tipo de Problema
abordado}\label{tipo-de-problema-abordado}

O tipo de problema abordado é o da produção de análises descritivas,
quantitativas e de modelagem computacional ou estatística, que permitam
caracterizar como e porque ocorre a produção científica e acadêmica de
um grupo de pesquisadores. Essa caracterização visa subsidiar a tomada
de decisão por membros do Sistema Nacional de Pós-Graduação.

\subsubsection{Conjunto de Ferramentas e
Técnicas}\label{conjunto-de-ferramentas-e-tecnicas}

O conjunto de ferramentas utilizada na confecção deste:

\begin{itemize}
\tightlist
\item
  A linguagem R e suas dependências, para manipulação dos dados;
\item
  Funções das bibliotecas de ciência de dados em R propostas por Wickham
  e Grolemund (2016);
\item
  Abordagem baseada em mineração de texto, para melhor analisar os
  dados;
\item
  Metodologia CRISP-DM;
\item
  O ambiente RStudio, para utilização das ferramentas citadas
  anteriormente.
\end{itemize}

\subsection{Modelo de Referência
CRISP-DM}\label{modelo-de-referencia-crisp-dm}

Miner (2012), aprofunda: ``(\ldots{}) In CRISP-DM, the complete life
cycle of a data mining project is represented with \textbf{six phases}:
business understanding (determining the purpose of the study), data
understanding (data exploration and understanding), data preparation,
modeling, evaluation, and deployment.(\ldots{}). {[}Miner, Gary.
Practical Text Mining and Statistical Analysis for Non-structure Text
Data Applications. Academic Press, 2012.{]}

\subsubsection{Por que usar o CRISP-DM?}\label{por-que-usar-o-crisp-dm}

Imagine uma analogia entre um projeto de datamining e a preparação de
uma receita de bolo para ser usada em uma fábrica. Para iniciar a
produção, com base numa receita de comprovada eficácia (metodológica e
científica), você tem que minerar os ingredientes (dados) em um grande
supermercado (\emph{dataset}). Com os ingredientes você precisa aplicar
um método (a forma de misturá-los), colocar os ingredientes numa
determinada ordem, mexer por um certo tempo, aquecer por tantos minutos
até o bolo ficar pronto e ser aprovado em um ou mais testes de
degustação.

Tendo por objetivo fazer com que essa receita (script de mineração de
dados) possa ser executada com sucesso diversas vezes, numa fábrica,
será que outro cozinheiro (cientista) que reproduzisse a receita
(método) chegaria ao mesmo resultado? Se a metodologia (receita) já foi
bastante testada, então é bem provável que o resultado será o mesmo e
seu produto (receita de bolo) será aceito para a produção
(\emph{deployment}) de análises para consumo futuro, com base em
fundamentos científicos.

\subsubsection{Organização hierárquica de atividades em
fases}\label{organizacao-hierarquica-de-atividades-em-fases}

Dentro de cada fase no CRISP-DM existe uma estrutura hierárquica de
atividades genéricas para serem realizadas. Cada uma dessas atividades
\textbf{genéricas} pode determinar a execução de atividades
\textbf{específicas}.

Voltando ao exemplo do bolo, a atividade '' 1. Entendimento do Bolo''
poderia conter uma atividade genérica chamada ``1.1. Determinar para que
o bolo servirá (simples café da manhã? bolo de aniversário? bolo de
casamento?)``. Dentro dessa atividade genérica poderia haver atividades
específicas como ``1.1.1.Entrevistar o contratante para obter detalhes
de onde o bolo será usado?``; ``1.1.2. Conversar com os convidados sob
alguma necessidade especial (sem lactose? sem glútem?)``, etc.

\subsubsection{Seis Fases do CRISP-DM}\label{seis-fases-do-crisp-dm}

Com base no apresentado, segue uma descrição um pouco mais detalhada das
seis fases de um projeto no CRISP-DM, interpretadas no contexto do
relatório que você e seu grupo deverão produzir.

Todas as fases deverão ser adequadamente relatadas no relatório, em
seções que aparecem após a seçãoda metodologia

\begin{enumerate}
\def\labelenumi{\arabic{enumi}.}
\tightlist
\item
  O propósito da fase de \textbf{Entendimento do Negócio} é o
  desenvolvimento dos objetivos e declaração das necessidades do projeto
  sob a perspectiva do negócio, para transformar isso tudo em definição
  de um problema de data mining.
\end{enumerate}

As atividades genéricas dentro dessa fase envolvem:

\begin{itemize}
\item
  Identificar o que a organização realmente necessita alcançar. No caso
  específico desta disciplina, a necessidade do Sistema Nacional de
  Pós-Graduação do Brasil de produzir análises de alta qualidade de suas
  pós-graduações, com baixo custo. Como produzir um projeto de mineração
  de dados se você não sabe o que necessita encontrar ou resolver? Se
  você não entender os objetivos da organização pode levar ao erro de
  procurar as respostas certas para as perguntas erradas.
\item
  Avaliação das Circunstâncias. Envolve identificar quais recursos ou
  dificuldades podem influenciar os objetivos da mineração ou do projeto
  em si. No caso específico desta disciplina, isso envolve refletir,
  entre vários outros aspectos, sobre as limitações de tempo do projeto,
  que precisa ser realizado dentro de um semestre letivo, de modo que
  considerável parte das atividades já foram pré-organizadas pelos
  docentes responsáveis pela disciplina.
\item
  O projeto de mineração é o grande objetivo desta etapa e o relatório
  precisa conter uma seção sobre Metodologia, apresentando em detalhes o
  que se pretende fazer adiante.
\end{itemize}

\begin{enumerate}
\def\labelenumi{\arabic{enumi}.}
\setcounter{enumi}{1}
\tightlist
\item
  A fase de \textbf{Entendimento dos Dados} inicia determinando quais
  são os dados realmente disponíveis na organização, se existe permissão
  para utilizá-los, se existem dados confidenciais ou cobertos pelo
  sigilo. Por exemplo, um \emph{dataset} das declarações de imposto de
  renda da Receita Federal certamente seria protegido pelo sigilo
  fiscal. Dados de pacientes de hospitais podem conter restrições.
\end{enumerate}

Também é necessário acessar os dados para compreendê-los melhor para ter
o \emph{insight} de como será feita a modelagem mais tarde.

Na fase de entendimento dos dados pode-se trabalhar com quatro
atividades genéricas:

\begin{itemize}
\item
  Coleta inicial dos dados. Essa atividade envolve a análise das
  permissões de acesso e outras questões envolvendo sigilo e outros
  proprietários dos dados (terceiros). Por exemplo, eu poderia estar
  acessando uma base de dados que foi obtida de outro órgão por
  convênio, mas nesse convênio (contrato) não foi dada permissão para
  qualquer outro tipo de acesso ou exploração dos dados. Neste projeto,
  a coleta inicial foi feita pelos autores deste relatório. O relatório
  final deve conter indicações de como foi realizada a coleta inicial
  dos dados.
\item
  Descrição dos dados. A descrição dos dados verifica se os dados sendo
  acessados terão potencial para responder às questões de \emph{data
  mining}. Além disso, deve-se avaliar qual o volume de dados, a
  estrutura dos dados (tipos), codificações usadas, etc. Neste projeto,
  a descrição dos dados é responsabilidade parcial dos alunos, tendo em
  vista que este modelo já oferece uma descrição inicial. O relatório
  final deve conter descrições significativas e aprofundadas dos dados.
\item
  Análise exploratória dos dados. A análise exploratória dos dados
  possibilita um entendimento mais profundo da relação estatística
  existente entre os dados dos \emph{datasets} para um melhor
  entendimento da qualidade daqueles dados para o objetivo do projeto.
  Neste projeto, a análise exploratória dos dados é responsabilidade
  parcial dos alunos, tendo em vista que este relatório apresenta uma
  análise exploratória preliminar. O relatório final deve conter
  análises exploratórias dos dados que sejam significativas e
  aprofundadas.
\item
  Verificação da qualidade dos dados. A verificação da qualidade dos
  dados envolve responder se os dados disponíveis estão realmente
  completos. As informações disponíveis são suficientes para o trabalho
  proposto? Neste projeto, a verificação da qualidade dos dados é
  responsabilidade dos alunos.
\end{itemize}

\begin{enumerate}
\def\labelenumi{\arabic{enumi}.}
\setcounter{enumi}{2}
\tightlist
\item
  Na fase de \textbf{Preparação dos Dados} os \emph{datasets} que serão
  utilizados em todo o trabalho são construídos a partir dos dados
  brutos. Aqui os dados são ``filtrados'' retirando-se partes que não
  interessam e selecionando-se os ``campos'' necessários para o trabalho
  de mineração.
\end{enumerate}

São 5 as atividades genéricas nesta fase de preparação dos dados:

\begin{itemize}
\item
  Seleção dos dados. Envolve identificar quais dados, da nossa
  ``montanha de dados'', serão realmente utilizados. Quais variáveis dos
  dados brutos serão convertidas para o \emph{dataset}? Não é raro
  cometer o erro de selecionar dados para um modelo preditivo com base
  em uma falsa ideia de que aqueles dados contém a resposta para o
  modelo que se quer construir. Surge o cuidado de se separar o sinal do
  ruído (Silver, Nate. The Signal and the Noise: Why so many predictions
  fail --- but some don't. USA: The Penguin Press HC, 2012.).
\item
  Limpeza dos dados.
\item
  Construção dos dados. Envolve a criação de novas variáveis a partir de
  outras presentes nos \emph{datasets}.
\item
  Integração dos dados. Envolve a união (merge) de diferentes tabelas
  para criar um único \emph{dataset} para ser utilizado no R, por
  exemplo.
\item
  Formatação dos dados. Envolve a realização de pequenas alterações na
  estrutura dos dados, como a ordem das variáveis, para permitir a
  execução de determinado método de data mining.
\end{itemize}

\begin{enumerate}
\def\labelenumi{\arabic{enumi}.}
\setcounter{enumi}{3}
\tightlist
\item
  A fase de \textbf{Modelagem} no CRISP-DM envolve a construção e
  avaliação do modelo, podendo ser realizada em quatro atividades
  genéricas:
\end{enumerate}

\begin{itemize}
\item
  Seleção das técnicas de modelagem.
\item
  Realização de testes de modelagem, onde diferentes modelos são
  previamente testados e avaliados. Pode-se dividir o \emph{dataset}
  criado na etapa anterior para se ter uma base de treino na construção
  de modelos, e outra pequena parte para validar e avaliar a eficiência
  de cada modelo criado até se chegar ao mais ``eficiente''.
\item
  Construção do modelo definitivo, com base na melhor experiência do
  passo anterior.
\item
  Avaliação do modelo.
\end{itemize}

\begin{enumerate}
\def\labelenumi{\arabic{enumi}.}
\setcounter{enumi}{4}
\tightlist
\item
  Na fase de \textbf{Avaliação} do CRISP-DM os resultados não são apenas
  avaliados, mas se verifica se existem questões relacionadas à
  organização que não foram suficientemente abordadas. Deve-se refletir
  se o uso arepetido do modelo criado pode trazer algum ``efeito
  colateral'' para a organização.
\end{enumerate}

Nesta fase, pode-se trabalhar com 3 atividades genéricas:

\begin{itemize}
\item
  Avaliação dos resultados
\item
  Revisão do processo, por meio da qual verifica-se se o modelo foi
  construído adequadamente. As variáveis (passadas) para construir o
  modelo estarão disponíveis no futuro?
\item
  Determinação dos etapas seguintes. Pode ser necessário decidir-se por
  finalizar o projeto, passar à etapa de desenvolvimento, ou rever
  algumas fases anteriores para a melhoria do projeto.
\end{itemize}

\begin{enumerate}
\def\labelenumi{\arabic{enumi}.}
\setcounter{enumi}{5}
\tightlist
\item
  Na fase de \textbf{Implantação} (\emph{deployment}) se realiza o
  planejamento de implantação dos produtos desenvolvidos (scripts, no
  caso do executado nesta disciplina) para o ambiente operacional, para
  seu uso repetitivo, envolvendo atividades de monitoramento e
  manutenção do sistema (script) desenvolvido. A fase de implantação
  concluir com a produção e apresentação do relatório final com os
  resultados do projeto.
\end{enumerate}

São atividades genéricas na fase de \textbf{implantação}:

\begin{itemize}
\tightlist
\item
  Planejamento da transição dos produtos;
\item
  Planejamento do monitoramento dos produtos em utilização no ambiente
  operacional;
\item
  Planejamento de manuteção a ser eventualmente efetuada no produto
  (scripts);
\item
  Produção do relatório final;
\item
  Apresentação do relatório final;
\item
  Revisão sobre a execução do projeto, com registro de lições aprendidas
  etc.
\end{itemize}

No contexto do projeto realizado no âmbito desta disciplina, a
responsabilidade por execução de todas essas atividades é dos alunos,
com exceção da apresentação do relatório final, que não será realizada.

\section{CRISP-DM Fase 1 - Entendimento do
Negócio}\label{crisp-dm-fase-1---entendimento-do-negocio}

\subsection{O que é o Sistema Nacional de Pós-Graduação?
(Contextualização)}\label{o-que-e-o-sistema-nacional-de-pos-graduacao-contextualizacao}

A produção do conhecimento científico, no Brasil, é predominantemente
efetuada por meio do Sistema Nacional de Pós-Graduação - SNPG, e mais
fortemente relacionada com a formação de doutores nesse sistema (Pátaro
e Mezzomo, 2013), por meio de cursos de pós-graduação \emph{strictu
sensu}.

Fernandes e Sampaio (2017) já indicaram que a ciência é reconhecidamente
um elemento essencial para o desenvolvimento social e econômico de
qualquer nação. Assim sendo, faz-se mister aprimorar o SNPG como forma
de promoção desse crescimento, visando maximizar o retorno decorrente do
emprego dos recursos nele aplicados. A promoção do crescimento do SNPG
se dá predominantemente por meio de avaliações regulares de seus
programas de pós-graduação, sob responsabilidade da CAPES, que realiza a
cada quatro anos um complexo (Leite, 2018, p.~13) e custoso processo de
coleta de dados, análise e deliberação sobre as pós-graduações
\emph{strictu sensu}, em coerência com o estabelecido no Plano Nacional
de Pós-Graduação (PNPG) 2012-2020 (CAPES, 2010) e nos diversos
documentos que definem os critérios de organização da pós-graduação em
cada área do conhecimento (CAPES, 2018). Leite (2018) faz uma
apresentação geral de como se organizam e são avaliadas as
pós-graduações no Brasil.

O Plano Nacional de Pós-Graduação (PNPG), por outro lado, define
diretrizes estratégicas para desenvolvimento da pós-graduação
brasileira, que deve abordar prioritariamente grandes temas de interesse
nacional, tais como a redução das assimetrias de desenvolvimento entre
as regiões do Brasil, a formação de professores para a educação básica,
a formação de recursos humanos para as empresas, a resposta aos grandes
desafios brasileiros sobre Água, Energia, Transporte, Controle de
Fronteiras, Agronegócio, Amazônia, Amazônia Azul (Mar), Saúde, Defesa,
Programa Espacial, além de Justiça, Segurança Pública, Criminologia e
Desequilíbrio Regional. O PNPG também traça as diretrizes para
financiamento da pós-graduação e sua internacionalização, apresentando
conclusões e recomendações.

As avaliações do SNPG, ao atribuirem mensurações de desempenho às
diversas pós-graduações que dele fazem parte, geram incentivos e
penalidades aos programas, tendo em vista a limitada disponibilidade de
recursos para investimento em bolsas, taxas de bancada etc. Embora o
sistema seja altamente sofisticado ele é também altamente criticado
(Azevedo et al., 2016), sobretudo porque há percalços na busca por um
equilíbrio entre as diferentes concepções de finalidade da ciência. Se
de um lado a promoção do conhecimento gerado predominantemente nas ditas
ciências \emph{hard} constribui para criar fluxos econômicos mais
intensos, isso não significa que essa promoção possa ocorrer em
detrimento da menor promoção na geração de conhecimento sobre problemas
sociais, predominantemente gerado nas ditas ciências \emph{soft},
especialmente das áreas de humanidades, sob pena de ampliação de
desigualdades (Azevedo et al., 2016).

Não há solução simples, mas postula-se, nesta disciplina, que uma maior
agilidade na avaliação e a utilização de critérios mais objetivos,
poderá facilitar a melhoria do sistema.

\subsubsection{Os Colégios, Grandes Áreas e Áreas da Pós-Graduação
Brasileira}\label{os-colegios-grandes-areas-e-areas-da-pos-graduacao-brasileira}

A partir de 2018, as diversas áreas da pós-graduação brasileira foram
organizadas na forma de colégios, grandes áreas e áreas, conforme
apresentam as tabelas a seguir.

\paragraph{Colégio de Ciências da
vida}\label{colegio-de-ciencias-da-vida}

\begin{longtable}[]{@{}lll@{}}
\toprule
CIÊNCIAS AGRÁRIAS & CIÊNCIAS BIOLÓGICAS & CIÊNCIAS DA
SAÚDE\tabularnewline
\midrule
\endhead
Ciência de Alimentos & Biodiversidade & Educação Física\tabularnewline
Ciências Agrárias I & Ciências Biológicas I & Enfermagem\tabularnewline
Medicina Veterinária & Ciências Biológicas II & Farmácia\tabularnewline
Zootecnia / Recursos Pesqueiros & Ciências Biológicas III & Medicina
I\tabularnewline
- & - & Medicina II\tabularnewline
- & - & Medicina III\tabularnewline
- & - & Nutrição\tabularnewline
- & - & Odontologia\tabularnewline
- & - & Saúde Coletiva\tabularnewline
\bottomrule
\end{longtable}

\paragraph{Colégio de Ciências Exatas, Tecnológicas e
Multidisciplinar}\label{colegio-de-ciencias-exatas-tecnologicas-e-multidisciplinar}

\begin{longtable}[]{@{}lll@{}}
\toprule
CIÊNCIAS EXATAS E DA TERRA & ENGENHARIAS &
MULTIDISCIPLINAR\tabularnewline
\midrule
\endhead
Astronomia / Física & Engenharias I & Biotecnologia\tabularnewline
Ciência da Computação & Engenharias II & Ciências
Ambientais\tabularnewline
Geociências & Engenharias III & Ensino\tabularnewline
Matemática / Probabilidade e Estatística & Engenharias IV &
Interdisciplinar\tabularnewline
Química & - & Materiais\tabularnewline
\bottomrule
\end{longtable}

\paragraph{Colégio de Humanidades}\label{colegio-de-humanidades}

\begin{longtable}[]{@{}lll@{}}
\toprule
CIÊNCIAS HUMANAS & CIÊNCIAS SOCIAIS APLICADAS & LINGUÍSTICA, LETRAS E
ARTES\tabularnewline
\midrule
\endhead
Antropol/Arqueol & Admin.Púb./Empr.,C.Contáb. e Tur. &
Artes\tabularnewline
Ciência Pol. e Rel. Int. & Arquit., Urban. e Design & Linguística e
Literatura\tabularnewline
Ciências da Religião e Teol. & Comunicação e Informação &
-\tabularnewline
Educação & Direito & -\tabularnewline
Filosofia & Economia & -\tabularnewline
Geografia & Planej. Urbano e Reg. / Demografia & -\tabularnewline
História & Serviço Social & -\tabularnewline
Psicologia & - & -\tabularnewline
Sociologia & - & -\tabularnewline
\bottomrule
\end{longtable}

Cada um desses colégios, grandes áreas e áreas de conhecimento possuem
dinâmicas próprias, e, portanto, não há um modelo universal que se
aplique a todas. Existem aspectos comuns, mas também grandes
peculiaridades, descritas parcialmente nos correspondentes documentos de
área disponíveis em CAPES (2018).

\subsection{A UnB dentro do Sistema Nacional de Pós-Graduação
(Contextualização)}\label{a-unb-dentro-do-sistema-nacional-de-pos-graduacao-contextualizacao}

\subsubsection{O que é a UnB?}\label{o-que-e-a-unb}

A Universidade de Brasília, atualmente, oferece dois tipos de programa
de pós-graduação, a pós-graduação acadêmica e a pós-graduação
profissional. Quanto à rpodução acadêmica, atualmente, são
contabilizados mais de 27 mil artigos publicados ou aceitos e mais de 23
mil orientações acadêmicas.

\subsubsection{Descrição das pós-graduações da
UnB}\label{descricao-das-pos-graduacoes-da-unb}

Como responsabilidade do grupo, foram alocados 4 programas para serem
analisados. Os programas, com seus subprogramas relativos são:

\begin{itemize}
\tightlist
\item
  Pós-Graduação em Comunicação Possui uma área de concentração,
  Comunicação e Sociedade, que concentra as 4 linhas de pesquisa do
  programa:

  \begin{itemize}
  \tightlist
  \item
    Jornalismo e Sociedade
  \item
    Políticas de Comunicação e Cultura
  \item
    Teorias e Tecnologias de Comunicação
  \item
    Imagem, Som e Escrita
  \end{itemize}
\item
  Pós-Graduação em Ciência da Informação Possui uma área de
  concentração, Gestão da Informação, que concetra 2 linhas de pesquisa:

  \begin{itemize}
  \tightlist
  \item
    Comunicação e Mediação da Informação
  \item
    Organização da Informação
  \end{itemize}
\item
  Pós-Graduação em Política Social Possui uma área de concentração,
  Estado, Políticas Sociais e Direitos, que abrange 4 linhas de
  pesquisa:

  \begin{itemize}
  \tightlist
  \item
    Política Social, Estado e Sociedade
  \item
    Classes,Lutas sociais e Direitos
  \item
    Trabalho, Questão Social e Emancipação
  \item
    Exploração e Opressão de Sexo/genero, Raça/Etnia e Sexualidades.
  \end{itemize}
\item
  Pós-Graduação em Sociologia Possui uma área de concentração, Sociedade
  e Transformação, e 7 linhas de pesquisa:

  \begin{itemize}
  \tightlist
  \item
    Cidade, Cultura e Sociedade
  \item
    Educação, Ciência e Tecnologia
  \item
    Feminismo, Relações de Gênero e de Raça
  \item
    Pensamento e Teoria Social
  \item
    Política, Valores, Religião e Sociedade
  \item
    Trabalho e Sociedade
  \item
    Violência, Segurança e Cidadania
  \end{itemize}
\end{itemize}

\subsubsection{Outros aspectos que caracterizam a produção científica e
acadêmica da
UnB}\label{outros-aspectos-que-caracterizam-a-producao-cientifica-e-academica-da-unb}

Os programas de pós-graduação da Universidade de Brasília possuem, em
geral uma boa avaliação quanto às pontuações atribuídas. Os programas
considerados de excelência, que possuem nota 6 ou 7, possuem produções
em outros idiomas, o que sugere uma conexão com centros de pesquisa
internacionais.

\subsection{O que a Organização precisa realmente
alcançar?}\label{o-que-a-organizacao-precisa-realmente-alcancar}

Vários stakeholders estão envolvidos no projeto em curso, e poderíamos
considerar cada um deles como distintas organizações que possuem
interesses distintos e complementares. Elas são: * A Disciplina Ciência
de Dados para Todos 2018.1, que quer comprovar que seus alunos dominam
ferramentas e técnicas de ciência de dados, para fins de avaliação de
rendimento da disciplina. * A UnB, representada pelos decanatos de
pós-graduação (DPG) e de pesquisa e inovação (DPI), que querem dispor de
instrumentos para realização de avaliações contínuas de suas
pós-graduações. * O SNPG, que assim com o DPG e DPI, também pode se
beneficiar do uso de instrumentos para realização de avaliações
contínuas de suas pós-graduações. * Os interessados em melhor conhecer o
que é produzido pelo Sistema Nacional de Pós-graduação, como empresas
privadas, que querem desfrutar dos benefícios gerados pela ciência
brasileira.

A fim de dar maior fidelidade e homogeneidade ao exercício realizado na
disciplina, focaremos em atendimento aos interesses comuns das
organizações DPI, DPG e CAPES, que desejam dispor de instrumentos ágeis
para avaliação contínua da pós-graduação brasileira.

Com base no exposto, o objetivo do trabalho final a ser alcançado pelos
produtos d emineração de dados desenvolvido pelos alunos da disciplina
Ciência de Dados para Todos é produzir, tomando por base inicial os
dados fornecidos pelos professores responsáveis pela disciplina,
ferramentas para análise e avaliação contínuas e de baixo custo, do
desempenho de um conjunto de pós-graduações que estão vinculadas a uma
mesma subárea ou grupo de conhecimento. Cada área de pós-graduação
apresenta suas características peculiares, assim como cada um dos
programas vinculados a essas áreas. Como já informado, características
peculiares de cada programa podem ser obtidas a partir de visita ao
sítio da CAPES (2018).

\subsection{Avaliação das
Circunstâncias}\label{avaliacao-das-circunstancias}

Este documento serve como base para a realização dos trabalhos dos
alunos. apresenta limitações no tocante à quantidade pequena de dados
que serão empregados para análises e avaliações, tendo em vista sua
finalidade maior que é a didática, de permitir aos alunos demonstrarem a
capacidade de aplicação das técnicas e ferramentas apreendidas durante o
semestre.

\subsubsection{Avaliação preliminar das pós-graduações na
UnB}\label{avaliacao-preliminar-das-pos-graduacoes-na-unb}

\paragraph{Pós Graduação em Ciência da
Informação}\label{pos-graduacao-em-ciencia-da-informacao}

Na última avaliação, feita em 2017, obteve nota 5. Possui 27 docentes,
383 artigos publicados, sendo que 68 foram realizadas em 2017. \#\#\#\#
Pós Graduação em Comunicação Na última avaliação, feita em 2017, obteve
nota 4. Possui 31 docentes, 309 artigos publicados, sendo que 36 foram
realizadas em 2017. \#\#\#\# Pós Graduação em Política Social Na última
avaliação, feita em 2017, obteve nota 6. Possui 21 docentes, 192 artigos
publicados, sendo que 25 foram realizadas em 2017. \#\#\#\# Pós
Graduação em Sociologia Na última avaliação, feita em 2017, obteve nota
7. Possui 27 docentes, 253 artigos publicados, sendo que 28 foram
realizadas em 2017.

\subsubsection{Avaliação preliminar da produção científica e acadêmica
da
UnB}\label{avaliacao-preliminar-da-producao-cientifica-e-academica-da-unb}

Texto a desenvolver.

\section{CRISP-DM Fase 2 - Entendimento dos
Dados}\label{crisp-dm-fase-2---entendimento-dos-dados}

Doravante, a fim de facilitar aos alunos seguirem a metodologia
CRISP-DM, os nomes das seções e subseções de texto serão prefixadas com
o número e nome da fase e atividade genérica do CRISP-DM. Fica facultado
aos grupos seguir ou não a sequência prevista, tendo em vista que se
pode retornar às fases anteriores, bem como podem haver atividades que
não foram adequadas às características do problema específico sob
análise.

\subsection{CRISP-DM Fase.Atividade 2.1 - Coleta inicial dos
dados}\label{crisp-dm-fase.atividade-2.1---coleta-inicial-dos-dados}

Todos os arquivos com dados iniciais a seguir apresentados foram
fornecidos pelos professores responsáveis pela disciplina. Os dados
foram gerados no mês de maio de 2018, e compilam informações entre os
anos de 2010 e 2017. Os arquivos estão no formato JSON, e seus atributos
iniciais e conteúdos são apresentados a seguir.

\subsubsection{Perfil profissional dos docentes vinculados às
pós-graduações}\label{perfil-profissional-dos-docentes-vinculados-as-pos-graduacoes}

\begin{Shaded}
\begin{Highlighting}[]
\NormalTok{json.perfil <-}\StringTok{ "unbpos/unbpos.profile.json"}
\KeywordTok{file.info}\NormalTok{(json.perfil)}
\end{Highlighting}
\end{Shaded}

\begin{verbatim}
##                                size isdir mode               mtime
## unbpos/unbpos.profile.json 75162725 FALSE  666 2018-09-14 22:25:06
##                                          ctime               atime  uid
## unbpos/unbpos.profile.json 2018-11-17 21:29:10 2018-11-18 23:32:37 1000
##                             gid uname grname
## unbpos/unbpos.profile.json 1000 icaro  icaro
\end{verbatim}

O arquivo unbpos/unbpos.profile.json apresenta dados sobre o perfil de
todos os docentes vinculados a programas de pós-graduação da UnB, entre
2010 e 2017. Esse arquivo foi fornecido pelos docentes responsáveis pela
disciplina.

\begin{Shaded}
\begin{Highlighting}[]
\NormalTok{json.ci.perfil <-}\StringTok{ "ciencia_informacao/ci.profile.json"}
\KeywordTok{file.info}\NormalTok{(json.ci.perfil)}
\end{Highlighting}
\end{Shaded}

\begin{verbatim}
##                                       size isdir mode               mtime
## ciencia_informacao/ci.profile.json 1126228 FALSE  666 2018-06-21 01:20:00
##                                                  ctime               atime
## ciencia_informacao/ci.profile.json 2018-11-16 15:18:59 2018-11-18 23:32:46
##                                     uid  gid uname grname
## ciencia_informacao/ci.profile.json 1000 1000 icaro  icaro
\end{verbatim}

\begin{Shaded}
\begin{Highlighting}[]
\NormalTok{json.co.perfil <-}\StringTok{ "comunicacao/co.profile.json"}
\KeywordTok{file.info}\NormalTok{(json.co.perfil)}
\end{Highlighting}
\end{Shaded}

\begin{verbatim}
##                                size isdir mode               mtime
## comunicacao/co.profile.json 1247316 FALSE  666 2018-06-21 01:35:36
##                                           ctime               atime  uid
## comunicacao/co.profile.json 2018-11-16 15:20:36 2018-11-18 00:27:01 1000
##                              gid uname grname
## comunicacao/co.profile.json 1000 icaro  icaro
\end{verbatim}

\begin{Shaded}
\begin{Highlighting}[]
\NormalTok{json.ps.perfil <-}\StringTok{ "politica_social/ps.profile.json"}
\KeywordTok{file.info}\NormalTok{(json.ps.perfil)}
\end{Highlighting}
\end{Shaded}

\begin{verbatim}
##                                   size isdir mode               mtime
## politica_social/ps.profile.json 848798 FALSE  666 2018-06-21 02:14:28
##                                               ctime               atime
## politica_social/ps.profile.json 2018-11-16 15:21:18 2018-11-18 02:51:37
##                                  uid  gid uname grname
## politica_social/ps.profile.json 1000 1000 icaro  icaro
\end{verbatim}

\begin{Shaded}
\begin{Highlighting}[]
\NormalTok{json.so.perfil <-}\StringTok{ "sociologia/so.profile.json"}
\KeywordTok{file.info}\NormalTok{(json.so.perfil)}
\end{Highlighting}
\end{Shaded}

\begin{verbatim}
##                              size isdir mode               mtime
## sociologia/so.profile.json 793037 FALSE  666 2018-06-21 02:24:24
##                                          ctime               atime  uid
## sociologia/so.profile.json 2018-11-16 15:19:44 2018-11-18 02:51:37 1000
##                             gid uname grname
## sociologia/so.profile.json 1000 icaro  icaro
\end{verbatim}

Da mesma forma, os arquivos \texttt{json.ci.perfil},
\texttt{json.co.perfil}, \texttt{json.ps.perfil} e
\texttt{json.so.perfil}, apresentam, respectivamente, os dados sobre o
perfil dos docentes vinculados programas de pós-graduação em Ciência da
Informação, Comunicação, Política Social e Sociologia.

\subsubsection{Orientações de mestrado e doutorado realizadas pelos
docentes vinculados às
pós-graduações}\label{orientacoes-de-mestrado-e-doutorado-realizadas-pelos-docentes-vinculados-as-pos-graduacoes}

\begin{Shaded}
\begin{Highlighting}[]
\NormalTok{json.advise <-}\StringTok{ "unbpos/unbpos.advise.json"}
\KeywordTok{file.info}\NormalTok{(json.advise)}
\end{Highlighting}
\end{Shaded}

\begin{verbatim}
##                               size isdir mode               mtime
## unbpos/unbpos.advise.json 29828920 FALSE  666 2018-09-14 22:24:56
##                                         ctime               atime  uid
## unbpos/unbpos.advise.json 2018-11-17 21:29:10 2018-11-18 23:32:47 1000
##                            gid uname grname
## unbpos/unbpos.advise.json 1000 icaro  icaro
\end{verbatim}

O arquivo unbpos/unbpos.advise.json apresenta dados sobre o orientações
de mestrado e doutorado feitas por todos os docentes vinculados a
programas de pós-graduação da UnB, entre 2010 e 2017. Esse arquivo foi
fornecido pelos docentes responsáveis pela disciplina.

\begin{Shaded}
\begin{Highlighting}[]
\NormalTok{json.ci.advise <-}\StringTok{ "ciencia_informacao/ci.advise.json"}
\KeywordTok{file.info}\NormalTok{(json.ci.advise)}
\end{Highlighting}
\end{Shaded}

\begin{verbatim}
##                                     size isdir mode               mtime
## ciencia_informacao/ci.advise.json 526741 FALSE  666 2018-06-21 01:20:00
##                                                 ctime               atime
## ciencia_informacao/ci.advise.json 2018-11-16 15:18:45 2018-11-18 23:32:49
##                                    uid  gid uname grname
## ciencia_informacao/ci.advise.json 1000 1000 icaro  icaro
\end{verbatim}

\begin{Shaded}
\begin{Highlighting}[]
\NormalTok{json.co.advise <-}\StringTok{ "comunicacao/co.advise.json"}
\KeywordTok{file.info}\NormalTok{(json.co.advise)}
\end{Highlighting}
\end{Shaded}

\begin{verbatim}
##                              size isdir mode               mtime
## comunicacao/co.advise.json 569439 FALSE  666 2018-06-21 01:35:36
##                                          ctime               atime  uid
## comunicacao/co.advise.json 2018-11-16 15:20:23 2018-11-18 02:51:37 1000
##                             gid uname grname
## comunicacao/co.advise.json 1000 icaro  icaro
\end{verbatim}

\begin{Shaded}
\begin{Highlighting}[]
\NormalTok{json.ps.advise <-}\StringTok{ "politica_social/ps.advise.json"}
\KeywordTok{file.info}\NormalTok{(json.ps.advise)}
\end{Highlighting}
\end{Shaded}

\begin{verbatim}
##                                  size isdir mode               mtime
## politica_social/ps.advise.json 452241 FALSE  666 2018-06-21 02:14:28
##                                              ctime               atime
## politica_social/ps.advise.json 2018-11-16 15:21:03 2018-11-18 02:51:37
##                                 uid  gid uname grname
## politica_social/ps.advise.json 1000 1000 icaro  icaro
\end{verbatim}

\begin{Shaded}
\begin{Highlighting}[]
\NormalTok{json.so.advise <-}\StringTok{ "sociologia/so.advise.json"}
\KeywordTok{file.info}\NormalTok{(json.so.advise)}
\end{Highlighting}
\end{Shaded}

\begin{verbatim}
##                             size isdir mode               mtime
## sociologia/so.advise.json 358097 FALSE  666 2018-06-21 02:24:24
##                                         ctime               atime  uid
## sociologia/so.advise.json 2018-11-16 15:19:27 2018-11-18 02:51:37 1000
##                            gid uname grname
## sociologia/so.advise.json 1000 icaro  icaro
\end{verbatim}

Da mesma forma, os arquivos \texttt{json.ci.advise},
\texttt{json.co.advise}, \texttt{json.ps.advise} e
\texttt{json.so.advise}, apresentam, respectivamente, os dados sobre as
orientações de mestrado e doutorado feitas pelos docentes vinculados
programas de pós-graduação em Ciência da Informação, Comunicação,
Política Social e Sociologia.

\subsubsection{Produção bibliográfica gerada pelos docentes vinculados
às
pós-graduações}\label{producao-bibliografica-gerada-pelos-docentes-vinculados-as-pos-graduacoes}

\begin{Shaded}
\begin{Highlighting}[]
\NormalTok{json.producao.bibliografica <-}\StringTok{ "unbpos/unbpos.publication.json"}
\KeywordTok{file.info}\NormalTok{(json.producao.bibliografica) }
\end{Highlighting}
\end{Shaded}

\begin{verbatim}
##                                    size isdir mode               mtime
## unbpos/unbpos.publication.json 33546293 FALSE  666 2018-09-14 22:25:14
##                                              ctime               atime
## unbpos/unbpos.publication.json 2018-11-17 21:29:10 2018-11-18 23:32:49
##                                 uid  gid uname grname
## unbpos/unbpos.publication.json 1000 1000 icaro  icaro
\end{verbatim}

O arquivo unbpos/unbpos.publication.json apresenta dados sobre a
produção bibliográfica gerada por todos os docentes vinculados a
programas de pós-graduação da UnB, entre 2010 e 2017.

\begin{Shaded}
\begin{Highlighting}[]
\NormalTok{json.ci.producao.bibliografica <-}\StringTok{ "ciencia_informacao/ci.publication.json"}
\KeywordTok{file.info}\NormalTok{(json.ci.producao.bibliografica)}
\end{Highlighting}
\end{Shaded}

\begin{verbatim}
##                                          size isdir mode
## ciencia_informacao/ci.publication.json 465994 FALSE  666
##                                                      mtime
## ciencia_informacao/ci.publication.json 2018-06-21 01:20:00
##                                                      ctime
## ciencia_informacao/ci.publication.json 2018-11-16 15:19:03
##                                                      atime  uid  gid uname
## ciencia_informacao/ci.publication.json 2018-11-18 23:32:51 1000 1000 icaro
##                                        grname
## ciencia_informacao/ci.publication.json  icaro
\end{verbatim}

\begin{Shaded}
\begin{Highlighting}[]
\NormalTok{json.co.producao.bibliografica <-}\StringTok{ "comunicacao/co.publication.json"}
\KeywordTok{file.info}\NormalTok{(json.co.producao.bibliografica)}
\end{Highlighting}
\end{Shaded}

\begin{verbatim}
##                                   size isdir mode               mtime
## comunicacao/co.publication.json 539523 FALSE  666 2018-06-21 01:35:36
##                                               ctime               atime
## comunicacao/co.publication.json 2018-11-16 15:20:40 2018-11-18 02:51:37
##                                  uid  gid uname grname
## comunicacao/co.publication.json 1000 1000 icaro  icaro
\end{verbatim}

\begin{Shaded}
\begin{Highlighting}[]
\NormalTok{json.ps.producao.bibliografica <-}\StringTok{ "politica_social/ps.publication.json"}
\KeywordTok{file.info}\NormalTok{(json.ps.producao.bibliografica)}
\end{Highlighting}
\end{Shaded}

\begin{verbatim}
##                                       size isdir mode               mtime
## politica_social/ps.publication.json 299198 FALSE  666 2018-06-21 02:14:28
##                                                   ctime
## politica_social/ps.publication.json 2018-11-16 15:21:29
##                                                   atime  uid  gid uname
## politica_social/ps.publication.json 2018-11-18 02:51:37 1000 1000 icaro
##                                     grname
## politica_social/ps.publication.json  icaro
\end{verbatim}

\begin{Shaded}
\begin{Highlighting}[]
\NormalTok{json.so.producao.bibliografica <-}\StringTok{ "sociologia/so.publication.json"}
\KeywordTok{file.info}\NormalTok{(json.so.producao.bibliografica)}
\end{Highlighting}
\end{Shaded}

\begin{verbatim}
##                                  size isdir mode               mtime
## sociologia/so.publication.json 331814 FALSE  666 2018-06-21 02:24:24
##                                              ctime               atime
## sociologia/so.publication.json 2018-11-16 15:19:48 2018-11-18 02:51:37
##                                 uid  gid uname grname
## sociologia/so.publication.json 1000 1000 icaro  icaro
\end{verbatim}

Da mesma forma, os arquivos \texttt{json.ci.producao.bibliografica},
\texttt{json.co.producao.bibliografica},
\texttt{json.ps.producao.bibliografica} e
\texttt{json.so.producao.bibliografica}, apresentam, respectivamente, os
dados sobre a produção bibliográfica gerada pelos docentes vinculados
programas de pós-graduação em Ciência da Informação, Comunicação,
Política Social e Sociologia.

\subsubsection{Agrupamento dos docentes conforme áreas de
atuação}\label{agrupamento-dos-docentes-conforme-areas-de-atuacao}

\begin{Shaded}
\begin{Highlighting}[]
\NormalTok{json.researchers_by_area <-}\StringTok{ "unbpos/unbpos.researchers_by_area.json"} 
\KeywordTok{file.info}\NormalTok{(json.researchers_by_area)}
\end{Highlighting}
\end{Shaded}

\begin{verbatim}
##                                         size isdir mode
## unbpos/unbpos.researchers_by_area.json 64366 FALSE  666
##                                                      mtime
## unbpos/unbpos.researchers_by_area.json 2018-09-14 22:25:14
##                                                      ctime
## unbpos/unbpos.researchers_by_area.json 2018-11-17 21:29:10
##                                                      atime  uid  gid uname
## unbpos/unbpos.researchers_by_area.json 2018-11-18 23:32:51 1000 1000 icaro
##                                        grname
## unbpos/unbpos.researchers_by_area.json  icaro
\end{verbatim}

O arquivo unbpos/unbpos.researchers\_by\_area.json apresenta as
vinculações de todos os docentes que declararam atuar em cada uma das
áreas de pós-graduação do Sistema Nacional de Pós-Graduação da CAPES,
conforme apresenta-se registrada essa informação no currículo Lattes de
cada um, em data recente.

\subsubsection{Redes de colaboração entre
docentes}\label{redes-de-colaboracao-entre-docentes}

\begin{Shaded}
\begin{Highlighting}[]
\KeywordTok{file.info}\NormalTok{(}\StringTok{'unbpos/unbpos.graph.json'}\NormalTok{)}
\end{Highlighting}
\end{Shaded}

\begin{verbatim}
##                            size isdir mode               mtime
## unbpos/unbpos.graph.json 503798 FALSE  666 2018-09-14 22:24:58
##                                        ctime               atime  uid  gid
## unbpos/unbpos.graph.json 2018-11-17 21:29:10 2018-11-17 22:08:15 1000 1000
##                          uname grname
## unbpos/unbpos.graph.json icaro  icaro
\end{verbatim}

O arquivo \texttt{unbpos/unbpos.graph.json} apresenta redes de
colaboração na co-autoria de artigos científicos, feitas entre os
docentes vinculados a programas de pós-graduação da UnB, entre 2010 e
2017.

\begin{Shaded}
\begin{Highlighting}[]
\KeywordTok{file.info}\NormalTok{(}\StringTok{'ciencia_informacao/ci.graph.json'}\NormalTok{)}
\end{Highlighting}
\end{Shaded}

\begin{verbatim}
##                                  size isdir mode               mtime
## ciencia_informacao/ci.graph.json 3722 FALSE  666 2018-06-21 01:20:00
##                                                ctime               atime
## ciencia_informacao/ci.graph.json 2018-11-16 15:18:51 2018-11-17 22:25:10
##                                   uid  gid uname grname
## ciencia_informacao/ci.graph.json 1000 1000 icaro  icaro
\end{verbatim}

\begin{Shaded}
\begin{Highlighting}[]
\KeywordTok{file.info}\NormalTok{(}\StringTok{'comunicacao/co.graph.json'}\NormalTok{)}
\end{Highlighting}
\end{Shaded}

\begin{verbatim}
##                           size isdir mode               mtime
## comunicacao/co.graph.json 4409 FALSE  666 2018-06-21 01:35:36
##                                         ctime               atime  uid
## comunicacao/co.graph.json 2018-11-16 15:20:28 2018-11-18 02:51:37 1000
##                            gid uname grname
## comunicacao/co.graph.json 1000 icaro  icaro
\end{verbatim}

\begin{Shaded}
\begin{Highlighting}[]
\KeywordTok{file.info}\NormalTok{(}\StringTok{'politica_social/ps.graph.json'}\NormalTok{)}
\end{Highlighting}
\end{Shaded}

\begin{verbatim}
##                               size isdir mode               mtime
## politica_social/ps.graph.json 2647 FALSE  666 2018-06-21 02:14:28
##                                             ctime               atime  uid
## politica_social/ps.graph.json 2018-11-16 15:21:07 2018-11-18 02:51:37 1000
##                                gid uname grname
## politica_social/ps.graph.json 1000 icaro  icaro
\end{verbatim}

\begin{Shaded}
\begin{Highlighting}[]
\KeywordTok{file.info}\NormalTok{(}\StringTok{'sociologia/so.graph.json'}\NormalTok{)}
\end{Highlighting}
\end{Shaded}

\begin{verbatim}
##                          size isdir mode               mtime
## sociologia/so.graph.json 3180 FALSE  666 2018-06-21 02:24:24
##                                        ctime               atime  uid  gid
## sociologia/so.graph.json 2018-11-16 15:19:33 2018-11-18 02:51:37 1000 1000
##                          uname grname
## sociologia/so.graph.json icaro  icaro
\end{verbatim}

Do mesmo modo, esses arquivos apresentam redes de colaboração na
co-autoria de artigos. Representando, respectivamente, os programas de
pós-graduação em Ciência da Informação, Comunicação, Política Social e
Sociologia.

\subsection{CRISP-DM Fase.Atividade 2.2 - Descrição dos
Dados}\label{crisp-dm-fase.atividade-2.2---descricao-dos-dados}

Para ler e manipular inicialmente esses dados, serão usadas
primordialmente as bibliotecas seguintes

\begin{Shaded}
\begin{Highlighting}[]
\KeywordTok{library}\NormalTok{(jsonlite)}
\KeywordTok{library}\NormalTok{(listviewer)}
\KeywordTok{library}\NormalTok{(readxl)}
\KeywordTok{library}\NormalTok{(readr)}
\KeywordTok{library}\NormalTok{(readtext)}
\KeywordTok{library}\NormalTok{(ggplot2)}
\KeywordTok{library}\NormalTok{(tidyverse)}
\KeywordTok{library}\NormalTok{(stringr)}
\end{Highlighting}
\end{Shaded}

Como já informado, a descrição dos dados verifica se os dados sendo
acessados terão potencial para responder às questões de \emph{data
mining}. Além disso, deve-se avaliar qual o volume de dados, a estrutura
dos dados (tipos), codificações usadas, etc. Neste projeto, a descrição
dos dados é responsabilidade parcial dos alunos, tendo em vista que esta
seção já oferece uma descrição inicial simplificada. O relatório final
deve conter descrições significativas e aprofundadas dos dados.

\subsubsection{Descrição dos dados do
perfil}\label{descricao-dos-dados-do-perfil}

O arquivo unb.perfis.json, que contém dados que caracterizam o perfil
profissional de todos os docentes do grupo sob análise, podem ser lido
por meio do comando seguinte.

\begin{Shaded}
\begin{Highlighting}[]
\NormalTok{unb.prof <-}\StringTok{ }\KeywordTok{fromJSON}\NormalTok{(}\StringTok{"unbpos/unbpos.profile.json"}\NormalTok{)}
\end{Highlighting}
\end{Shaded}

O mesmo pode ser replicado para cada uma das áreas de interesse.

\begin{Shaded}
\begin{Highlighting}[]
\NormalTok{ci.prof <-}\StringTok{ }\KeywordTok{fromJSON}\NormalTok{(}\StringTok{"./ciencia_informacao/ci.profile.json"}\NormalTok{)}
\NormalTok{co.prof <-}\StringTok{ }\KeywordTok{fromJSON}\NormalTok{(}\StringTok{"./comunicacao/co.profile.json"}\NormalTok{)}
\NormalTok{ps.prof <-}\StringTok{ }\KeywordTok{fromJSON}\NormalTok{(}\StringTok{"./politica_social/ps.profile.json"}\NormalTok{)}
\NormalTok{so.prof <-}\StringTok{ }\KeywordTok{fromJSON}\NormalTok{(}\StringTok{"./sociologia/so.profile.json"}\NormalTok{)}
\end{Highlighting}
\end{Shaded}

A quantidade de docentes sob análise é apresentada a seguir: Quantidade
de docentes na pós-graduação da UnB.

\begin{Shaded}
\begin{Highlighting}[]
\KeywordTok{length}\NormalTok{(unb.prof)}
\end{Highlighting}
\end{Shaded}

\begin{verbatim}
## [1] 1764
\end{verbatim}

Quantidade em cada um dos programas.

\begin{Shaded}
\begin{Highlighting}[]
\CommentTok{# ciencia da informacao}
\KeywordTok{length}\NormalTok{(ci.prof)}
\end{Highlighting}
\end{Shaded}

\begin{verbatim}
## [1] 27
\end{verbatim}

\begin{Shaded}
\begin{Highlighting}[]
\CommentTok{# comunicacao}
\KeywordTok{length}\NormalTok{(co.prof)}
\end{Highlighting}
\end{Shaded}

\begin{verbatim}
## [1] 31
\end{verbatim}

\begin{Shaded}
\begin{Highlighting}[]
\CommentTok{# politica social}
\KeywordTok{length}\NormalTok{(ps.prof)}
\end{Highlighting}
\end{Shaded}

\begin{verbatim}
## [1] 21
\end{verbatim}

\begin{Shaded}
\begin{Highlighting}[]
\CommentTok{# sociologia}
\KeywordTok{length}\NormalTok{(so.prof)}
\end{Highlighting}
\end{Shaded}

\begin{verbatim}
## [1] 27
\end{verbatim}

Para gerar uma apresentação inicial dos dados que estão contido nos
dados de perfil dos docentes, pode-se usar a função glimpse, da
biblioteca dplyr, como ilustra o código seguinte, que apresenta os
atributos típicos que podem ser obtidos relativamente a um pesquisador
específico, o mais antigo docente ainda em exercício na UnB a ter criado
seu registro na plataforma Lattes.

\begin{Shaded}
\begin{Highlighting}[]
\KeywordTok{glimpse}\NormalTok{(unb.prof[[}\DecValTok{1}\NormalTok{]], }\DataTypeTok{width =} \DecValTok{30}\NormalTok{)}
\end{Highlighting}
\end{Shaded}

\begin{verbatim}
## List of 7
##  $ nome                  : chr "Norai Romeu Rocco"
##  $ resumo_cv             : chr "Possui graduação em Matemática (licenciatura plena) pela Universidade Estadual Paulista Júlio de Mesquita Filho"| __truncated__
##  $ areas_de_atuacao      :'data.frame':  5 obs. of  4 variables:
##   ..$ grande_area  : chr [1:5] "CIENCIAS_EXATAS_E_DA_TERRA" "CIENCIAS_EXATAS_E_DA_TERRA" "CIENCIAS_EXATAS_E_DA_TERRA" "CIENCIAS_EXATAS_E_DA_TERRA" ...
##   ..$ area         : chr [1:5] "Matemática" "Matemática" "Matemática" "Ciência da Computação" ...
##   ..$ sub_area     : chr [1:5] "" "Álgebra" "Álgebra" "Matemática da Computação" ...
##   ..$ especialidade: chr [1:5] "" "" "Grupos de Álgebra Não-Comutaviva" "Matemática Simbólica" ...
##  $ endereco_profissional :List of 8
##   ..$ instituicao: chr "Universidade de Brasília"
##   ..$ orgao      : chr "Instituto de Ciências Exatas"
##   ..$ unidade    : chr "Departamento de Matemática"
##   ..$ DDD        : chr "061"
##   ..$ telefone   : chr "31076442"
##   ..$ bairro     : chr "Asa Norte"
##   ..$ cep        : chr "70910900"
##   ..$ cidade     : chr "Brasília"
##  $ producao_bibiografica :List of 4
##   ..$ ARTIGO_ACEITO                         :'data.frame':   1 obs. of  10 variables:
##   .. ..$ natureza        : chr "NAO_INFORMADO"
##   .. ..$ titulo          : chr "Finiteness conditions for the non-abelian tensor product of groups"
##   .. ..$ periodico       : chr "MONATSHEFTE FUR MATHEMATIK"
##   .. ..$ ano             : chr "2017"
##   .. ..$ volume          : chr ""
##   .. ..$ issn            : chr "00269255"
##   .. ..$ paginas         : chr " - "
##   .. ..$ doi             : chr "10.1007/s00605-017-1143-x"
##   .. ..$ autores         :List of 1
##   .. ..$ autores-endogeno:List of 1
##   ..$ DEMAIS_TIPOS_DE_PRODUCAO_BIBLIOGRAFICA:'data.frame':   7 obs. of  9 variables:
##   .. ..$ natureza          : chr [1:7] "DIVULGAÇÃO DE RESULTADOS DE PESQUISA" "DIVULGAÇÃO DE RESULTADOS DE PESQUISA" "DIVULGAÇÃO DE RESULTADOS DE PESQUISA" "DIVULGAÇÃO DE RESULTADOS DE PESQUISA" ...
##   .. ..$ titulo            : chr [1:7] "NON-ABELIAN TENSOR SQUARE OF FINITE-BY-NILPOTENT GROUPS" "The q-tensor square of finitely generated nilpotent groups, q >=0" "The q-tensor square of finitely generated nilpotent groups, q >=0" "THE NON-ABELIAN TENSOR SQUARE OF RESIDUALLY FINITE GROUPS" ...
##   .. ..$ ano               : chr [1:7] "2015" "2016" "2016" "2016" ...
##   .. ..$ pais_de_publicacao: chr [1:7] "Estados Unidos" "Estados Unidos" "Estados Unidos" "Estados Unidos" ...
##   .. ..$ editora           : chr [1:7] "" "ArXiv.com - Cornell University Library" "ArXiv.com - Cornell University Library" "ArXiv.com - Cornell University Library" ...
##   .. ..$ doi               : chr [1:7] "" "" "" "" ...
##   .. ..$ numero_de_paginas : chr [1:7] "8" "12" "12" "11" ...
##   .. ..$ autores           :List of 7
##   .. ..$ autores-endogeno  :List of 7
##   ..$ EVENTO                                :'data.frame':   1 obs. of  11 variables:
##   .. ..$ natureza        : chr "RESUMO"
##   .. ..$ titulo          : chr "On Semidirect Products and non-abelian Tensor Products of Groups"
##   .. ..$ nome_do_evento  : chr "XIX Colóquio Latinoamericano de Álgebra"
##   .. ..$ ano_do_trabalho : chr "2012"
##   .. ..$ pais_do_evento  : chr "Chile"
##   .. ..$ cidade_do_evento: chr "Pucón - Chile"
##   .. ..$ doi             : chr ""
##   .. ..$ classificacao   : chr "INTERNACIONAL"
##   .. ..$ paginas         : chr " - "
##   .. ..$ autores         :List of 1
##   .. ..$ autores-endogeno:List of 1
##   ..$ PERIODICO                             :'data.frame':   6 obs. of  10 variables:
##   .. ..$ natureza        : chr [1:6] "COMPLETO" "COMPLETO" "COMPLETO" "COMPLETO" ...
##   .. ..$ titulo          : chr [1:6] "On the q-tensor square of a group" "A survey of non-abelian tensor products of groups and related constructions" "The q-tensor square of finitely generated nilpotent groups, q odd" "Non-abelian tensor square of finite-by-nilpotent groups" ...
##   .. ..$ periodico       : chr [1:6] "Journal of Group Theory" "Boletim da Sociedade Paranaense de Matemática" "JOURNAL OF ALGEBRA AND ITS APPLICATIONS" "Archiv der Mathematik (Printed ed.)" ...
##   .. ..$ ano             : chr [1:6] "2011" "2012" "2016" "2016" ...
##   .. ..$ volume          : chr [1:6] "14" "30" "16" "107" ...
##   .. ..$ issn            : chr [1:6] "14335883" "21751188" "02194988" "0003889X" ...
##   .. ..$ paginas         : chr [1:6] "785 - 805" "77 - 89" "1750211 - " "127 - 133" ...
##   .. ..$ doi             : chr [1:6] "10.1515/JGT.2010.084" "10.5269/bspm.v30i1.13350" "10.1142/S0219498817502115" "10.1007/s00013-016-0930-2" ...
##   .. ..$ autores         :List of 6
##   .. ..$ autores-endogeno:List of 6
##  $ orientacoes_academicas:List of 3
##   ..$ ORIENTACAO_CONCLUIDA_DOUTORADO   :'data.frame':    3 obs. of  13 variables:
##   .. ..$ natureza                   : chr [1:3] "Tese de doutorado" "Tese de doutorado" "Tese de doutorado"
##   .. ..$ titulo                     : chr [1:3] "Cotas superiores para o expoente e o número mínimo de geradores do quadrado q-tensorial de grupos nilpotentes, q geq 0." "Uma Apresentação Policíclica para o Quadrado q-Tensorial de um Grupo Policíclico" "Quadrado Tensorial Não-Abeliano de p-Grupos Finitos com Subgrupo Derivado de Ordem p, p ímpar"
##   .. ..$ ano                        : chr [1:3] "2011" "2011" "2017"
##   .. ..$ id_lattes_aluno            : chr [1:3] "9037151037918091" "8664599889120339" "0723203301483174"
##   .. ..$ nome_aluno                 : chr [1:3] "Eunice Cândida Pereira Rodrigues" "Ivonildes Ribeiro Martins" "Cleilton Aparecido Canal"
##   .. ..$ instituicao                : chr [1:3] "Universidade de Brasília" "Universidade de Brasília" "Universidade de Brasília"
##   .. ..$ curso                      : chr [1:3] "Matemática" "Matemática" "Matemática"
##   .. ..$ codigo_do_curso            : chr [1:3] "51500035" "51500035" "51500035"
##   .. ..$ bolsa                      : chr [1:3] "SIM" "SIM" "NAO"
##   .. ..$ agencia_financiadora       : chr [1:3] "Fundação de Amparo à Pesquisa do Estado de Mato Grosso" "Conselho Nacional de Desenvolvimento Científico e Tecnológico" ""
##   .. ..$ codigo_agencia_financiadora: chr [1:3] "035600000004" "002200000000" ""
##   .. ..$ nome_orientadores          :List of 3
##   .. ..$ id_lattes_orientadores     :List of 3
##   ..$ ORIENTACAO_CONCLUIDA_MESTRADO    :'data.frame':    3 obs. of  13 variables:
##   .. ..$ natureza                   : chr [1:3] "Dissertação de mestrado" "Dissertação de mestrado" "Dissertação de mestrado"
##   .. ..$ titulo                     : chr [1:3] "Algumas Cotas Súperiores para aordem do Quadrado Tensorial não abeliano de um Grupo" "O Grau de Permutabilidade de Subgrupos de um Grupo Finito" "Sobre pE-grupos e pA-grupos finitos"
##   .. ..$ ano                        : chr [1:3] "2010" "2011" "2012"
##   .. ..$ id_lattes_aluno            : chr [1:3] "5367744818899315" "" "0121355793029434"
##   .. ..$ nome_aluno                 : chr [1:3] "Bruno Cesar Rodrigues Lima" "Mônica Aparecida Crunivel Valadão" "Marina Gabriella Ribeiro Bardella"
##   .. ..$ instituicao                : chr [1:3] "Universidade de Brasília" "Universidade de Brasília" "Universidade de Brasília"
##   .. ..$ curso                      : chr [1:3] "Matemática" "Matemática" "Matemática"
##   .. ..$ codigo_do_curso            : chr [1:3] "51500035" "51500035" "51500035"
##   .. ..$ bolsa                      : chr [1:3] "NAO" "SIM" "SIM"
##   .. ..$ agencia_financiadora       : chr [1:3] "" "Coordenação de Aperfeiçoamento de Pessoal de Nível Superior" "Coordenação de Aperfeiçoamento de Pessoal de Nível Superior"
##   .. ..$ codigo_agencia_financiadora: chr [1:3] "" "045000000000" "045000000000"
##   .. ..$ nome_orientadores          :List of 3
##   .. ..$ id_lattes_orientadores     :List of 3
##   ..$ ORIENTACAO_EM_ANDAMENTO_DOUTORADO:'data.frame':    1 obs. of  13 variables:
##   .. ..$ natureza                   : chr "Tese de doutorado"
##   .. ..$ titulo                     : chr "Quadrado Tensorial não Abeliano de certas classes de Grupos Finitos"
##   .. ..$ ano                        : chr "2014"
##   .. ..$ id_lattes_aluno            : chr "1933036212945705"
##   .. ..$ nome_aluno                 : chr "Juliana Silva Canella"
##   .. ..$ instituicao                : chr "Universidade de Brasília"
##   .. ..$ curso                      : chr "Matemática"
##   .. ..$ codigo_do_curso            : chr "51500035"
##   .. ..$ bolsa                      : chr "SIM"
##   .. ..$ agencia_financiadora       : chr "Coordenação de Aperfeiçoamento de Pessoal de Nível Superior"
##   .. ..$ codigo_agencia_financiadora: chr "045000000000"
##   .. ..$ nome_orientadores          :List of 1
##   .. ..$ id_lattes_orientadores     :List of 1
##  $ senioridade           : chr "8"
\end{verbatim}

Uma breve inspeção visual dos atributos anteriormente apresentados
permite inferir que o pesquisador sob análise:

\begin{itemize}
\tightlist
\item
  Atua predominantemente na área de matemática.
\item
  Trabalha no Instituto de Ciências Exatas da UnB.
\item
  Possui três artigos recentes publicados, além de um aceito para
  publicação.
\item
  Possui uma orientação de doutorado em andamento, iniciada em 2014.
\item
  Foi classificado com senioridade 8.
\end{itemize}

É possível fazer o mesmo com os dados específicos de cada programa de
pós-graduação.

\begin{Shaded}
\begin{Highlighting}[]
\CommentTok{# ciencia da informacao}
\KeywordTok{glimpse}\NormalTok{(ci.prof[[}\DecValTok{3}\NormalTok{]], }\DataTypeTok{width =} \DecValTok{30}\NormalTok{)}
\end{Highlighting}
\end{Shaded}

\begin{verbatim}
## List of 7
##  $ nome                  : chr "André Porto Ancona Lopez"
##  $ resumo_cv             : chr "André Porto Ancona Lopez (apalopez@gmail.com) é Doutor em História Social (2001), mestre em História Social (19"| __truncated__
##  $ areas_de_atuacao      :'data.frame':  6 obs. of  4 variables:
##   ..$ grande_area  : chr [1:6] "CIENCIAS_SOCIAIS_APLICADAS" "CIENCIAS_SOCIAIS_APLICADAS" "CIENCIAS_SOCIAIS_APLICADAS" "CIENCIAS_SOCIAIS_APLICADAS" ...
##   ..$ area         : chr [1:6] "Ciência da Informação" "Ciência da Informação" "Ciência da Informação" "Ciência da Informação" ...
##   ..$ sub_area     : chr [1:6] "Arquivologia" "Acesso à Informação" "Tipologia Documental" "Organização de Arquivos" ...
##   ..$ especialidade: chr [1:6] "Documentos Imagéticos" "" "" "" ...
##  $ endereco_profissional :List of 8
##   ..$ instituicao: chr "Universidade de Brasília"
##   ..$ orgao      : chr "Faculdade de Ciência da Informação"
##   ..$ unidade    : chr ""
##   ..$ DDD        : chr "61"
##   ..$ telefone   : chr "31072633"
##   ..$ bairro     : chr "Asa Norte"
##   ..$ cep        : chr "70910900"
##   ..$ cidade     : chr "Brasília"
##  $ producao_bibiografica :List of 5
##   ..$ CAPITULO_DE_LIVRO                     :'data.frame':   14 obs. of  13 variables:
##   .. ..$ tipo                    : chr [1:14] "Capítulo de livro publicado" "Capítulo de livro publicado" "Capítulo de livro publicado" "Capítulo de livro publicado" ...
##   .. ..$ titulo_do_capitulo      : chr [1:14] "La gestión documental y la transparencia" "Mapeamento das áreas de Biblioteconomia. Ciência da Informação, Arquivologia e Museologia" "Identificação de tipologias documentais em acervos de trabalhadores" "A formação de arquivistas no Brasil: notas para um debate" ...
##   .. ..$ titulo_do_livro         : chr [1:14] "Transparencia en las Américas" "Associação Nacional de Pesquisa e Pós-Graduação em Ciência da Informação - Ancib: reflexão e proposta para dinamização" "Arquivos do mundo dos trabalhadores: coletânea do 2º Seminário Internacional" "Estudos avançados em Arquivologia" ...
##   .. ..$ ano                     : chr [1:14] "2010" "2011" "2012" "2012" ...
##   .. ..$ doi                     : chr [1:14] "" "" "" "" ...
##   .. ..$ pais_de_publicacao      : chr [1:14] "México" "Brasil" "Brasil" "Brasil" ...
##   .. ..$ isbn                    : chr [1:14] "9789685954570" "0000000000" "9788589210348" "9788579832666" ...
##   .. ..$ nome_da_editora         : chr [1:14] "Instituto Federal de Acceso a la Información y Protección de Datos (IFAI)" "Ideia" "CUT ; Arquivo Nacional" "Cultura Acadêmica" ...
##   .. ..$ numero_da_edicao_revisao: chr [1:14] "1" "" "1" "1" ...
##   .. ..$ organizadores           : chr [1:14] "Jacqueline Peschard Mariscal ; Sigrid Arzt Colunga; María Marván Laborde; María Elena Pérez-Jaén Zermeño; Ángel"| __truncated__ "Joana Coeli Ribeiro Garcia; Maria das Graças Targino" "Antonio José Marques; Inez Tereznha Stampa" "Marta Lígia Pomim Valentim" ...
##   .. ..$ paginas                 : chr [1:14] "229 - 258" "55 - 79" "15 - 31" "181 - 196" ...
##   .. ..$ autores                 :List of 14
##   .. ..$ autores-endogeno        :List of 14
##   ..$ DEMAIS_TIPOS_DE_PRODUCAO_BIBLIOGRAFICA:'data.frame':   2 obs. of  9 variables:
##   .. ..$ natureza          : chr [1:2] "Blog" "Blog"
##   .. ..$ titulo            : chr [1:2] "Blog Iberoamericano de Enseñanza Arquivistica Universitária" "DigifotoWeb: repositório digital de materiais fotográficos de arquivo"
##   .. ..$ ano               : chr [1:2] "2010" "2010"
##   .. ..$ pais_de_publicacao: chr [1:2] "Brasil" "Brasil"
##   .. ..$ editora           : chr [1:2] "Google Blogspot" "Google Blogspot"
##   .. ..$ doi               : chr [1:2] "" ""
##   .. ..$ numero_de_paginas : chr [1:2] "" ""
##   .. ..$ autores           :List of 2
##   .. ..$ autores-endogeno  :List of 2
##   ..$ EVENTO                                :'data.frame':   51 obs. of  11 variables:
##   .. ..$ natureza        : chr [1:51] "COMPLETO" "COMPLETO" "COMPLETO" "COMPLETO" ...
##   .. ..$ titulo          : chr [1:51] "Blogs de Diplomática e Tipologia Documental como instrumento de aprendizagem: uma experiencia da Universidade de Brasília" "O uso da folksonomia na organização e preservação do acervo imagético da Confederação Nacional dos Trabalhadore"| __truncated__ "O uso da folksonomia na organização e preservação do acervo imagético da Confederação Nacional dos Trabalhadore"| __truncated__ "A Informação, a Ciência, o Cientista e o Conhecimento" ...
##   .. ..$ nome_do_evento  : chr [1:51] "II Congreso Internacional Comunicación 3.0" "Conference on Technology, Culture and Memory: strategies for preservation and information access" "Conference on Technology, Culture and Memory: strategies for preservation and information access" "X CINFORM - Encontro Nacional de Ensino e Pesquisa em Informação" ...
##   .. ..$ ano_do_trabalho : chr [1:51] "2010" "2011" "2011" "2011" ...
##   .. ..$ pais_do_evento  : chr [1:51] "Espanha" "Brasil" "Brasil" "Brasil" ...
##   .. ..$ cidade_do_evento: chr [1:51] "Salamanca" "Recife" "Recife" "Salvador" ...
##   .. ..$ doi             : chr [1:51] "" "" "" "" ...
##   .. ..$ classificacao   : chr [1:51] "INTERNACIONAL" "INTERNACIONAL" "INTERNACIONAL" "NACIONAL" ...
##   .. ..$ paginas         : chr [1:51] "1086 - 1098" " - " " - " " - " ...
##   .. ..$ autores         :List of 51
##   .. ..$ autores-endogeno:List of 51
##   ..$ LIVRO                                 :'data.frame':   1 obs. of  13 variables:
##   .. ..$ titulo                  : chr "Diretrizes para o desenvolvimento de projetos de cunho científico"
##   .. ..$ ano                     : chr "2010"
##   .. ..$ tipo                    : chr "LIVRO_PUBLICADO"
##   .. ..$ natureza                : chr "TEXTO_INTEGRAL"
##   .. ..$ pais_de_publicacao      : chr "Brasil"
##   .. ..$ isbn                    : chr "0000000000000"
##   .. ..$ doi                     : chr ""
##   .. ..$ nome_da_editora         : chr "UnB: Gestão de Segurança da Informação e Comunicações"
##   .. ..$ numero_da_edicao_revisao: chr "1"
##   .. ..$ numero_de_paginas       : chr "24"
##   .. ..$ numero_de_volumes       : chr "1"
##   .. ..$ autores                 :List of 1
##   .. ..$ autores-endogeno        :List of 1
##   ..$ PERIODICO                             :'data.frame':   26 obs. of  10 variables:
##   .. ..$ natureza        : chr [1:26] "COMPLETO" "COMPLETO" "COMPLETO" "COMPLETO" ...
##   .. ..$ titulo          : chr [1:26] "Blogs como ferramenta de ensino-aprendizagem de diplomática e tipologia documental: uma estratégia didática par"| __truncated__ "Contextualización archivística de documentos fotográficos" "Archivos y ciudadanía: el acceso a la información pública" "Building the archives profession in Brazil" ...
##   .. ..$ periodico       : chr [1:26] "Perspectivas em Gestão & Conhecimento" "Alexandria (Peru)" "Revista General de Información y Documentación" "Comma: international journal on archives" ...
##   .. ..$ ano             : chr [1:26] "2011" "2011" "2011" "2012" ...
##   .. ..$ volume          : chr [1:26] "1" "5" "21" "10" ...
##   .. ..$ issn            : chr [1:26] "2236417X" "19911653" "11321873" "16801865" ...
##   .. ..$ paginas         : chr [1:26] "86 - 99" "3 - 16" "249 - 264" "73 - 83" ...
##   .. ..$ doi             : chr [1:26] "" "" "" "" ...
##   .. ..$ autores         :List of 26
##   .. ..$ autores-endogeno:List of 26
##  $ orientacoes_academicas:List of 7
##   ..$ ORIENTACAO_CONCLUIDA_DOUTORADO              :'data.frame': 5 obs. of  13 variables:
##   .. ..$ natureza                   : chr [1:5] "Tese de doutorado" "Tese de doutorado" "Tese de doutorado" "Tese de doutorado" ...
##   .. ..$ titulo                     : chr [1:5] "Monitoramento informacional de ambiente de negócios na micro e pequena empresa (MPE): estudo do comércio vareji"| __truncated__ "A autenticidade da informação no processo de produção de prova testemunhal no Inquérito Parlamentar" "Os registros imagéticos/fotografias na interação sociocultural e econômica no movimento junino" "Intertextualidade, Ciência da Informação e a criação de sentido em fotografias: o caso de Formiga (MG)" ...
##   .. ..$ ano                        : chr [1:5] "2011" "2014" "2015" "2015" ...
##   .. ..$ id_lattes_aluno            : chr [1:5] "0979364277580186" "7980249454439837" "3445829566458687" "3854885552647613" ...
##   .. ..$ nome_aluno                 : chr [1:5] "Eber Luis Capistrano Martins" "Tarciso Aparecido Higino de Carvalho" "Luiz Carlos Flôres de Assumpção" "Niraldo José do Nascimento" ...
##   .. ..$ instituicao                : chr [1:5] "Universidade de Brasília" "Universidade de Brasília" "Universidade de Brasília" "Universidade de Brasília" ...
##   .. ..$ curso                      : chr [1:5] "Ciências da Informação" "Ciências da Informação" "Ciências da Informação" "Ciências da Informação" ...
##   .. ..$ codigo_do_curso            : chr [1:5] "51500183" "51500183" "51500183" "51500183" ...
##   .. ..$ bolsa                      : chr [1:5] "SIM" "NAO" "NAO" "SIM" ...
##   .. ..$ agencia_financiadora       : chr [1:5] "Conselho Nacional de Desenvolvimento Científico e Tecnológico" "" "" "Conselho Nacional de Desenvolvimento Científico e Tecnológico" ...
##   .. ..$ codigo_agencia_financiadora: chr [1:5] "002200000000" "" "" "002200000000" ...
##   .. ..$ nome_orientadores          :List of 5
##   .. ..$ id_lattes_orientadores     :List of 5
##   ..$ ORIENTACAO_CONCLUIDA_MESTRADO               :'data.frame': 11 obs. of  13 variables:
##   .. ..$ natureza                   : chr [1:11] "Dissertação de mestrado" "Dissertação de mestrado" "Dissertação de mestrado" "Dissertação de mestrado" ...
##   .. ..$ titulo                     : chr [1:11] "O lado invisível da participação política: gestão da informação dos mecanismos digitais de participação polític"| __truncated__ "O impacto da gestão arquivística no processo de produção digital da TV Senado" "Organização e acesso de documentos sonoros na web: uma proposta de relação entre as teorias da Ciência da Infor"| __truncated__ "Registros imagéticos e a sustentabilidade: representações sobre o uso da imagem em projetos de captação de recu"| __truncated__ ...
##   .. ..$ ano                        : chr [1:11] "2010" "2010" "2012" "2013" ...
##   .. ..$ id_lattes_aluno            : chr [1:11] "0077214616801148" "" "7941926847621109" "3445829566458687" ...
##   .. ..$ nome_aluno                 : chr [1:11] "Andrea Sampaio Perna" "Edna de Sousa Carvalho" "Clara Bessa da Costa" "Luiz Carlos Flôres de Assumpção" ...
##   .. ..$ instituicao                : chr [1:11] "Universidade de Brasília" "Universidade de Brasília" "Universidade de Brasília" "Universidade de Brasília" ...
##   .. ..$ curso                      : chr [1:11] "Ciências da Informação" "Ciências da Informação" "Ciências da Informação" "Ciências da Informação" ...
##   .. ..$ codigo_do_curso            : chr [1:11] "51500183" "51500183" "51500183" "51500183" ...
##   .. ..$ bolsa                      : chr [1:11] "NAO" "NAO" "NAO" "SIM" ...
##   .. ..$ agencia_financiadora       : chr [1:11] "" "" "" "Coordenação de Aperfeiçoamento de Pessoal de Nível Superior" ...
##   .. ..$ codigo_agencia_financiadora: chr [1:11] "" "" "" "045000000000" ...
##   .. ..$ nome_orientadores          :List of 11
##   .. ..$ id_lattes_orientadores     :List of 11
##   ..$ ORIENTACAO_CONCLUIDA_POS_DOUTORADO          :'data.frame': 1 obs. of  13 variables:
##   .. ..$ natureza                   : chr "Supervisão de pós-doutorado"
##   .. ..$ titulo                     : chr ""
##   .. ..$ ano                        : chr "2012"
##   .. ..$ id_lattes_aluno            : chr ""
##   .. ..$ nome_aluno                 : chr "Antonia Salvador Benitez"
##   .. ..$ instituicao                : chr "Universidade de Brasília"
##   .. ..$ curso                      : chr ""
##   .. ..$ codigo_do_curso            : chr ""
##   .. ..$ bolsa                      : chr "NAO"
##   .. ..$ agencia_financiadora       : chr ""
##   .. ..$ codigo_agencia_financiadora: chr ""
##   .. ..$ nome_orientadores          :List of 1
##   .. ..$ id_lattes_orientadores     :List of 1
##   ..$ ORIENTACAO_EM_ANDAMENTO_DOUTORADO           :'data.frame': 3 obs. of  13 variables:
##   .. ..$ natureza                   : chr [1:3] "Tese de doutorado" "Tese de doutorado" "Tese de doutorado"
##   .. ..$ titulo                     : chr [1:3] "O estado do conhecimento sobre o tema da fotografia na Ciência da Informação no Brasil: inovações e recorrência"| __truncated__ "Acervos imagéticos e acesso à informação: estudo sobre os documentos relacionados à luta contra a violação dos "| __truncated__ "Fotografias periciais: definição diplomática de documentos imagéticos forenses"
##   .. ..$ ano                        : chr [1:3] "2014" "2015" "2015"
##   .. ..$ id_lattes_aluno            : chr [1:3] "8058407443429697" "6292189136309964" ""
##   .. ..$ nome_aluno                 : chr [1:3] "Alessandra dos Santos Araújo" "Laila Figueiredo Di Pietro" "Edson Freitas Jr."
##   .. ..$ instituicao                : chr [1:3] "Universidade de Brasília" "Universidade de Brasília" "Universidade de Brasília"
##   .. ..$ curso                      : chr [1:3] "Ciências da Informação" "Ciências da Informação" "Ciências da Informação"
##   .. ..$ codigo_do_curso            : chr [1:3] "51500183" "51500183" "51500183"
##   .. ..$ bolsa                      : chr [1:3] "SIM" "SIM" "NAO"
##   .. ..$ agencia_financiadora       : chr [1:3] "Coordenação de Aperfeiçoamento de Pessoal de Nível Superior" "Coordenação de Aperfeiçoamento de Pessoal de Nível Superior" ""
##   .. ..$ codigo_agencia_financiadora: chr [1:3] "045000000000" "045000000000" ""
##   .. ..$ nome_orientadores          :List of 3
##   .. ..$ id_lattes_orientadores     :List of 3
##   ..$ ORIENTACAO_EM_ANDAMENTO_INICIACAO_CIENTIFICA:'data.frame': 1 obs. of  13 variables:
##   .. ..$ natureza                   : chr "Iniciação Científica"
##   .. ..$ titulo                     : chr "A influência da fotografia em revistas na formulação do pensamento político da sociedade civil na Argentina"
##   .. ..$ ano                        : chr "2015"
##   .. ..$ id_lattes_aluno            : chr ""
##   .. ..$ nome_aluno                 : chr "Adriano Casemiro Nogueira Campos de Sousa"
##   .. ..$ instituicao                : chr "Universidade de Brasília"
##   .. ..$ curso                      : chr "Abi - Ciências Sociais"
##   .. ..$ codigo_do_curso            : chr "60662107"
##   .. ..$ bolsa                      : chr "NAO"
##   .. ..$ agencia_financiadora       : chr ""
##   .. ..$ codigo_agencia_financiadora: chr ""
##   .. ..$ nome_orientadores          :List of 1
##   .. ..$ id_lattes_orientadores     :List of 1
##   ..$ ORIENTACAO_EM_ANDAMENTO_MESTRADO            :'data.frame': 3 obs. of  13 variables:
##   .. ..$ natureza                   : chr [1:3] "Dissertação de mestrado" "Dissertação de mestrado" "Dissertação de mestrado"
##   .. ..$ titulo                     : chr [1:3] "A autenticidade arquivística de documentos eletrônicos como subsidio para elaboração de um padrão de logs para "| __truncated__ "Um dimensionamento de capacidade de TI a partir da geração automática de configuração" "Redes Neurais aplicadas aos registros para identificação de padrões em transações"
##   .. ..$ ano                        : chr [1:3] "2015" "2015" "2015"
##   .. ..$ id_lattes_aluno            : chr [1:3] "3572277728003382" "4885748587935162" "7944701818626706"
##   .. ..$ nome_aluno                 : chr [1:3] "Lucas de Sousa Brito" "Odilon de Freitas Militão Neto" "Rodrigo Ribeiro Pereira"
##   .. ..$ instituicao                : chr [1:3] "Universidade de Brasília" "Universidade de Brasília" "Universidade de Brasília"
##   .. ..$ curso                      : chr [1:3] "Ciências da Informação" "Ciências da Informação" "Ciências da Informação"
##   .. ..$ codigo_do_curso            : chr [1:3] "51500183" "51500183" "51500183"
##   .. ..$ bolsa                      : chr [1:3] "NAO" "NAO" "NAO"
##   .. ..$ agencia_financiadora       : chr [1:3] "" "" ""
##   .. ..$ codigo_agencia_financiadora: chr [1:3] "" "" ""
##   .. ..$ nome_orientadores          :List of 3
##   .. ..$ id_lattes_orientadores     :List of 3
##   ..$ OUTRAS_ORIENTACOES_CONCLUIDAS               :'data.frame': 14 obs. of  13 variables:
##   .. ..$ natureza                   : chr [1:14] "TRABALHO_DE_CONCLUSAO_DE_CURSO_GRADUACAO" "INICIACAO_CIENTIFICA" "INICIACAO_CIENTIFICA" "INICIACAO_CIENTIFICA" ...
##   .. ..$ titulo                     : chr [1:14] "Organização de documentos audiovisuais e imagéticos: uma abordagem em diplomática e tipologia documental" "Analise do tratamento arquivístico dado pelo Arquivo Público do Distrito Federal às fotografias sobre a reconst"| __truncated__ "Análise dos efeitos da aplicação da Resolução 14 do CONARQ em documentos imagéticos de arquivo, no Arquivo Públ"| __truncated__ "Panorama arquivístico do tratamento de documentos imagéticos de arquivo pelo Instituto Euvaldo Lodi" ...
##   .. ..$ ano                        : chr [1:14] "2010" "2011" "2011" "2011" ...
##   .. ..$ id_lattes_aluno            : chr [1:14] "" "" "" "" ...
##   .. ..$ nome_aluno                 : chr [1:14] "Nathalia Ferreira de Carvalho" "Kelly Pontes de Souza" "Pedro Davi Silva Carvalho" "Marcella Mendes Gonçalves" ...
##   .. ..$ instituicao                : chr [1:14] "Universidade de Brasília" "Universidade de Brasília" "Universidade de Brasília" "Universidade de Brasília" ...
##   .. ..$ curso                      : chr [1:14] "Biblioteconomia" "Arquivologia" "Arquivologia" "Arquivologia" ...
##   .. ..$ codigo_do_curso            : chr [1:14] "90000022" "90000021" "90000021" "90000021" ...
##   .. ..$ bolsa                      : chr [1:14] "NAO" "NAO" "NAO" "NAO" ...
##   .. ..$ agencia_financiadora       : chr [1:14] "" "" "" "" ...
##   .. ..$ codigo_agencia_financiadora: chr [1:14] "" "" "" "" ...
##   .. ..$ nome_orientadores          :List of 14
##   .. ..$ id_lattes_orientadores     :List of 14
##  $ senioridade           : chr "8"
\end{verbatim}

\begin{Shaded}
\begin{Highlighting}[]
\CommentTok{# comunicacao }
\KeywordTok{glimpse}\NormalTok{(co.prof[[}\DecValTok{7}\NormalTok{]], }\DataTypeTok{width =} \DecValTok{30}\NormalTok{)}
\end{Highlighting}
\end{Shaded}

\begin{verbatim}
## List of 7
##  $ nome                  : chr "Claudia Guilmar Linhares Sanz"
##  $ resumo_cv             : chr "Atualmente, pesquisadora do Zentrum für Literatur- und Kulturforschung (ZfL), em Berlim, desenvolvendo a pesqui"| __truncated__
##  $ areas_de_atuacao      :'data.frame':  5 obs. of  4 variables:
##   ..$ grande_area  : chr [1:5] "CIENCIAS_SOCIAIS_APLICADAS" "CIENCIAS_HUMANAS" "CIENCIAS_HUMANAS" "LINGUISTICA_LETRAS_E_ARTES" ...
##   ..$ area         : chr [1:5] "Comunicação" "Filosofia" "Educação" "Artes" ...
##   ..$ sub_area     : chr [1:5] "" "" "" "" ...
##   ..$ especialidade: chr [1:5] "" "" "" "" ...
##  $ endereco_profissional :List of 8
##   ..$ instituicao: chr "Universidade de Brasília"
##   ..$ orgao      : chr "Faculdade de Educação"
##   ..$ unidade    : chr ""
##   ..$ DDD        : chr "61"
##   ..$ telefone   : chr "31076160"
##   ..$ bairro     : chr "Asa Norte"
##   ..$ cep        : chr "70910900"
##   ..$ cidade     : chr "Brasília"
##  $ producao_bibiografica :List of 4
##   ..$ CAPITULO_DE_LIVRO                     :'data.frame':   3 obs. of  13 variables:
##   .. ..$ tipo                    : chr [1:3] "Capítulo de livro publicado" "Capítulo de livro publicado" "Capítulo de livro publicado"
##   .. ..$ titulo_do_capitulo      : chr [1:3] "A memórias das casas" "Estados fotográficos, fósseis e fantasmas" "A potência da imagem e o acontecimento na escola"
##   .. ..$ titulo_do_livro         : chr [1:3] "Sobrados da Zona Oeste" "Fronteiras e trangressões na fotografia contemporânea" "Estilos de aprendizagem, tecnolog ias e inovações na educação"
##   .. ..$ ano                     : chr [1:3] "2012" "2013" "2013"
##   .. ..$ doi                     : chr [1:3] "" "" ""
##   .. ..$ pais_de_publicacao      : chr [1:3] "Brasil" "Brasil" "Brasil"
##   .. ..$ isbn                    : chr [1:3] "9788562114168" "9788598205854" "9788564593183"
##   .. ..$ nome_da_editora         : chr [1:3] "Editora Olhares" "Casa das Musas" "UnB"
##   .. ..$ numero_da_edicao_revisao: chr [1:3] "" "1" "1"
##   .. ..$ organizadores           : chr [1:3] "Fernando Martinho" "Susana Dobal e Osmar Gonçalves" "Leda Maria Rangearo Fiorentini; Liliane Campos Machado; Maria do Carmo Nascimento Diniz; Otília Maria Alves da "| __truncated__
##   .. ..$ paginas                 : chr [1:3] "8 - 18" "179 - 187" "79 - 86"
##   .. ..$ autores                 :List of 3
##   .. ..$ autores-endogeno        :List of 3
##   ..$ DEMAIS_TIPOS_DE_PRODUCAO_BIBLIOGRAFICA:'data.frame':   13 obs. of  9 variables:
##   .. ..$ natureza          : chr [1:13] "ARTIGO" "Texto de abertura de exposição" "ARTIGO" "ARTIGO" ...
##   .. ..$ titulo            : chr [1:13] "Fotografias dentro da mochila ou tecnologias iluminadas" "Photoland: fotografia transformada em paisagem" "Nascimentos fotográficos" "O que vem à luz sobre um fundamento que lhe vem estranho" ...
##   .. ..$ ano               : chr [1:13] "2010" "2010" "2011" "2011" ...
##   .. ..$ pais_de_publicacao: chr [1:13] "Brasil" "Brasil" "Brasil" "Brasil" ...
##   .. ..$ editora           : chr [1:13] "Fórum Latino americano de Fotografia" "Paraty em Foco" "Icônica" "Icônica" ...
##   .. ..$ doi               : chr [1:13] "" "" "" "" ...
##   .. ..$ numero_de_paginas : chr [1:13] "" "" "" "5" ...
##   .. ..$ autores           :List of 13
##   .. ..$ autores-endogeno  :List of 13
##   ..$ EVENTO                                :'data.frame':   4 obs. of  11 variables:
##   .. ..$ natureza        : chr [1:4] "COMPLETO" "COMPLETO" "RESUMO_EXPANDIDO" "COMPLETO"
##   .. ..$ titulo          : chr [1:4] "O tempo e o espaço no território da fotografia" "Tempo para nada, só para fazer fotografias" "Cybernetic Theory and the Meaning of Education in Contemporary Society" "Tal mãe, tal filha: famílias fitness e os empresários de si mesmos no contexto da ?boa forma'"
##   .. ..$ nome_do_evento  : chr [1:4] "FESTPOA- Festival Internacional de Fotografia de Porto Alegre" "FESTPOA- Festival Internacional de Fotografia de Porto Alegre" "13th Annual International Conference on Communication and Mass Media" "VIII Enpecom - Encontro de Pesquisa em Comunicac&#807;a&#771;o"
##   .. ..$ ano_do_trabalho : chr [1:4] "2010" "2011" "2015" "2016"
##   .. ..$ pais_do_evento  : chr [1:4] "Brasil" "Brasil" "Brasil" "Brasil"
##   .. ..$ cidade_do_evento: chr [1:4] "Porto Alegre" "Porto Alegre" "Atenas" "Curitiba"
##   .. ..$ doi             : chr [1:4] "" "" "" ""
##   .. ..$ classificacao   : chr [1:4] "INTERNACIONAL" "INTERNACIONAL" "INTERNACIONAL" "NACIONAL"
##   .. ..$ paginas         : chr [1:4] "33 - 44" "69 - 82" "11 - 93" " - "
##   .. ..$ autores         :List of 4
##   .. ..$ autores-endogeno:List of 4
##   ..$ PERIODICO                             :'data.frame':   5 obs. of  10 variables:
##   .. ..$ natureza        : chr [1:5] "COMPLETO" "COMPLETO" "COMPLETO" "COMPLETO" ...
##   .. ..$ titulo          : chr [1:5] "Insônia Fotográfica: promessa e fracasso do aniquilamento do tempo" "Quando o tempo fugiu do instantâneo" "Entre o tempo perdido e o instante: cronofotografia, ciência e temporalidade moderna" "A fábula da câmera invisível na escola e o regime contemporâneo de imagens" ...
##   .. ..$ periodico       : chr [1:5] "FACOM (FAAP)" "Studium (UNICAMP)" "Boletim do Museu Paraense Emílio Goeldi. Ciências Humanas" "REVISTA ECO-PÓS (ONLINE)" ...
##   .. ..$ ano             : chr [1:5] "2011" "2011" "2014" "2015" ...
##   .. ..$ volume          : chr [1:5] "23" "32" "9" "V.18" ...
##   .. ..$ issn            : chr [1:5] "16768221" "15194388" "19818122" "21758689" ...
##   .. ..$ paginas         : chr [1:5] "16 - 30" "4 - " "443 - 462" "119 - " ...
##   .. ..$ doi             : chr [1:5] "" "" "10.1590/1981-81222014000200011" "" ...
##   .. ..$ autores         :List of 5
##   .. ..$ autores-endogeno:List of 5
##  $ orientacoes_academicas:List of 3
##   ..$ ORIENTACAO_CONCLUIDA_MESTRADO   :'data.frame': 3 obs. of  13 variables:
##   .. ..$ natureza                   : chr [1:3] "Dissertação de mestrado" "Dissertação de mestrado" "Dissertação de mestrado"
##   .. ..$ titulo                     : chr [1:3] "Caiu na Net: Violação de intimidade e regime de vigilância distribuída" "Sua melhor versão de si: os sentidos e as práticas da boa forma na contemporaneidade" "SER NO TEMPO: IMAGEM, TECNOLOGIA E SUBJETIVIDADE NO MUNDO CONTEMPORÂNEO"
##   .. ..$ ano                        : chr [1:3] "2015" "2015" "2016"
##   .. ..$ id_lattes_aluno            : chr [1:3] "3513003179027403" "7502890940192840" "1921859680948956"
##   .. ..$ nome_aluno                 : chr [1:3] "Bruno Ramos Crayesmeyer" "Angélica Fonsêca de Freitas" "Paulo Henrique Martins de Jesus"
##   .. ..$ instituicao                : chr [1:3] "Universidade de Brasília" "Universidade de Brasília" "Universidade de Brasília"
##   .. ..$ curso                      : chr [1:3] "Comunicação" "Comunicação" "Programa de Pós-Graduação em Comunicação"
##   .. ..$ codigo_do_curso            : chr [1:3] "90000035" "90000035" "90000039"
##   .. ..$ bolsa                      : chr [1:3] "NAO" "SIM" "SIM"
##   .. ..$ agencia_financiadora       : chr [1:3] "" "Conselho Nacional de Desenvolvimento Científico e Tecnológico" "Coordenação de Aperfeiçoamento de Pessoal de Nível Superior"
##   .. ..$ codigo_agencia_financiadora: chr [1:3] "" "002200000000" "045000000000"
##   .. ..$ nome_orientadores          :List of 3
##   .. ..$ id_lattes_orientadores     :List of 3
##   ..$ ORIENTACAO_EM_ANDAMENTO_MESTRADO:'data.frame': 3 obs. of  13 variables:
##   .. ..$ natureza                   : chr [1:3] "Dissertação de mestrado" "Dissertação de mestrado" "Dissertação de mestrado"
##   .. ..$ titulo                     : chr [1:3] "Fotografia e tempo na penumbra: Francesca Woodman e a dança com fantasmas" "Tempo, Imagem e Subjetividade: Os sentidos de ser velho na sociedade de risco" "Comunicação museológica: A influência da linguagem poética em museus históricos do Distrito Federal."
##   .. ..$ ano                        : chr [1:3] "2016" "2017" "2017"
##   .. ..$ id_lattes_aluno            : chr [1:3] "0819148077988194" "3063666731999466" "0887390289207360"
##   .. ..$ nome_aluno                 : chr [1:3] "Fabiane da Silva de Souza" "Mirella Ramos Costa Pessoa" "Ingridde dos Santos Alves"
##   .. ..$ instituicao                : chr [1:3] "Universidade de Brasília" "Universidade de Brasília" "Universidade de Brasília"
##   .. ..$ curso                      : chr [1:3] "Comunicação" "Comunicação" "Programa de Pós-Graduação em Comunicação"
##   .. ..$ codigo_do_curso            : chr [1:3] "90000041" "90000041" "90000042"
##   .. ..$ bolsa                      : chr [1:3] "SIM" "SIM" "SIM"
##   .. ..$ agencia_financiadora       : chr [1:3] "Conselho Nacional de Desenvolvimento Científico e Tecnológico" "Coordenação de Aperfeiçoamento de Pessoal de Nível Superior" "Coordenação de Aperfeiçoamento de Pessoal de Nível Superior"
##   .. ..$ codigo_agencia_financiadora: chr [1:3] "002200000000" "045000000000" "045000000000"
##   .. ..$ nome_orientadores          :List of 3
##   .. ..$ id_lattes_orientadores     :List of 3
##   ..$ OUTRAS_ORIENTACOES_CONCLUIDAS   :'data.frame': 11 obs. of  13 variables:
##   .. ..$ natureza                   : chr [1:11] "TRABALHO_DE_CONCLUSAO_DE_CURSO_GRADUACAO" "INICIACAO_CIENTIFICA" "INICIACAO_CIENTIFICA" "MONOGRAFIA_DE_CONCLUSAO_DE_CURSO_APERFEICOAMENTO_E_ESPECIALIZACAO" ...
##   .. ..$ titulo                     : chr [1:11] "Fotografia e desjo: uma analise ensaística" "Memória e imagem na obra de Walter Benjamin" "Infância, imagem e educação: um estudo acerca dos escritos de Walter Benjamin" "EJA COMBINADA: um caminho para uma organização mais adequada aos tempos do aluno trabalhador" ...
##   .. ..$ ano                        : chr [1:11] "2011" "2012" "2015" "2015" ...
##   .. ..$ id_lattes_aluno            : chr [1:11] "" "" "" "" ...
##   .. ..$ nome_aluno                 : chr [1:11] "Henrique Assis; Pedro Lino" "Letícia Marotta Pedersoli de Oliveira" "Aluízio Augusto Carvalho Santos" "Adriane de Cravalho" ...
##   .. ..$ instituicao                : chr [1:11] "Universidade Católica de Brasília" "Universidade de Brasília" "Universidade de Brasília" "Universidade de Brasília" ...
##   .. ..$ curso                      : chr [1:11] "Comunicação Social" "Comunicação Social" "Pedagogia" "Curso de Especialização em Educação na Diversidade e Cidania com ênfase EJA" ...
##   .. ..$ codigo_do_curso            : chr [1:11] "90000025" "60293420" "60065990" "90000040" ...
##   .. ..$ bolsa                      : chr [1:11] "NAO" "NAO" "NAO" "NAO" ...
##   .. ..$ agencia_financiadora       : chr [1:11] "" "" "" "" ...
##   .. ..$ codigo_agencia_financiadora: chr [1:11] "" "" "" "" ...
##   .. ..$ nome_orientadores          :List of 11
##   .. ..$ id_lattes_orientadores     :List of 11
##  $ senioridade           : chr "8"
\end{verbatim}

\begin{Shaded}
\begin{Highlighting}[]
\CommentTok{# politica social}
\KeywordTok{glimpse}\NormalTok{(ps.prof[[}\DecValTok{8}\NormalTok{]], }\DataTypeTok{width =} \DecValTok{30}\NormalTok{)}
\end{Highlighting}
\end{Shaded}

\begin{verbatim}
## List of 7
##  $ nome                  : chr "Ivanete Salete Boschetti"
##  $ resumo_cv             : chr "Graduada em Serviço Social pela Universidade Católica Dom Bosco (1985), mestre em Política Social pela Universi"| __truncated__
##  $ areas_de_atuacao      :'data.frame':  6 obs. of  4 variables:
##   ..$ grande_area  : chr [1:6] "CIENCIAS_SOCIAIS_APLICADAS" "CIENCIAS_SOCIAIS_APLICADAS" "CIENCIAS_SOCIAIS_APLICADAS" "CIENCIAS_SOCIAIS_APLICADAS" ...
##   ..$ area         : chr [1:6] "Serviço Social" "Serviço Social" "Serviço Social" "Serviço Social" ...
##   ..$ sub_area     : chr [1:6] "Serviço Social Aplicado" "Fundamentos do Serviço Social" "Trabalho" "Previdência Social" ...
##   ..$ especialidade: chr [1:6] "Seguridade Social" "Formação Profissional" "" "Assistência Social" ...
##  $ endereco_profissional :List of 8
##   ..$ instituicao: chr "Universidade de Brasília"
##   ..$ orgao      : chr "Instituto de Ciências Humanas"
##   ..$ unidade    : chr "Departamento de Serviço Social"
##   ..$ DDD        : chr "61"
##   ..$ telefone   : chr "31076631"
##   ..$ bairro     : chr "Asa Norte"
##   ..$ cep        : chr "70919970"
##   ..$ cidade     : chr "Brasília"
##  $ producao_bibiografica :List of 6
##   ..$ CAPITULO_DE_LIVRO                     :'data.frame':   9 obs. of  13 variables:
##   .. ..$ tipo                    : chr [1:9] "Capítulo de livro publicado" "Capítulo de livro publicado" "Capítulo de livro publicado" "Capítulo de livro publicado" ...
##   .. ..$ titulo_do_capitulo      : chr [1:9] "Os Custos da Crise para a Política Social" "As Políticas Sociais e os Programas Sociais Atuais na França e no Brasil" "Les Politiques Publiques et les Programmes Sociaux Actuels en France et au Brésil" "América Latina, Política Social e Pobreza: novo modelo de desenvolvimento?" ...
##   .. ..$ titulo_do_livro         : chr [1:9] "Capitalismo em Crise. Política Social e Direitos" "O Trabalho Social França Brasil" "Le Travail Social en France et au Brésil" "Financeirização, Fundo Público e Poítica Social" ...
##   .. ..$ ano                     : chr [1:9] "2010" "2011" "2011" "2012" ...
##   .. ..$ doi                     : chr [1:9] "" "" "" "" ...
##   .. ..$ pais_de_publicacao      : chr [1:9] "Brasil" "Brasil" "Brasil" "Brasil" ...
##   .. ..$ isbn                    : chr [1:9] "9788524916694" "9788579950186" "9788579950186" "9788524919961" ...
##   .. ..$ nome_da_editora         : chr [1:9] "Cortez" "SESC/CBCISS" "SESC;CBCISS" "Cortez" ...
##   .. ..$ numero_da_edicao_revisao: chr [1:9] "1a" "1" "1" "1a" ...
##   .. ..$ organizadores           : chr [1:9] "Ivanete Boschetti; Elaine Rossetti Behring; Silvana Mara de Morais dos Santos; Regina Célia Tamaso Mioto" "SESC;CBCISS" "SESC;CBCISS" "Evilásio Salvador; Elaine Rossetti Behring; Ivanete Boschetti, Sara Granemann" ...
##   .. ..$ paginas                 : chr [1:9] "64 - 85" "63 - 76" "255 - 269" "31 - 58" ...
##   .. ..$ autores                 :List of 9
##   .. ..$ autores-endogeno        :List of 9
##   ..$ DEMAIS_TIPOS_DE_PRODUCAO_BIBLIOGRAFICA:'data.frame':   10 obs. of  9 variables:
##   .. ..$ natureza          : chr [1:10] "CFESS Manifesta" "CFESS Manifesta" "CFESS Manifesta" "CFESS Manifesta" ...
##   .. ..$ titulo            : chr [1:10] "O Amor Exige Expressão e Reverência Coletiva" "Lutas Sociais e Exercício Profisisonal no Contexto da Crise do Capital: mediações e a consolidação do Projeto É"| __truncated__ "Pela Sustentabilidade dos Conselhos Profissionais e Aprovação do PL Anuidade" "Trabalho com Direitos. Pelo Fim da Desigualdade" ...
##   .. ..$ ano               : chr [1:10] "2010" "2010" "2010" "2010" ...
##   .. ..$ pais_de_publicacao: chr [1:10] "Brasil" "Brasil" "Brasil" "Brasil" ...
##   .. ..$ editora           : chr [1:10] "CFESS" "CFESS" "CFESS" "CFESS" ...
##   .. ..$ doi               : chr [1:10] "" "" "" "" ...
##   .. ..$ numero_de_paginas : chr [1:10] "02" "04" "02" "02" ...
##   .. ..$ autores           :List of 10
##   .. ..$ autores-endogeno  :List of 10
##   ..$ EVENTO                                :'data.frame':   11 obs. of  11 variables:
##   .. ..$ natureza        : chr [1:11] "COMPLETO" "COMPLETO" "COMPLETO" "RESUMO" ...
##   .. ..$ titulo          : chr [1:11] "Política Social e Direitos" "Condições de Trabalho e Projeto Ético Político Profissional" "AS ?REFORMAS? DA PROTEÇÃO SOCIAL EM TEMPOS DE CRISE" "Les effets de la crise actuelle sur la protection sociale au Brésil et en France" ...
##   .. ..$ nome_do_evento  : chr [1:11] "XII Encontro Nacional de Pesquisadores em Serviço Social - ENPES" "Seminário Nacional: O Trabalho de Assistentes Sociais no SUAS" "XIII Encontro Nacional de Pesquisadores em Serviço Social" "Joint World Conference in Social Work and Social Development: Action and Impact" ...
##   .. ..$ ano_do_trabalho : chr [1:11] "2010" "2011" "2012" "2012" ...
##   .. ..$ pais_do_evento  : chr [1:11] "Brasil" "Brasil" "Brasil" "Brasil" ...
##   .. ..$ cidade_do_evento: chr [1:11] "Rio de Janeiro" "Rio de Janeiro" "Juiz de Fora MG" "Estocolmo, Suécia" ...
##   .. ..$ doi             : chr [1:11] "" "" "" "" ...
##   .. ..$ classificacao   : chr [1:11] "NACIONAL" "NACIONAL" "NACIONAL" "INTERNACIONAL" ...
##   .. ..$ paginas         : chr [1:11] " - " "`291 - 307" "s/p - s/p" " - " ...
##   .. ..$ autores         :List of 11
##   .. ..$ autores-endogeno:List of 11
##   ..$ LIVRO                                 :'data.frame':   6 obs. of  13 variables:
##   .. ..$ titulo                  : chr [1:6] "Política Social:Fundamentos e História" "Capitalismo em Crise. Política Social e Direitos" "Direito se Conquista: a Luta dos/as assistentes Sociais pelas 30 horas Semanais" "Política Social. Fundamentos e História" ...
##   .. ..$ ano                     : chr [1:6] "2010" "2010" "2011" "2011" ...
##   .. ..$ tipo                    : chr [1:6] "LIVRO_PUBLICADO" "LIVRO_ORGANIZADO_OU_EDICAO" "LIVRO_ORGANIZADO_OU_EDICAO" "LIVRO_PUBLICADO" ...
##   .. ..$ natureza                : chr [1:6] "TEXTO_INTEGRAL" "LIVRO" "LIVRO" "TEXTO_INTEGRAL" ...
##   .. ..$ pais_de_publicacao      : chr [1:6] "Brasil" "Brasil" "Brasil" "Brasil" ...
##   .. ..$ isbn                    : chr [1:6] "8524912596" "9788524916694" "579.496.707-2" "8524912596" ...
##   .. ..$ doi                     : chr [1:6] "" "" "" "" ...
##   .. ..$ nome_da_editora         : chr [1:6] "Cortez" "Cortez" "CFESS" "Cortez Editora" ...
##   .. ..$ numero_da_edicao_revisao: chr [1:6] "7a" "1a" "1a" "9a" ...
##   .. ..$ numero_de_paginas       : chr [1:6] "213" "309" "155" "213" ...
##   .. ..$ numero_de_volumes       : chr [1:6] "1" "1" "1" "1" ...
##   .. ..$ autores                 :List of 6
##   .. ..$ autores-endogeno        :List of 6
##   ..$ PERIODICO                             :'data.frame':   10 obs. of  10 variables:
##   .. ..$ natureza        : chr [1:10] "COMPLETO" "COMPLETO" "COMPLETO" "COMPLETO" ...
##   .. ..$ titulo          : chr [1:10] "Condições de trabalho e a luta dos(as) assistentes sociais pela jornada semanal de 30 horas" "Desafios e Atuação da ABEPSS no Contexto da ?Reforma? do Ensino Superior no Final dos Anos 1990 - Gestão 1998-2000 Qualis B1" "Os Intensos Anos de ?Atitude Crítica para Avançar na Luta? - Qualis B4" "A insidiosa corrosão dos sistemas de proteção social europeus" ...
##   .. ..$ periodico       : chr [1:10] "Serviço Social & Sociedade" "Temporalis (Brasília)" "Revista Inscrita (Rio de Janeiro)" "Serviço Social & Sociedade" ...
##   .. ..$ ano             : chr [1:10] "2011" "2011" "2012" "2012" ...
##   .. ..$ volume          : chr [1:10] "" "2" "13" "" ...
##   .. ..$ issn            : chr [1:10] "01016628" "15187934" "14150921" "01016628" ...
##   .. ..$ paginas         : chr [1:10] "557 - 584" "27 - 42" "28 - 33" "754 - 803" ...
##   .. ..$ doi             : chr [1:10] "10.1590/S0101-66282011000300010" "" "" "10.1590/S0101-66282012000400008" ...
##   .. ..$ autores         :List of 10
##   .. ..$ autores-endogeno:List of 10
##   ..$ TEXTO_EM_JORNAIS                      :'data.frame':   1 obs. of  9 variables:
##   .. ..$ natureza        : chr "REVISTA_MAGAZINE"
##   .. ..$ titulo          : chr "Tendências da Assistência Social após 20 anos de LOAS"
##   .. ..$ periodico       : chr "Revista do Congemas"
##   .. ..$ ano             : chr "2013"
##   .. ..$ issn            : chr ""
##   .. ..$ paginas         : chr "23 - 29"
##   .. ..$ doi             : chr ""
##   .. ..$ autores         :List of 1
##   .. ..$ autores-endogeno:List of 1
##  $ orientacoes_academicas:List of 7
##   ..$ ORIENTACAO_CONCLUIDA_DOUTORADO              :'data.frame': 11 obs. of  13 variables:
##   .. ..$ natureza                   : chr [1:11] "Tese de doutorado" "Tese de doutorado" "Tese de doutorado" "Tese de doutorado" ...
##   .. ..$ titulo                     : chr [1:11] "(Des)Estruturação do Trabalho e Condições para a Universalização da Previdência Social no Brasil" "Mercosul e a Política de Assistência Social:" "A Concepção da Política de Assistência Social e sua Efetivação em Municípios Goianos: o Novo Jargão e o Conserv"| __truncated__ "Proteção Social no Capitalismo: Contribuições à Crítica de Matrizes Teóricas e Ideológicas Conflitantes" ...
##   .. ..$ ano                        : chr [1:11] "2011" "2011" "2012" "2013" ...
##   .. ..$ id_lattes_aluno            : chr [1:11] "6985760672107950" "5314743908635602" "5100268908025158" "4621637454405096" ...
##   .. ..$ nome_aluno                 : chr [1:11] "Maria Lucia Lopes da Silva" "Karen Santana de Almeida Vieira" "Maisa Miralva Silva" "Camila Potyara Pereira" ...
##   .. ..$ instituicao                : chr [1:11] "Universidade de Brasília" "Universidade de Brasília" "Universidade de Brasília" "Universidade de Brasília" ...
##   .. ..$ curso                      : chr [1:11] "Política Social" "Política Social" "Política Social" "Política Social" ...
##   .. ..$ codigo_do_curso            : chr [1:11] "51500353" "51500353" "51500353" "51500353" ...
##   .. ..$ bolsa                      : chr [1:11] "NAO" "NAO" "NAO" "SIM" ...
##   .. ..$ agencia_financiadora       : chr [1:11] "" "" "" "Coordenação de Aperfeiçoamento de Pessoal de Nível Superior" ...
##   .. ..$ codigo_agencia_financiadora: chr [1:11] "" "" "" "045000000000" ...
##   .. ..$ nome_orientadores          :List of 11
##   .. ..$ id_lattes_orientadores     :List of 11
##   ..$ ORIENTACAO_CONCLUIDA_MESTRADO               :'data.frame': 4 obs. of  13 variables:
##   .. ..$ natureza                   : chr [1:4] "Dissertação de mestrado" "Dissertação de mestrado" "Dissertação de mestrado" "Dissertação de mestrado"
##   .. ..$ titulo                     : chr [1:4] "O Direito Socio-assistencial de Segurança de Renda no Brasil" "O Crescimento da Previdência Complementar como Efeito da Contrarreforma do Estado no Brasil: o Direito à Proteç"| __truncated__ "Conceito de Deficiência na Materialização do Acesso ao BPC: Impactos na Proteção e na Relação Assistência Social e Trabalho" "A Política de Trabalho no Brasil e a Efetivação de Vínculos Estáveis: retorno à Individualização da Proteção Social"
##   .. ..$ ano                        : chr [1:4] "2011" "2011" "2014" "2014"
##   .. ..$ id_lattes_aluno            : chr [1:4] "8276613839586798" "9076244548086829" "0126476057249391" "5310348041508543"
##   .. ..$ nome_aluno                 : chr [1:4] "Alvaro André Santarém Amorim" "Pollyana Moreira de Assis" "Taís Leite Flores" "Fabiana Esteves Boaventura"
##   .. ..$ instituicao                : chr [1:4] "Universidade de Brasília" "Universidade de Brasília" "Universidade de Brasília" "Universidade de Brasília"
##   .. ..$ curso                      : chr [1:4] "Política Social" "Política Social" "Política Social" "Política Social"
##   .. ..$ codigo_do_curso            : chr [1:4] "51500353" "51500353" "51500353" "51500353"
##   .. ..$ bolsa                      : chr [1:4] "NAO" "NAO" "NAO" "SIM"
##   .. ..$ agencia_financiadora       : chr [1:4] "" "" "" "Coordenação de Aperfeiçoamento de Pessoal de Nível Superior"
##   .. ..$ codigo_agencia_financiadora: chr [1:4] "" "" "" "045000000000"
##   .. ..$ nome_orientadores          :List of 4
##   .. ..$ id_lattes_orientadores     :List of 4
##   ..$ ORIENTACAO_CONCLUIDA_POS_DOUTORADO          :'data.frame': 3 obs. of  13 variables:
##   .. ..$ natureza                   : chr [1:3] "Supervisão de pós-doutorado" "Supervisão de pós-doutorado" "Supervisão de pós-doutorado"
##   .. ..$ titulo                     : chr [1:3] "" "" ""
##   .. ..$ ano                        : chr [1:3] "2015" "2015" "2017"
##   .. ..$ id_lattes_aluno            : chr [1:3] "" "" ""
##   .. ..$ nome_aluno                 : chr [1:3] "Silvana Mara de Morais dos Santos" "Andrea Lima da Silva" "Esther Luiza de Souza Lemos"
##   .. ..$ instituicao                : chr [1:3] "Universidade de Brasília" "Universidade de Brasília" "Universidade de Brasília"
##   .. ..$ curso                      : chr [1:3] "" "" ""
##   .. ..$ codigo_do_curso            : chr [1:3] "" "" ""
##   .. ..$ bolsa                      : chr [1:3] "SIM" "SIM" "SIM"
##   .. ..$ agencia_financiadora       : chr [1:3] "Coordenação de Aperfeiçoamento de Pessoal de Nível Superior" "Coordenação de Aperfeiçoamento de Pessoal de Nível Superior" "Coordenação de Aperfeiçoamento de Pessoal de Nível Superior"
##   .. ..$ codigo_agencia_financiadora: chr [1:3] "045000000000" "045000000000" "045000000000"
##   .. ..$ nome_orientadores          :List of 3
##   .. ..$ id_lattes_orientadores     :List of 3
##   ..$ ORIENTACAO_EM_ANDAMENTO_DOUTORADO           :'data.frame': 4 obs. of  13 variables:
##   .. ..$ natureza                   : chr [1:4] "Tese de doutorado" "Tese de doutorado" "Tese de doutorado" "Tese de doutorado"
##   .. ..$ titulo                     : chr [1:4] "O fetiche da inserção individual no mercado de trabalho no Brasil: precarização do trabalho e das políticas de "| __truncated__ "O extermínio da juventude negra e os desafios às políticas sociais" "O debate teórico e as polêmicas sobre a assistência social nas produções do Serviço Social" "TRABALHO, OPRESSÕES E ONTOLOGIA DO SER SOCIAL: A ALIENAÇÃO E A UNIDADE EXPLORAÇÃO-OPRESSÃO"
##   .. ..$ ano                        : chr [1:4] "2014" "2015" "2016" "2017"
##   .. ..$ id_lattes_aluno            : chr [1:4] "5310348041508543" "" "" ""
##   .. ..$ nome_aluno                 : chr [1:4] "Fabiana Esteves Boaventura" "Leonardo Ortegal" "Raquel Sabará de Freitas" "PAULO WESCLEY PINHEIRO"
##   .. ..$ instituicao                : chr [1:4] "Universidade de Brasília" "Universidade de Brasília" "Universidade de Brasília" "Universidade de Brasília"
##   .. ..$ curso                      : chr [1:4] "Política Social" "Política Social" "Política Social" "Política Social"
##   .. ..$ codigo_do_curso            : chr [1:4] "51500353" "51500353" "51500353" "51500353"
##   .. ..$ bolsa                      : chr [1:4] "SIM" "NAO" "SIM" "NAO"
##   .. ..$ agencia_financiadora       : chr [1:4] "" "" "Coordenação de Aperfeiçoamento de Pessoal de Nível Superior" ""
##   .. ..$ codigo_agencia_financiadora: chr [1:4] "" "" "045000000000" ""
##   .. ..$ nome_orientadores          :List of 4
##   .. ..$ id_lattes_orientadores     :List of 4
##   ..$ ORIENTACAO_EM_ANDAMENTO_INICIACAO_CIENTIFICA:'data.frame': 4 obs. of  13 variables:
##   .. ..$ natureza                   : chr [1:4] "Iniciação Científica" "Iniciação Científica" "Iniciação Científica" "Iniciação Científica"
##   .. ..$ titulo                     : chr [1:4] "O Debate sobre Dívida Pública e Expropriação de Direitos nas Teses de Doutorados e Livros do Serviço Social" "O Debate sobre Dívida Pública e Expropriação de Direitos nos Artigos Científicos na Área do Serviço Social" "A Dívida Pública e os Processos de Expropriação dos Direitos Sociais no Brasileiro" "Expropriação e Acumulação de Capital na Literatura Francesa Marxista"
##   .. ..$ ano                        : chr [1:4] "2017" "2017" "2017" "2017"
##   .. ..$ id_lattes_aluno            : chr [1:4] "" "" "" ""
##   .. ..$ nome_aluno                 : chr [1:4] "Victoria Maria da Costa" "Giovanna Guarese" "Gabriela Rodrigues de Moraes" "Sofia Rodrigues"
##   .. ..$ instituicao                : chr [1:4] "Universidade de Brasília" "Universidade de Brasília" "Universidade de Brasília" "Universidade de Brasília"
##   .. ..$ curso                      : chr [1:4] "Serviço Social" "Serviço Social" "Serviço Social" "Serviço Social"
##   .. ..$ codigo_do_curso            : chr [1:4] "60317493" "60317493" "60317493" "60317493"
##   .. ..$ bolsa                      : chr [1:4] "SIM" "SIM" "SIM" "SIM"
##   .. ..$ agencia_financiadora       : chr [1:4] "Coordenação de Aperfeiçoamento de Pessoal de Nível Superior" "Universidade de Brasília" "Conselho Nacional de Desenvolvimento Científico e Tecnológico" "Coordenação de Aperfeiçoamento de Pessoal de Nível Superior"
##   .. ..$ codigo_agencia_financiadora: chr [1:4] "045000000000" "002000000996" "002200000000" "045000000000"
##   .. ..$ nome_orientadores          :List of 4
##   .. ..$ id_lattes_orientadores     :List of 4
##   ..$ ORIENTACAO_EM_ANDAMENTO_MESTRADO            :'data.frame': 3 obs. of  13 variables:
##   .. ..$ natureza                   : chr [1:3] "Dissertação de mestrado" "Dissertação de mestrado" "Dissertação de mestrado"
##   .. ..$ titulo                     : chr [1:3] "Política Econômica e Bloco Político no Governo Lula" "A concepção da família na seguridade social brasileira e a garantia dos direitos das pessoas LGBT: Uma analise "| __truncated__ "POLÍTICA SOCIAL E CIDADANIA LGBT: A GARANTIA DE DIREITOS E OS LIMITES EMANCIPATÓRIOS NO CAPITALISMO"
##   .. ..$ ano                        : chr [1:3] "2016" "2017" "2017"
##   .. ..$ id_lattes_aluno            : chr [1:3] "" "6298226014768391" "1134437109362587"
##   .. ..$ nome_aluno                 : chr [1:3] "Franklin Rabelo de Melo" "Helena Godoy Brito" "DJONATAN KAIC RIBEIRO DE SOUZA"
##   .. ..$ instituicao                : chr [1:3] "Universidade de Brasília" "Universidade de Brasília" "Universidade de Brasília"
##   .. ..$ curso                      : chr [1:3] "Política Social" "Política Social" "Política Social"
##   .. ..$ codigo_do_curso            : chr [1:3] "51500353" "51500353" "51500353"
##   .. ..$ bolsa                      : chr [1:3] "NAO" "SIM" "NAO"
##   .. ..$ agencia_financiadora       : chr [1:3] "" "Coordenação de Aperfeiçoamento de Pessoal de Nível Superior" ""
##   .. ..$ codigo_agencia_financiadora: chr [1:3] "" "045000000000" ""
##   .. ..$ nome_orientadores          :List of 3
##   .. ..$ id_lattes_orientadores     :List of 3
##   ..$ OUTRAS_ORIENTACOES_CONCLUIDAS               :'data.frame': 32 obs. of  13 variables:
##   .. ..$ natureza                   : chr [1:32] "TRABALHO_DE_CONCLUSAO_DE_CURSO_GRADUACAO" "TRABALHO_DE_CONCLUSAO_DE_CURSO_GRADUACAO" "TRABALHO_DE_CONCLUSAO_DE_CURSO_GRADUACAO" "INICIACAO_CIENTIFICA" ...
##   .. ..$ titulo                     : chr [1:32] "Os Desafios da Avaliação Social para Acesso ao BPC" "O Avanço dos Planos Privados de Saúde Pós Constituição de 1988: a Consolidação de um Sistema Dual" "População de Rua e (Des) Proteção Social: Análise da Política Nacional para a População em Situação de Rua" "As Tendências Contemporâneas da Política de Direitos da Cidadania para Mulheres no Brasil" ...
##   .. ..$ ano                        : chr [1:32] "2010" "2011" "2011" "2013" ...
##   .. ..$ id_lattes_aluno            : chr [1:32] "" "" "" "" ...
##   .. ..$ nome_aluno                 : chr [1:32] "Alan Teles da Silva" "Jorge Augusto Bezerra" "Getúlio Henrique Alves" "Helena Fragoso de Mendonça Santiago" ...
##   .. ..$ instituicao                : chr [1:32] "Universidade de Brasília" "Universidade de Brasília" "Universidade de Brasília" "Universidade de Brasília" ...
##   .. ..$ curso                      : chr [1:32] "Serviço Social" "Serviço Social" "Serviço Social" "Serviço Social" ...
##   .. ..$ codigo_do_curso            : chr [1:32] "90000004" "90000004" "90000004" "60083557" ...
##   .. ..$ bolsa                      : chr [1:32] "NAO" "NAO" "NAO" "NAO" ...
##   .. ..$ agencia_financiadora       : chr [1:32] "" "" "" "" ...
##   .. ..$ codigo_agencia_financiadora: chr [1:32] "" "" "" "" ...
##   .. ..$ nome_orientadores          :List of 32
##   .. ..$ id_lattes_orientadores     :List of 32
##  $ senioridade           : chr "8"
\end{verbatim}

\begin{Shaded}
\begin{Highlighting}[]
\CommentTok{# sociologia}
\KeywordTok{glimpse}\NormalTok{(so.prof[[}\DecValTok{2}\NormalTok{]], }\DataTypeTok{width =} \DecValTok{30}\NormalTok{)}
\end{Highlighting}
\end{Shaded}

\begin{verbatim}
## List of 7
##  $ nome                  : chr "Christiane Girard Ferreira Nunes"
##  $ resumo_cv             : chr "Graduação em Sociologia - Universite de Paris VIII (1981), Mestrado em Sociologia - Universite Paris VIIII (198"| __truncated__
##  $ areas_de_atuacao      :'data.frame':  3 obs. of  4 variables:
##   ..$ grande_area  : chr [1:3] "CIENCIAS_HUMANAS" "CIENCIAS_HUMANAS" "CIENCIAS_HUMANAS"
##   ..$ area         : chr [1:3] "Sociologia" "Sociologia" "Sociologia"
##   ..$ sub_area     : chr [1:3] "" "SOCIOLOGIA DO TRABALHO" "Sociologia Clínica"
##   ..$ especialidade: chr [1:3] "" "" ""
##  $ endereco_profissional :List of 8
##   ..$ instituicao: chr "Universidade de Brasília"
##   ..$ orgao      : chr "Instituto de Ciências Humanas"
##   ..$ unidade    : chr ""
##   ..$ DDD        : chr "061"
##   ..$ telefone   : chr "31077333"
##   ..$ bairro     : chr "ASA NORTE"
##   ..$ cep        : chr "70910-900"
##   ..$ cidade     : chr "Brasilia"
##  $ producao_bibiografica :List of 6
##   ..$ ARTIGO_ACEITO                         :'data.frame':   2 obs. of  10 variables:
##   .. ..$ natureza        : chr [1:2] "NAO_INFORMADO" "NAO_INFORMADO"
##   .. ..$ titulo          : chr [1:2] "Entre o prescrito e o real: o papel da subjetividade na efetivação dos direitos das empregadas domesticas no Brasil" "A Arte de trabalhar"
##   .. ..$ periodico       : chr [1:2] "Sociedade e Estado (UnB. Impresso)" "Lamparina: Revista de Ensino do Teatro"
##   .. ..$ ano             : chr [1:2] "2014" "2014"
##   .. ..$ volume          : chr [1:2] "" ""
##   .. ..$ issn            : chr [1:2] "01026992" "21776121"
##   .. ..$ paginas         : chr [1:2] " - " " - "
##   .. ..$ doi             : chr [1:2] "" ""
##   .. ..$ autores         :List of 2
##   .. ..$ autores-endogeno:List of 2
##   ..$ CAPITULO_DE_LIVRO                     :'data.frame':   4 obs. of  13 variables:
##   .. ..$ tipo                    : chr [1:4] "Capítulo de livro publicado" "Capítulo de livro publicado" "Capítulo de livro publicado" "Capítulo de livro publicado"
##   .. ..$ titulo_do_capitulo      : chr [1:4] "As Representações dos Empresários Sobre Inovação" "Sociologia Clinica" "o contexto do trabalho dos professores (as) na educação superior: as mudanças no mundo do trabalho do professor" "A pluralidade dos movimentos sociais (em temáticas de Políticas Públicas Espaciais)"
##   .. ..$ titulo_do_livro         : chr [1:4] "PAEDI - Pesquisa sobre Atitudes Empresariais para Desenvolvimento e Inovação" "Dicionário Critica de Gestão e Psicodinâmica do trabalho" "O contexto do trabalho dos professores (as) na educação superior: as mudanças no mundo do trabalho e no mundo dos professores" "Território, agentes-atores e políticas públicas espaciais"
##   .. ..$ ano                     : chr [1:4] "2012" "2013" "2013" "2017"
##   .. ..$ doi                     : chr [1:4] "" "" "" ""
##   .. ..$ pais_de_publicacao      : chr [1:4] "Brasil" "Brasil" "Brasil" "Brasil"
##   .. ..$ isbn                    : chr [1:4] "9788578111465" "9788536243559" "9788536242514" "9788564898820"
##   .. ..$ nome_da_editora         : chr [1:4] "Ipea" "Juruá editora psicologia biblioteca" "Juruá editora" ""
##   .. ..$ numero_da_edicao_revisao: chr [1:4] "1" "22" "22" "1"
##   .. ..$ organizadores           : chr [1:4] "Lenita Maria Turchi; João Alberto de Negri; Alvaro Comin" "Fernando de Oliveira Viera; Ana Magnolia Mendes, Alvaro Roberto Crespo de Melo" "Leda Gonçalves de Freitas" "Marilia Steinberger"
##   .. ..$ paginas                 : chr [1:4] "421 - 464" "409 - 413" "49 - 67" "1 - "
##   .. ..$ autores                 :List of 4
##   .. ..$ autores-endogeno        :List of 4
##   ..$ DEMAIS_TIPOS_DE_PRODUCAO_BIBLIOGRAFICA:'data.frame':   1 obs. of  9 variables:
##   .. ..$ natureza          : chr "artigo"
##   .. ..$ titulo            : chr "A arte de trabalhar"
##   .. ..$ ano               : chr "2014"
##   .. ..$ pais_de_publicacao: chr "Brasil"
##   .. ..$ editora           : chr "UFMG belas artes"
##   .. ..$ doi               : chr ""
##   .. ..$ numero_de_paginas : chr "57-70"
##   .. ..$ autores           :List of 1
##   .. ..$ autores-endogeno  :List of 1
##   ..$ LIVRO                                 :'data.frame':   1 obs. of  13 variables:
##   .. ..$ titulo                  : chr "Manifeste convivialiste- déclaration d´interdépendance"
##   .. ..$ ano                     : chr "2013"
##   .. ..$ tipo                    : chr "LIVRO_PUBLICADO"
##   .. ..$ natureza                : chr "COLETANEA"
##   .. ..$ pais_de_publicacao      : chr "França"
##   .. ..$ isbn                    : chr "9782356872517"
##   .. ..$ doi                     : chr ""
##   .. ..$ nome_da_editora         : chr "Le Bord de l´Eau"
##   .. ..$ numero_da_edicao_revisao: chr "1"
##   .. ..$ numero_de_paginas       : chr "39"
##   .. ..$ numero_de_volumes       : chr "1"
##   .. ..$ autores                 :List of 1
##   .. ..$ autores-endogeno        :List of 1
##   ..$ PERIODICO                             :'data.frame':   4 obs. of  10 variables:
##   .. ..$ natureza        : chr [1:4] "COMPLETO" "COMPLETO" "COMPLETO" "COMPLETO"
##   .. ..$ titulo          : chr [1:4] "Rumo a um novo mercado: uma abordagem sociológica do comércio justo e solidário" "Entre o prescrito e o Real: O papel da subjetividade na efetivação dos direitos das empregadas domesticas" "A arte de trabalhar" "Planejamento urbano, arquitetura e urbanismo: a serviços de uma outra geografia? Brasilmar Ferreira Nunes (em memória)"
##   .. ..$ periodico       : chr [1:4] "Mercado de Trabalho (Rio de Janeiro. 1996)" "Sociedade e Estado (UnB. Impresso)" "Lamparina: Revista de Ensino do Teatro" "Sociedade e Estado"
##   .. ..$ ano             : chr [1:4] "2011" "2013" "2014" "2016"
##   .. ..$ volume          : chr [1:4] "49" "28" "1" "31"
##   .. ..$ issn            : chr [1:4] "16760883" "01026992" "21776121" "19805462"
##   .. ..$ paginas         : chr [1:4] "67 - 77" "587-606 - " "57 - 70" "1980-5462 - "
##   .. ..$ doi             : chr [1:4] "" "" "" ""
##   .. ..$ autores         :List of 4
##   .. ..$ autores-endogeno:List of 4
##   ..$ TEXTO_EM_JORNAIS                      :'data.frame':   1 obs. of  9 variables:
##   .. ..$ natureza        : chr "JORNAL_DE_NOTICIAS"
##   .. ..$ titulo          : chr "Sobre a perpetuação da pobreza"
##   .. ..$ periodico       : chr "Correio Braziliense"
##   .. ..$ ano             : chr "2017"
##   .. ..$ issn            : chr ""
##   .. ..$ paginas         : chr " - "
##   .. ..$ doi             : chr ""
##   .. ..$ autores         :List of 1
##   .. ..$ autores-endogeno:List of 1
##  $ orientacoes_academicas:List of 5
##   ..$ ORIENTACAO_CONCLUIDA_DOUTORADO   :'data.frame':    11 obs. of  13 variables:
##   .. ..$ natureza                   : chr [1:11] "Tese de doutorado" "Tese de doutorado" "Tese de doutorado" "Tese de doutorado" ...
##   .. ..$ titulo                     : chr [1:11] "?A construção política e o engajamento militante na economia solidária?" "?Sentidos da prática docente ? o caso da Universidade Católica de Brasília?" "?Análise da ambigüidade discursiva em uma experiência na economia solidária?" "Direito ao trabalho e conquista da cidadania: um estudo sobre cooperativas sociais no contexto da economia solidária" ...
##   .. ..$ ano                        : chr [1:11] "2010" "2010" "2010" "2010" ...
##   .. ..$ id_lattes_aluno            : chr [1:11] "0580793909196813" "" "8596631752170536" "8450612520149090" ...
##   .. ..$ nome_aluno                 : chr [1:11] "Jonas de Oliveira Bertucci" "Ricardo Spíndola Mariz" "Simone Aparecida Lisniowski" "Rita de Cássia Andrade Martins" ...
##   .. ..$ instituicao                : chr [1:11] "Universidade de Brasília" "Universidade de Brasília" "Universidade de Brasília" "Universidade de Brasília" ...
##   .. ..$ curso                      : chr [1:11] "Sociologia" "Sociologia" "Sociologia" "Sociologia" ...
##   .. ..$ codigo_do_curso            : chr [1:11] "51500094" "51500094" "51500094" "51500094" ...
##   .. ..$ bolsa                      : chr [1:11] "NAO" "NAO" "NAO" "NAO" ...
##   .. ..$ agencia_financiadora       : chr [1:11] "" "" "" "" ...
##   .. ..$ codigo_agencia_financiadora: chr [1:11] "" "" "" "" ...
##   .. ..$ nome_orientadores          :List of 11
##   .. ..$ id_lattes_orientadores     :List of 11
##   ..$ ORIENTACAO_CONCLUIDA_MESTRADO    :'data.frame':    5 obs. of  13 variables:
##   .. ..$ natureza                   : chr [1:5] "Dissertação de mestrado" "Dissertação de mestrado" "Dissertação de mestrado" "Dissertação de mestrado" ...
##   .. ..$ titulo                     : chr [1:5] "Construção do Mercado Solidário Brasileiro: contribuições das redes econômicas solidárias" "Peguei o diploma, e agora? - Desafios, dilemas e estratégias de inserção ocupacional de jovens recém-graduados "| __truncated__ "Formas de Vida, Formas de Conhecimento e Processos de Mudança Sócio-Cultural" "Reestruturação Produtiva da Economia e Terceirização (ou semi-escravidão?): os sentidos do trabalho e as disput"| __truncated__ ...
##   .. ..$ ano                        : chr [1:5] "2011" "2011" "2012" "2014" ...
##   .. ..$ id_lattes_aluno            : chr [1:5] "" "9756886246113422" "5142964346538200" "2068612156860278" ...
##   .. ..$ nome_aluno                 : chr [1:5] "Ivette Tatiana Castilla Carrascal" "Tauvana da Silva Yung" "Thamires Castelar Torres Sales" "Samuel Nogueira Costa" ...
##   .. ..$ instituicao                : chr [1:5] "Universidade de Brasília" "Universidade de Brasília" "Universidade de Brasília" "Universidade de Brasília" ...
##   .. ..$ curso                      : chr [1:5] "Sociologia" "Sociologia" "Sociologia" "Sociologia" ...
##   .. ..$ codigo_do_curso            : chr [1:5] "51500094" "51500094" "51500094" "51500094" ...
##   .. ..$ bolsa                      : chr [1:5] "NAO" "NAO" "SIM" "NAO" ...
##   .. ..$ agencia_financiadora       : chr [1:5] "" "" "" "" ...
##   .. ..$ codigo_agencia_financiadora: chr [1:5] "" "" "" "" ...
##   .. ..$ nome_orientadores          :List of 5
##   .. ..$ id_lattes_orientadores     :List of 5
##   ..$ ORIENTACAO_EM_ANDAMENTO_DOUTORADO:'data.frame':    5 obs. of  13 variables:
##   .. ..$ natureza                   : chr [1:5] "Tese de doutorado" "Tese de doutorado" "Tese de doutorado" "Tese de doutorado" ...
##   .. ..$ titulo                     : chr [1:5] "A definir" "A definir" "A definir" "A definir" ...
##   .. ..$ ano                        : chr [1:5] "2015" "2016" "2016" "2016" ...
##   .. ..$ id_lattes_aluno            : chr [1:5] "" "2587452729961465" "" "2881285886940119" ...
##   .. ..$ nome_aluno                 : chr [1:5] "Tauvana da Silva Yung" "Márcio Henrique de Carvalho" "Carolina Vicente Ferreira Lima" "Marcello Cavalcanti Barra" ...
##   .. ..$ instituicao                : chr [1:5] "Universidade de Brasília" "Universidade de Brasília" "Universidade de Brasília" "Universidade de Brasília" ...
##   .. ..$ curso                      : chr [1:5] "Sociologia" "Sociologia" "Sociologia" "Sociologia" ...
##   .. ..$ codigo_do_curso            : chr [1:5] "51500094" "51500094" "51500094" "51500094" ...
##   .. ..$ bolsa                      : chr [1:5] "NAO" "NAO" "SIM" "NAO" ...
##   .. ..$ agencia_financiadora       : chr [1:5] "" "" "Coordenação de Aperfeiçoamento de Pessoal de Nível Superior" "" ...
##   .. ..$ codigo_agencia_financiadora: chr [1:5] "" "" "045000000000" "" ...
##   .. ..$ nome_orientadores          :List of 5
##   .. ..$ id_lattes_orientadores     :List of 5
##   ..$ ORIENTACAO_EM_ANDAMENTO_GRADUACAO:'data.frame':    1 obs. of  13 variables:
##   .. ..$ natureza                   : chr "Trabalho de conclusão de curso de graduação"
##   .. ..$ titulo                     : chr "assédio moral e educação"
##   .. ..$ ano                        : chr "2014"
##   .. ..$ id_lattes_aluno            : chr ""
##   .. ..$ nome_aluno                 : chr "Orion Macunaima Basso Coppe"
##   .. ..$ instituicao                : chr "Universidade de Brasília"
##   .. ..$ curso                      : chr "Sociologia"
##   .. ..$ codigo_do_curso            : chr "90000014"
##   .. ..$ bolsa                      : chr "NAO"
##   .. ..$ agencia_financiadora       : chr ""
##   .. ..$ codigo_agencia_financiadora: chr ""
##   .. ..$ nome_orientadores          :List of 1
##   .. ..$ id_lattes_orientadores     :List of 1
##   ..$ OUTRAS_ORIENTACOES_CONCLUIDAS    :'data.frame':    1 obs. of  13 variables:
##   .. ..$ natureza                   : chr "TRABALHO_DE_CONCLUSAO_DE_CURSO_GRADUACAO"
##   .. ..$ titulo                     : chr "Trabalho e questão racial"
##   .. ..$ ano                        : chr "2014"
##   .. ..$ id_lattes_aluno            : chr ""
##   .. ..$ nome_aluno                 : chr "Marina Macedo Araujo"
##   .. ..$ instituicao                : chr "Universidade de Brasília"
##   .. ..$ curso                      : chr "Sociologia"
##   .. ..$ codigo_do_curso            : chr "90000014"
##   .. ..$ bolsa                      : chr "NAO"
##   .. ..$ agencia_financiadora       : chr ""
##   .. ..$ codigo_agencia_financiadora: chr ""
##   .. ..$ nome_orientadores          :List of 1
##   .. ..$ id_lattes_orientadores     :List of 1
##  $ senioridade           : chr "8"
\end{verbatim}

\paragraph{Potencial de utilização dos dados do perfil dos
docentes}\label{potencial-de-utilizacao-dos-dados-do-perfil-dos-docentes}

Esses dados terão potencial para responder às questões de \emph{data
mining}? O que é possível gerar a partir desses dados, para o conjunto
dos 1764 docentes da UnB? A fim de compreender a relevância dos dados
para a avaliação da produção acadêmica nas pós-graduações brasileiras
pode-se recorrer a trabalhos como os seguintes:

\begin{itemize}
\tightlist
\item
  Leite (2018) apresenta, em suas ``Considerações básicas sobre a
  Avaliação do Sistema Nacional de Pós-Graduação'', o conjunto dos itens
  que são tópicos de avaliação das pós-graduações pela CAPES, e que
  envolvem, entre outros:

  \begin{itemize}
  \tightlist
  \item
    Avaliação do corpo docente, com 20\% a 30\% de peso na avaliação
    total do programa, a depender do seu tipo. Analisando-se de forma
    mais detalhada os critérios de avaliação do corpo docente, indicados
    por Leite, o que é possível gerar com base nos dados disponíveis em
    unb.prof? Há dados que permitam identificar o perfil do docente,
    como proposto pela CAPES, inclusive no documento de área específica
    na qual atua o pesquisador? Que outros aspectos relevantes para a
    CAPES podem ser levantados com base nos dados dessa fonte?
  \item
    Avaliação do corpo discente, Teses e dissertações, com 30\% a 20\%
    de peso na avaliação total do programa, a depender de seu tipo. Os
    dados sobre orientação permitem fazer quais tipos de avaliações do
    corpo discente?
  \item
    Avaliação da produção intelectual, com 40\% de peso na avaliação
    total. Qual a relevância dos dados em unb.prof para essa avaliação?
    Que outros arquivos podem melhor subsidiar essa avaliação?
  \end{itemize}
\item
  Em busca de considerar outros fatores relevantes para a avaliação da
  pós-graduação, não considerados no modelo da CAPES, pode-se recorrer
  ao trabalho de Kalpazidou Schmidt e Graversen (2018), que apresentam
  um conjunto de fatores persistentes que facilitam a existência de
  ambientes de pesquisa inovadores e dinâmicos, dentre os quais se
  destaca:

  \begin{itemize}
  \tightlist
  \item
    Atividade em pesquisas com elevado grau de impacto social;
  \item
    Promoção de elevado grau de autonomia individual, tanto do ponto de
    vista teórico quanto metodológico;
  \item
    Possuem um bom clima de trabalho, baseado no trabalho em times;
  \item
    São internacioinalmente bem conhecidas etc.
  \end{itemize}

  Estariam esses fatores contemplados, de alguma forma, memso que
  parcialmente, nos dados presentes em unb.prof? Ou em qualquer outros
  dos arquivos? Cabe explorar.
\end{itemize}

\subsubsection{Descrição dos dados de
orientações}\label{descricao-dos-dados-de-orientacoes}

\begin{Shaded}
\begin{Highlighting}[]
\NormalTok{unb.adv <-}\StringTok{ }\KeywordTok{fromJSON}\NormalTok{(}\StringTok{"unbpos/unbpos.advise.json"}\NormalTok{)}
\KeywordTok{names}\NormalTok{(unb.adv)}
\end{Highlighting}
\end{Shaded}

\begin{verbatim}
## [1] "ORIENTACAO_EM_ANDAMENTO_DE_POS_DOUTORADO"    
## [2] "ORIENTACAO_EM_ANDAMENTO_DOUTORADO"           
## [3] "ORIENTACAO_EM_ANDAMENTO_MESTRADO"            
## [4] "ORIENTACAO_EM_ANDAMENTO_GRADUACAO"           
## [5] "ORIENTACAO_EM_ANDAMENTO_INICIACAO_CIENTIFICA"
## [6] "ORIENTACAO_CONCLUIDA_POS_DOUTORADO"          
## [7] "ORIENTACAO_CONCLUIDA_DOUTORADO"              
## [8] "ORIENTACAO_CONCLUIDA_MESTRADO"               
## [9] "OUTRAS_ORIENTACOES_CONCLUIDAS"
\end{verbatim}

\begin{Shaded}
\begin{Highlighting}[]
\KeywordTok{names}\NormalTok{(unb.adv}\OperatorTok{$}\NormalTok{ORIENTACAO_CONCLUIDA_DOUTORADO)}
\end{Highlighting}
\end{Shaded}

\begin{verbatim}
## [1] "2010" "2011" "2012" "2013" "2014" "2015" "2016" "2017"
\end{verbatim}

\begin{Shaded}
\begin{Highlighting}[]
\KeywordTok{length}\NormalTok{(unb.adv}\OperatorTok{$}\NormalTok{ORIENTACAO_CONCLUIDA_DOUTORADO}\OperatorTok{$}\StringTok{`}\DataTypeTok{2016}\StringTok{`}\OperatorTok{$}\NormalTok{natureza)}
\end{Highlighting}
\end{Shaded}

\begin{verbatim}
## [1] 606
\end{verbatim}

\begin{Shaded}
\begin{Highlighting}[]
\KeywordTok{head}\NormalTok{(}\KeywordTok{sort}\NormalTok{(}\KeywordTok{table}\NormalTok{(unb.adv}\OperatorTok{$}\NormalTok{ORIENTACAO_CONCLUIDA_DOUTORADO}\OperatorTok{$}\StringTok{`}\DataTypeTok{2017}\StringTok{`}\OperatorTok{$}\NormalTok{curso), }\DataTypeTok{decreasing =} \OtherTok{TRUE}\NormalTok{), }\DecValTok{10}\NormalTok{)}
\end{Highlighting}
\end{Shaded}

\begin{verbatim}
## 
##                           Ciências da Saúde 
##                                          17 
##                                   Geografia 
##                                          15 
##                                    Educação 
##                                          14 
##                Psicologia Clínica e Cultura 
##                                          14 
## Processos de Desenvolvimento Humano e Saúde 
##                                          13 
##                                    Economia 
##                                          12 
##                                   Geotecnia 
##                                          11 
##                             Biologia Animal 
##                                          10 
##                      Ciências da Informação 
##                                          10 
##                                    Geologia 
##                                          10
\end{verbatim}

\begin{Shaded}
\begin{Highlighting}[]
\KeywordTok{head}\NormalTok{(}\KeywordTok{sort}\NormalTok{(}\KeywordTok{table}\NormalTok{(unb.adv}\OperatorTok{$}\NormalTok{ORIENTACAO_CONCLUIDA_MESTRADO}\OperatorTok{$}\StringTok{`}\DataTypeTok{2017}\StringTok{`}\OperatorTok{$}\NormalTok{curso), }\DataTypeTok{decreasing =} \OtherTok{TRUE}\NormalTok{), }\DecValTok{10}\NormalTok{)}
\end{Highlighting}
\end{Shaded}

\begin{verbatim}
## 
##                                    Economia 
##                                          43 
##                                     Direito 
##                                          34 
##                           Ciências da Saúde 
##                                          28 
##             Ciências e Tecnologias em Saúde 
##                                          26 
##               Estruturas e Construção Civil 
##                                          25 
##                         Estudos de Tradução 
##                                          25 
##                                  Literatura 
##                                          21 
## Processos de Desenvolvimento Humano e Saúde 
##                                          20 
##                                    Geologia 
##                                          19 
##                          Ciências Mecânicas 
##                                          18
\end{verbatim}

\begin{Shaded}
\begin{Highlighting}[]
\NormalTok{ci.adv <-}\StringTok{ }\KeywordTok{fromJSON}\NormalTok{(}\StringTok{"./ciencia_informacao/ci.advise.json"}\NormalTok{)}
\KeywordTok{names}\NormalTok{(unb.adv) }\OperatorTok{==}\StringTok{ }\KeywordTok{names}\NormalTok{(ci.adv)}
\end{Highlighting}
\end{Shaded}

\begin{verbatim}
## [1] TRUE TRUE TRUE TRUE TRUE TRUE TRUE TRUE TRUE
\end{verbatim}

\begin{Shaded}
\begin{Highlighting}[]
\NormalTok{co.adv <-}\StringTok{ }\KeywordTok{fromJSON}\NormalTok{(}\StringTok{"./comunicacao/co.advise.json"}\NormalTok{)}
\NormalTok{ps.adv <-}\StringTok{ }\KeywordTok{fromJSON}\NormalTok{(}\StringTok{"./politica_social/ps.advise.json"}\NormalTok{)}
\NormalTok{so.adv <-}\StringTok{ }\KeywordTok{fromJSON}\NormalTok{(}\StringTok{"./sociologia/so.advise.json"}\NormalTok{)}
\end{Highlighting}
\end{Shaded}

A mesma leitura feita com o arquivo \texttt{unb.adv} pode ser realizada
com os arquivos dos programas específicos.

\subsubsection{Descrição dos dados de produção
bibliográfica}\label{descricao-dos-dados-de-producao-bibliografica}

\begin{Shaded}
\begin{Highlighting}[]
\NormalTok{unb.pub <-}\StringTok{ }\KeywordTok{fromJSON}\NormalTok{(}\StringTok{"unbpos/unbpos.publication.json"}\NormalTok{)}
\KeywordTok{names}\NormalTok{(unb.pub)}
\end{Highlighting}
\end{Shaded}

\begin{verbatim}
## [1] "PERIODICO"                             
## [2] "LIVRO"                                 
## [3] "CAPITULO_DE_LIVRO"                     
## [4] "TEXTO_EM_JORNAIS"                      
## [5] "EVENTO"                                
## [6] "ARTIGO_ACEITO"                         
## [7] "DEMAIS_TIPOS_DE_PRODUCAO_BIBLIOGRAFICA"
\end{verbatim}

\begin{Shaded}
\begin{Highlighting}[]
\KeywordTok{names}\NormalTok{(unb.pub}\OperatorTok{$}\NormalTok{PERIODICO}\OperatorTok{$}\StringTok{`}\DataTypeTok{2012}\StringTok{`}\NormalTok{)}
\end{Highlighting}
\end{Shaded}

\begin{verbatim}
##  [1] "natureza"         "titulo"           "periodico"       
##  [4] "ano"              "volume"           "issn"            
##  [7] "paginas"          "doi"              "autores"         
## [10] "autores-endogeno"
\end{verbatim}

\begin{Shaded}
\begin{Highlighting}[]
\KeywordTok{head}\NormalTok{(}\KeywordTok{sort}\NormalTok{(}\KeywordTok{table}\NormalTok{(unb.pub}\OperatorTok{$}\NormalTok{PERIODICO}\OperatorTok{$}\StringTok{`}\DataTypeTok{2017}\StringTok{`}\OperatorTok{$}\NormalTok{periodico), }\DataTypeTok{decreasing =} \OtherTok{TRUE}\NormalTok{), }\DecValTok{10}\NormalTok{)}
\end{Highlighting}
\end{Shaded}

\begin{verbatim}
## 
##               REVISTA DE SAÚDE PÚBLICA (ONLINE) 
##                                              23 
##                                        PLoS One 
##                                              21 
##                              ESPACIOS (CARACAS) 
##                                              16 
##                              Scientific Reports 
##                                              16 
##                        Ciencia & Saude Coletiva 
##                                              15 
##                 GENETICS AND MOLECULAR RESEARCH 
##                                              15 
##                          CADERNOS DE PROSPECÇÃO 
##                                              14 
##        JOURNAL OF SOUTH AMERICAN EARTH SCIENCES 
##                                              14 
##           Journal of Molecular Modeling (Print) 
##                                              13 
## RBC. REVISTA BRASILEIRA DE CARTOGRAFIA (ONLINE) 
##                                              13
\end{verbatim}

\begin{Shaded}
\begin{Highlighting}[]
\KeywordTok{head}\NormalTok{(}\KeywordTok{sort}\NormalTok{(}\KeywordTok{table}\NormalTok{(unb.pub}\OperatorTok{$}\NormalTok{LIVRO}\OperatorTok{$}\StringTok{`}\DataTypeTok{2015}\StringTok{`}\OperatorTok{$}\NormalTok{nome_da_editora), }\DataTypeTok{decreasing =} \OtherTok{TRUE}\NormalTok{), }\DecValTok{10}\NormalTok{)}
\end{Highlighting}
\end{Shaded}

\begin{verbatim}
## 
##                            ANPOF                                  
##                               23                               13 
##         Novas Edições Acadêmicas            Laccademia Publishing 
##                               12                                8 
## Editora Universidade de Brasília                  Pontes Editores 
##                                7                                7 
##                      Lumen Juris                       Fino Traço 
##                                6                                5 
##                         INEP/MEC                              LTr 
##                                5                                5
\end{verbatim}

\begin{Shaded}
\begin{Highlighting}[]
\NormalTok{ci.pub <-}\StringTok{ }\KeywordTok{fromJSON}\NormalTok{(}\StringTok{"./ciencia_informacao/ci.publication.json"}\NormalTok{)}
\NormalTok{co.pub <-}\StringTok{ }\KeywordTok{fromJSON}\NormalTok{(}\StringTok{"./comunicacao/co.publication.json"}\NormalTok{)}
\NormalTok{ps.pub <-}\StringTok{ }\KeywordTok{fromJSON}\NormalTok{(}\StringTok{"./politica_social/ps.publication.json"}\NormalTok{)}
\NormalTok{so.pub <-}\StringTok{ }\KeywordTok{fromJSON}\NormalTok{(}\StringTok{"./sociologia/so.publication.json"}\NormalTok{)}
\end{Highlighting}
\end{Shaded}

A mesma leitura feita com o arquivo \texttt{unb.pub} pode ser realizada
com os arquivos dos programas específicos.

\subsubsection{Descrição dos dados de agregação de docentes por
área}\label{descricao-dos-dados-de-agregacao-de-docentes-por-area}

\begin{Shaded}
\begin{Highlighting}[]
\NormalTok{unb.area <-}\StringTok{ }\KeywordTok{fromJSON}\NormalTok{(}\StringTok{"./unbpos/unbpos.researchers_by_area.json"}\NormalTok{)}
\NormalTok{unb.area.df <-}\StringTok{ }\KeywordTok{cbind}\NormalTok{(}\KeywordTok{names}\NormalTok{(unb.area}\OperatorTok{$}\StringTok{`}\DataTypeTok{Areas dos pesquisadores}\StringTok{`}\NormalTok{),}
\NormalTok{           (}\KeywordTok{sapply}\NormalTok{(unb.area}\OperatorTok{$}\StringTok{`}\DataTypeTok{Areas dos pesquisadores}\StringTok{`}\NormalTok{, }\ControlFlowTok{function}\NormalTok{(x) }\KeywordTok{length}\NormalTok{(x))))}
\KeywordTok{rownames}\NormalTok{(unb.area.df) <-}\StringTok{ }\KeywordTok{c}\NormalTok{(}\DecValTok{1}\OperatorTok{:}\KeywordTok{nrow}\NormalTok{(unb.area.df)); }\KeywordTok{colnames}\NormalTok{(unb.area.df) <-}\StringTok{ }\KeywordTok{c}\NormalTok{(}\StringTok{"Area"}\NormalTok{, }\StringTok{"Professores"}\NormalTok{)}
\KeywordTok{glimpse}\NormalTok{(unb.area.df)}
\end{Highlighting}
\end{Shaded}

\begin{verbatim}
##  chr [1:85, 1:2] "Administração" "Agronomia" "Antropologia" ...
##  - attr(*, "dimnames")=List of 2
##   ..$ : chr [1:85] "1" "2" "3" "4" ...
##   ..$ : chr [1:2] "Area" "Professores"
\end{verbatim}

\begin{Shaded}
\begin{Highlighting}[]
\KeywordTok{head}\NormalTok{(unb.area.df[])}
\end{Highlighting}
\end{Shaded}

\begin{verbatim}
##   Area                      Professores
## 1 "Administração"           "101"      
## 2 "Agronomia"               "65"       
## 3 "Antropologia"            "52"       
## 4 "Arqueologia"             "2"        
## 5 "Arquitetura e Urbanismo" "42"       
## 6 "Artes"                   "85"
\end{verbatim}

\subsubsection{Descrição dos dados de redes de
colaboração}\label{descricao-dos-dados-de-redes-de-colaboracao}

\subsection{CRISP-DM Fase.Atividade 2.3 - Análise exploratória dos
dados}\label{crisp-dm-fase.atividade-2.3---analise-exploratoria-dos-dados}

Como já informado, a análise exploratória dos dados possibilita um
entendimento mais profundo da relação estatística existente entre os
dados dos \emph{datasets} para um melhor entendimento da qualidade
daqueles dados para os objetivos do projeto.

Como já informado, a análise exploratória dos dados é responsabilidade
parcial dos alunos, tendo em vista que este relatório apresenta uma
análise exploratória preliminar. O relatório final deve conter análises
exploratórias dos dados que sejam significativas e aprofundadas.

\subsubsection{Arquivo Profile}\label{arquivo-profile}

\begin{Shaded}
\begin{Highlighting}[]
\CommentTok{# jsonedit(unb.prof)}
\CommentTok{# Número de áreas de atuação cumulativo}
\KeywordTok{sum}\NormalTok{(}\KeywordTok{sapply}\NormalTok{(unb.prof, }\ControlFlowTok{function}\NormalTok{(x) }\KeywordTok{nrow}\NormalTok{(x}\OperatorTok{$}\NormalTok{areas_de_atuacao)))}
\end{Highlighting}
\end{Shaded}

\begin{verbatim}
## [1] 7119
\end{verbatim}

\begin{Shaded}
\begin{Highlighting}[]
\CommentTok{# Número de áreas de atuação por pessoa}
\KeywordTok{table}\NormalTok{(}\KeywordTok{unlist}\NormalTok{(}\KeywordTok{sapply}\NormalTok{(unb.prof, }\ControlFlowTok{function}\NormalTok{(x) }\KeywordTok{nrow}\NormalTok{(x}\OperatorTok{$}\NormalTok{areas_de_atuacao))))}
\end{Highlighting}
\end{Shaded}

\begin{verbatim}
## 
##   1   2   3   4   5   6  10 
## 119 205 350 331 342 416   1
\end{verbatim}

\begin{Shaded}
\begin{Highlighting}[]
\CommentTok{# Número de pessoas por grande area}
\KeywordTok{table}\NormalTok{(}\KeywordTok{unlist}\NormalTok{(}\KeywordTok{sapply}\NormalTok{(unb.prof, }\ControlFlowTok{function}\NormalTok{(x) (x}\OperatorTok{$}\NormalTok{areas_de_atuacao}\OperatorTok{$}\NormalTok{grande_area))))}
\end{Highlighting}
\end{Shaded}

\begin{verbatim}
## 
##                                     CIENCIAS_AGRARIAS 
##                         27                        427 
##        CIENCIAS_BIOLOGICAS          CIENCIAS_DA_SAUDE 
##                        780                        601 
## CIENCIAS_EXATAS_E_DA_TERRA           CIENCIAS_HUMANAS 
##                       1075                       1641 
## CIENCIAS_SOCIAIS_APLICADAS                ENGENHARIAS 
##                       1172                        697 
## LINGUISTICA_LETRAS_E_ARTES                     OUTROS 
##                        634                         65
\end{verbatim}

\begin{Shaded}
\begin{Highlighting}[]
\CommentTok{# Número de pessoas que produziram os específicos tipos de produção}
\KeywordTok{table}\NormalTok{(}\KeywordTok{unlist}\NormalTok{(}\KeywordTok{sapply}\NormalTok{(unb.prof, }\ControlFlowTok{function}\NormalTok{(x) }\KeywordTok{names}\NormalTok{(x}\OperatorTok{$}\NormalTok{producao_bibiografica))))}
\end{Highlighting}
\end{Shaded}

\begin{verbatim}
## 
##                          ARTIGO_ACEITO 
##                                    349 
##                      CAPITULO_DE_LIVRO 
##                                   1346 
## DEMAIS_TIPOS_DE_PRODUCAO_BIBLIOGRAFICA 
##                                    532 
##                                 EVENTO 
##                                   1504 
##                                  LIVRO 
##                                    867 
##                              PERIODICO 
##                                   1716 
##                       TEXTO_EM_JORNAIS 
##                                    492
\end{verbatim}

\begin{Shaded}
\begin{Highlighting}[]
\CommentTok{# Número de publicações por tipo}
\KeywordTok{sum}\NormalTok{(}\KeywordTok{sapply}\NormalTok{(unb.prof, }\ControlFlowTok{function}\NormalTok{(x) }\KeywordTok{length}\NormalTok{(x}\OperatorTok{$}\NormalTok{producao_bibiografica}\OperatorTok{$}\NormalTok{ARTIGO_ACEITO}\OperatorTok{$}\NormalTok{ano)))}
\end{Highlighting}
\end{Shaded}

\begin{verbatim}
## [1] 563
\end{verbatim}

\begin{Shaded}
\begin{Highlighting}[]
\KeywordTok{sum}\NormalTok{(}\KeywordTok{sapply}\NormalTok{(unb.prof, }\ControlFlowTok{function}\NormalTok{(x) }\KeywordTok{length}\NormalTok{(x}\OperatorTok{$}\NormalTok{producao_bibiografica}\OperatorTok{$}\NormalTok{CAPITULO_DE_LIVRO}\OperatorTok{$}\NormalTok{ano)))}
\end{Highlighting}
\end{Shaded}

\begin{verbatim}
## [1] 8816
\end{verbatim}

\begin{Shaded}
\begin{Highlighting}[]
\KeywordTok{sum}\NormalTok{(}\KeywordTok{sapply}\NormalTok{(unb.prof, }\ControlFlowTok{function}\NormalTok{(x) }\KeywordTok{length}\NormalTok{(x}\OperatorTok{$}\NormalTok{producao_bibiografica}\OperatorTok{$}\NormalTok{LIVRO}\OperatorTok{$}\NormalTok{ano)))}
\end{Highlighting}
\end{Shaded}

\begin{verbatim}
## [1] 2932
\end{verbatim}

\begin{Shaded}
\begin{Highlighting}[]
\KeywordTok{sum}\NormalTok{(}\KeywordTok{sapply}\NormalTok{(unb.prof, }\ControlFlowTok{function}\NormalTok{(x) }\KeywordTok{length}\NormalTok{(x}\OperatorTok{$}\NormalTok{producao_bibiografica}\OperatorTok{$}\NormalTok{PERIODICO}\OperatorTok{$}\NormalTok{ano)))}
\end{Highlighting}
\end{Shaded}

\begin{verbatim}
## [1] 30352
\end{verbatim}

\begin{Shaded}
\begin{Highlighting}[]
\KeywordTok{sum}\NormalTok{(}\KeywordTok{sapply}\NormalTok{(unb.prof, }\ControlFlowTok{function}\NormalTok{(x) }\KeywordTok{length}\NormalTok{(x}\OperatorTok{$}\NormalTok{producao_bibiografica}\OperatorTok{$}\NormalTok{TEXTO_EM_JORNAIS}\OperatorTok{$}\NormalTok{ano)))}
\end{Highlighting}
\end{Shaded}

\begin{verbatim}
## [1] 3042
\end{verbatim}

\begin{Shaded}
\begin{Highlighting}[]
\CommentTok{# Número de pessoas por quantitativo de produções por pessoa 0 = 1; 1 = 2...}
\KeywordTok{table}\NormalTok{(}\KeywordTok{unlist}\NormalTok{(}\KeywordTok{sapply}\NormalTok{(unb.prof, }\ControlFlowTok{function}\NormalTok{(x) }\KeywordTok{length}\NormalTok{(x}\OperatorTok{$}\NormalTok{producao_bibiografica}\OperatorTok{$}\NormalTok{ARTIGO_ACEITO}\OperatorTok{$}\NormalTok{ano))))}
\end{Highlighting}
\end{Shaded}

\begin{verbatim}
## 
##    0    1    2    3    4    5    6    7    9   10   11   15 
## 1415  242   66   21    8    2    3    3    1    1    1    1
\end{verbatim}

\begin{Shaded}
\begin{Highlighting}[]
\KeywordTok{table}\NormalTok{(}\KeywordTok{unlist}\NormalTok{(}\KeywordTok{sapply}\NormalTok{(unb.prof, }\ControlFlowTok{function}\NormalTok{(x) }\KeywordTok{length}\NormalTok{(x}\OperatorTok{$}\NormalTok{producao_bibiografica}\OperatorTok{$}\NormalTok{CAPITULO_DE_LIVRO}\OperatorTok{$}\NormalTok{ano))))}
\end{Highlighting}
\end{Shaded}

\begin{verbatim}
## 
##   0   1   2   3   4   5   6   7   8   9  10  11  12  13  14  15  16  17 
## 418 258 204 139 130  99  68  64  52  43  32  36  31  33  19  14  10  11 
##  18  19  20  21  22  23  24  25  26  27  28  29  30  31  32  33  34  37 
##  13   9  11   6   7   5   6   4   2   7   4   1   3   4   2   3   3   1 
##  38  39  40  44  47  52  56  69 
##   3   1   2   1   1   1   2   1
\end{verbatim}

\begin{Shaded}
\begin{Highlighting}[]
\KeywordTok{table}\NormalTok{(}\KeywordTok{unlist}\NormalTok{(}\KeywordTok{sapply}\NormalTok{(unb.prof, }\ControlFlowTok{function}\NormalTok{(x) }\KeywordTok{length}\NormalTok{(x}\OperatorTok{$}\NormalTok{producao_bibiografica}\OperatorTok{$}\NormalTok{LIVRO}\OperatorTok{$}\NormalTok{ano))))}
\end{Highlighting}
\end{Shaded}

\begin{verbatim}
## 
##   0   1   2   3   4   5   6   7   8   9  10  11  12  14  15  16  18  19 
## 897 322 189 100  67  50  29  29  19  14  12   7   8   2   2   3   2   3 
##  20  21  26  28  31  32  39  49 
##   1   2   1   1   1   1   1   1
\end{verbatim}

\begin{Shaded}
\begin{Highlighting}[]
\KeywordTok{table}\NormalTok{(}\KeywordTok{unlist}\NormalTok{(}\KeywordTok{sapply}\NormalTok{(unb.prof, }\ControlFlowTok{function}\NormalTok{(x) }\KeywordTok{length}\NormalTok{(x}\OperatorTok{$}\NormalTok{producao_bibiografica}\OperatorTok{$}\NormalTok{PERIODICO}\OperatorTok{$}\NormalTok{ano))))}
\end{Highlighting}
\end{Shaded}

\begin{verbatim}
## 
##   0   1   2   3   4   5   6   7   8   9  10  11  12  13  14  15  16  17 
##  48  55  63  74  74 101  83  86  66  75  69  61  60  53  46  47  48  27 
##  18  19  20  21  22  23  24  25  26  27  28  29  30  31  32  33  34  35 
##  34  44  32  38  31  32  20  25  19  26  22  17  18  13  13  11  21  10 
##  36  37  38  39  40  41  42  43  44  45  46  47  48  49  50  51  52  53 
##  10  11  13   6   9   8   7   6   7  14   5   8   5   4   1   6   9   2 
##  54  55  56  57  58  59  60  61  62  63  64  66  67  68  69  70  71  73 
##   1   3   2   3   3   3   2   4   2   6   2   3   2   3   1   1   4   2 
##  74  75  76  77  78  83  86  88  89  90 103 104 126 146 222 233 
##   1   1   1   4   3   1   2   1   2   2   1   1   1   1   1   1
\end{verbatim}

\begin{Shaded}
\begin{Highlighting}[]
\KeywordTok{table}\NormalTok{(}\KeywordTok{unlist}\NormalTok{(}\KeywordTok{sapply}\NormalTok{(unb.prof, }\ControlFlowTok{function}\NormalTok{(x) }\KeywordTok{length}\NormalTok{(x}\OperatorTok{$}\NormalTok{producao_bibiografica}\OperatorTok{$}\NormalTok{TEXTO_EM_JORNAIS}\OperatorTok{$}\NormalTok{ano))))}
\end{Highlighting}
\end{Shaded}

\begin{verbatim}
## 
##    0    1    2    3    4    5    6    7    8    9   10   11   12   14   15 
## 1272  219   92   46   25   20    9    9   10    4    9    4    5    5    3 
##   17   18   19   21   22   23   25   27   29   30   31   34   35   38   39 
##    1    5    1    1    1    1    1    1    3    2    1    1    1    1    1 
##   49   51   67   86  109  146  148  176  178  181 
##    1    1    1    1    1    1    1    1    1    1
\end{verbatim}

\begin{Shaded}
\begin{Highlighting}[]
\CommentTok{# Número de produções por ano}
\KeywordTok{table}\NormalTok{(}\KeywordTok{unlist}\NormalTok{(}\KeywordTok{sapply}\NormalTok{(unb.prof, }\ControlFlowTok{function}\NormalTok{(x) (x}\OperatorTok{$}\NormalTok{producao_bibiografica}\OperatorTok{$}\NormalTok{ARTIGO_ACEITO}\OperatorTok{$}\NormalTok{ano))))}
\end{Highlighting}
\end{Shaded}

\begin{verbatim}
## 
## 2010 2011 2012 2013 2014 2015 2016 2017 
##   21   17   29   44   48   60   93  251
\end{verbatim}

\begin{Shaded}
\begin{Highlighting}[]
\KeywordTok{table}\NormalTok{(}\KeywordTok{unlist}\NormalTok{(}\KeywordTok{sapply}\NormalTok{(unb.prof, }\ControlFlowTok{function}\NormalTok{(x) (x}\OperatorTok{$}\NormalTok{producao_bibiografica}\OperatorTok{$}\NormalTok{CAPITULO_DE_LIVRO}\OperatorTok{$}\NormalTok{ano))))}
\end{Highlighting}
\end{Shaded}

\begin{verbatim}
## 
## 2010 2011 2012 2013 2014 2015 2016 2017 
## 1042 1052 1325  891 1135 1185 1162 1024
\end{verbatim}

\begin{Shaded}
\begin{Highlighting}[]
\KeywordTok{table}\NormalTok{(}\KeywordTok{unlist}\NormalTok{(}\KeywordTok{sapply}\NormalTok{(unb.prof, }\ControlFlowTok{function}\NormalTok{(x) (x}\OperatorTok{$}\NormalTok{producao_bibiografica}\OperatorTok{$}\NormalTok{LIVRO}\OperatorTok{$}\NormalTok{ano))))}
\end{Highlighting}
\end{Shaded}

\begin{verbatim}
## 
## 2010 2011 2012 2013 2014 2015 2016 2017 
##  353  301  373  383  388  426  371  337
\end{verbatim}

\begin{Shaded}
\begin{Highlighting}[]
\KeywordTok{table}\NormalTok{(}\KeywordTok{unlist}\NormalTok{(}\KeywordTok{sapply}\NormalTok{(unb.prof, }\ControlFlowTok{function}\NormalTok{(x) (x}\OperatorTok{$}\NormalTok{producao_bibiografica}\OperatorTok{$}\NormalTok{PERIODICO}\OperatorTok{$}\NormalTok{ano))))}
\end{Highlighting}
\end{Shaded}

\begin{verbatim}
## 
## 2010 2011 2012 2013 2014 2015 2016 2017 
## 3097 3193 3633 3859 3943 4154 4322 4151
\end{verbatim}

\begin{Shaded}
\begin{Highlighting}[]
\KeywordTok{table}\NormalTok{(}\KeywordTok{unlist}\NormalTok{(}\KeywordTok{sapply}\NormalTok{(unb.prof, }\ControlFlowTok{function}\NormalTok{(x) (x}\OperatorTok{$}\NormalTok{producao_bibiografica}\OperatorTok{$}\NormalTok{TEXTO_EM_JORNAIS}\OperatorTok{$}\NormalTok{ano))))}
\end{Highlighting}
\end{Shaded}

\begin{verbatim}
## 
## 2010 2011 2012 2013 2014 2015 2016 2017 
##  459  440  424  374  384  310  360  291
\end{verbatim}

\begin{Shaded}
\begin{Highlighting}[]
\CommentTok{# Número de pessoas que realizaram diferentes tipos de orientações}
\KeywordTok{length}\NormalTok{(}\KeywordTok{unlist}\NormalTok{(}\KeywordTok{sapply}\NormalTok{(unb.prof, }\ControlFlowTok{function}\NormalTok{(x) }\KeywordTok{names}\NormalTok{(x}\OperatorTok{$}\NormalTok{orientacoes_academicas))))}
\end{Highlighting}
\end{Shaded}

\begin{verbatim}
## [1] 7317
\end{verbatim}

\begin{Shaded}
\begin{Highlighting}[]
\CommentTok{# Número de pessoas por tipo de orientação}
\KeywordTok{table}\NormalTok{(}\KeywordTok{unlist}\NormalTok{(}\KeywordTok{sapply}\NormalTok{(unb.prof, }\ControlFlowTok{function}\NormalTok{(x) }\KeywordTok{names}\NormalTok{(x}\OperatorTok{$}\NormalTok{orientacoes_academicas))))}
\end{Highlighting}
\end{Shaded}

\begin{verbatim}
## 
##               ORIENTACAO_CONCLUIDA_DOUTORADO 
##                                          935 
##                ORIENTACAO_CONCLUIDA_MESTRADO 
##                                         1546 
##           ORIENTACAO_CONCLUIDA_POS_DOUTORADO 
##                                          267 
##            ORIENTACAO_EM_ANDAMENTO_DOUTORADO 
##                                         1061 
##            ORIENTACAO_EM_ANDAMENTO_GRADUACAO 
##                                          195 
## ORIENTACAO_EM_ANDAMENTO_INICIACAO_CIENTIFICA 
##                                          591 
##             ORIENTACAO_EM_ANDAMENTO_MESTRADO 
##                                         1168 
##                OUTRAS_ORIENTACOES_CONCLUIDAS 
##                                         1554
\end{verbatim}

\begin{Shaded}
\begin{Highlighting}[]
\CommentTok{#Número de orientações concluidas}
\KeywordTok{sum}\NormalTok{(}\KeywordTok{sapply}\NormalTok{(unb.prof, }\ControlFlowTok{function}\NormalTok{(x) }\KeywordTok{length}\NormalTok{(x}\OperatorTok{$}\NormalTok{orientacoes_academicas}\OperatorTok{$}\NormalTok{ORIENTACAO_CONCLUIDA_MESTRADO}\OperatorTok{$}\NormalTok{ano)))}
\end{Highlighting}
\end{Shaded}

\begin{verbatim}
## [1] 10875
\end{verbatim}

\begin{Shaded}
\begin{Highlighting}[]
\KeywordTok{sum}\NormalTok{(}\KeywordTok{sapply}\NormalTok{(unb.prof, }\ControlFlowTok{function}\NormalTok{(x) }\KeywordTok{length}\NormalTok{(x}\OperatorTok{$}\NormalTok{orientacoes_academicas}\OperatorTok{$}\NormalTok{ORIENTACAO_CONCLUIDA_DOUTORADO}\OperatorTok{$}\NormalTok{ano)))}
\end{Highlighting}
\end{Shaded}

\begin{verbatim}
## [1] 3899
\end{verbatim}

\begin{Shaded}
\begin{Highlighting}[]
\KeywordTok{sum}\NormalTok{(}\KeywordTok{sapply}\NormalTok{(unb.prof, }\ControlFlowTok{function}\NormalTok{(x) }\KeywordTok{length}\NormalTok{(x}\OperatorTok{$}\NormalTok{orientacoes_academicas}\OperatorTok{$}\NormalTok{ORIENTACAO_CONCLUIDA_POS_DOUTORADO}\OperatorTok{$}\NormalTok{ano)))}
\end{Highlighting}
\end{Shaded}

\begin{verbatim}
## [1] 740
\end{verbatim}

\begin{Shaded}
\begin{Highlighting}[]
\CommentTok{# Número de pessoas por quantitativo de orientações por pessoa 0 = 1; 1 = 2...}
\KeywordTok{table}\NormalTok{(}\KeywordTok{unlist}\NormalTok{(}\KeywordTok{sapply}\NormalTok{(unb.prof, }\ControlFlowTok{function}\NormalTok{(x) }\KeywordTok{length}\NormalTok{(x}\OperatorTok{$}\NormalTok{orientacoes_academicas}\OperatorTok{$}\NormalTok{ORIENTACAO_CONCLUIDA_MESTRADO}\OperatorTok{$}\NormalTok{ano))))}
\end{Highlighting}
\end{Shaded}

\begin{verbatim}
## 
##   0   1   2   3   4   5   6   7   8   9  10  11  12  13  14  15  16  17 
## 218 142 126 137 153 139 134 105 113  92  95  76  45  37  42  25  23   6 
##  18  19  20  21  22  23  24  25  26  27  28  30  31  34  38  41  45 
##  15  12   4   2   4   4   2   2   1   3   1   1   1   1   1   1   1
\end{verbatim}

\begin{Shaded}
\begin{Highlighting}[]
\KeywordTok{table}\NormalTok{(}\KeywordTok{unlist}\NormalTok{(}\KeywordTok{sapply}\NormalTok{(unb.prof, }\ControlFlowTok{function}\NormalTok{(x) }\KeywordTok{length}\NormalTok{(x}\OperatorTok{$}\NormalTok{orientacoes_academicas}\OperatorTok{$}\NormalTok{ORIENTACAO_CONCLUIDA_DOUTORADO}\OperatorTok{$}\NormalTok{ano))))}
\end{Highlighting}
\end{Shaded}

\begin{verbatim}
## 
##   0   1   2   3   4   5   6   7   8   9  10  11  12  13  14  16  17  19 
## 829 203 158 119 116  84  68  55  55  19  13  14  13   4   4   2   3   1 
##  20  21  22 
##   1   1   2
\end{verbatim}

\begin{Shaded}
\begin{Highlighting}[]
\KeywordTok{table}\NormalTok{(}\KeywordTok{unlist}\NormalTok{(}\KeywordTok{sapply}\NormalTok{(unb.prof, }\ControlFlowTok{function}\NormalTok{(x) }\KeywordTok{length}\NormalTok{(x}\OperatorTok{$}\NormalTok{orientacoes_academicas}\OperatorTok{$}\NormalTok{ORIENTACAO_CONCLUIDA_POS_DOUTORADO}\OperatorTok{$}\NormalTok{ano))))}
\end{Highlighting}
\end{Shaded}

\begin{verbatim}
## 
##    0    1    2    3    4    5    6    7    8    9   10   14   16   17   21 
## 1497  109   66   28   23   10   11    7    3    2    2    3    1    1    1
\end{verbatim}

\begin{Shaded}
\begin{Highlighting}[]
\CommentTok{# Número de orientações por ano}
\KeywordTok{table}\NormalTok{(}\KeywordTok{unlist}\NormalTok{(}\KeywordTok{sapply}\NormalTok{(unb.prof, }\ControlFlowTok{function}\NormalTok{(x) (x}\OperatorTok{$}\NormalTok{orientacoes_academicas}\OperatorTok{$}\NormalTok{ORIENTACAO_CONCLUIDA_MESTRADO}\OperatorTok{$}\NormalTok{ano))))}
\end{Highlighting}
\end{Shaded}

\begin{verbatim}
## 
## 2010 2011 2012 2013 2014 2015 2016 2017 
##  962 1167 1288 1541 1621 1513 1502 1281
\end{verbatim}

\begin{Shaded}
\begin{Highlighting}[]
\KeywordTok{table}\NormalTok{(}\KeywordTok{unlist}\NormalTok{(}\KeywordTok{sapply}\NormalTok{(unb.prof, }\ControlFlowTok{function}\NormalTok{(x) (x}\OperatorTok{$}\NormalTok{orientacoes_academicas}\OperatorTok{$}\NormalTok{ORIENTACAO_CONCLUIDA_DOUTORADO}\OperatorTok{$}\NormalTok{ano))))}
\end{Highlighting}
\end{Shaded}

\begin{verbatim}
## 
## 2010 2011 2012 2013 2014 2015 2016 2017 
##  304  360  433  557  550  554  617  524
\end{verbatim}

\begin{Shaded}
\begin{Highlighting}[]
\KeywordTok{table}\NormalTok{(}\KeywordTok{unlist}\NormalTok{(}\KeywordTok{sapply}\NormalTok{(unb.prof, }\ControlFlowTok{function}\NormalTok{(x) (x}\OperatorTok{$}\NormalTok{orientacoes_academicas}\OperatorTok{$}\NormalTok{ORIENTACAO_CONCLUIDA_POS_DOUTORADO}\OperatorTok{$}\NormalTok{ano))))}
\end{Highlighting}
\end{Shaded}

\begin{verbatim}
## 
## 2010 2011 2012 2013 2014 2015 2016 2017 
##   75   66   95  103  134  106   98   63
\end{verbatim}

\subsubsection{Arquivo Publicação}\label{arquivo-publicacao}

\begin{Shaded}
\begin{Highlighting}[]
\CommentTok{# Visualizar a estrutura do arquivo de Publicacao}
\CommentTok{#jsonedit(unb.pub)}
\CommentTok{#Criando um data-frame com todos os anos}
\NormalTok{unb.pub.df <-}\StringTok{ }\KeywordTok{data.frame}\NormalTok{()}
\ControlFlowTok{for}\NormalTok{ (i }\ControlFlowTok{in} \DecValTok{1}\OperatorTok{:}\KeywordTok{length}\NormalTok{(unb.pub[[}\DecValTok{1}\NormalTok{]]))}
\NormalTok{  unb.pub.df <-}\StringTok{ }\KeywordTok{rbind}\NormalTok{(unb.pub.df, unb.pub}\OperatorTok{$}\NormalTok{PERIODICO[[i]])}
\KeywordTok{glimpse}\NormalTok{(unb.pub.df)}
\end{Highlighting}
\end{Shaded}

\begin{verbatim}
## Observations: 24,456
## Variables: 10
## $ natureza           <chr> "COMPLETO", "COMPLETO", "COMPLETO", "COMPLE...
## $ titulo             <chr> "An unusual presentation of pediatric Cushi...
## $ periodico          <chr> "Journal of Pediatric Endocrinology & Metab...
## $ ano                <chr> "2010", "2010", "2010", "2010", "2010", "20...
## $ volume             <chr> "23", "5", "78", "32", "13", "7", "259", "2...
## $ issn               <chr> "0334018X", "17446651", "00099163", "180611...
## $ paginas            <chr> "607 - 612", "697 - 709", "457 - 463", "1 -...
## $ doi                <chr> "", "10.1586/eem.10.47", "10.1111/j.1399-00...
## $ autores            <list> [<"AZEVEDO, M. F.;Azevedo, M;AZEVEDO, M F;...
## $ `autores-endogeno` <list> ["0017467628165816", "0017467628165816", "...
\end{verbatim}

\begin{Shaded}
\begin{Highlighting}[]
\CommentTok{# Limpando o data-frame de listas}
\NormalTok{unb.pub.df}\OperatorTok{$}\NormalTok{autores <-}\StringTok{ }\KeywordTok{gsub}\NormalTok{(}\StringTok{"}\CharTok{\textbackslash{}"}\StringTok{,}\CharTok{\textbackslash{}"}\StringTok{|}\CharTok{\textbackslash{}"}\StringTok{, }\CharTok{\textbackslash{}"}\StringTok{"}\NormalTok{, }\StringTok{"; "}\NormalTok{, unb.pub.df}\OperatorTok{$}\NormalTok{autores)}
\NormalTok{unb.pub.df}\OperatorTok{$}\NormalTok{autores <-}\StringTok{ }\KeywordTok{gsub}\NormalTok{(}\StringTok{"}\CharTok{\textbackslash{}"}\StringTok{|c}\CharTok{\textbackslash{}\textbackslash{}}\StringTok{(|}\CharTok{\textbackslash{}\textbackslash{}}\StringTok{)"}\NormalTok{, }\StringTok{""}\NormalTok{, unb.pub.df}\OperatorTok{$}\NormalTok{autores)}
\NormalTok{unb.pub.df}\OperatorTok{$}\StringTok{`}\DataTypeTok{autores-endogeno}\StringTok{`}\NormalTok{ <-}\StringTok{ }\KeywordTok{gsub}\NormalTok{(}\StringTok{","}\NormalTok{, }\StringTok{";"}\NormalTok{, unb.pub.df}\OperatorTok{$}\StringTok{`}\DataTypeTok{autores-endogeno}\StringTok{`}\NormalTok{)}
\NormalTok{unb.pub.df}\OperatorTok{$}\StringTok{`}\DataTypeTok{autores-endogeno}\StringTok{`}\NormalTok{ <-}\StringTok{ }\KeywordTok{gsub}\NormalTok{(}\StringTok{"}\CharTok{\textbackslash{}"}\StringTok{|c}\CharTok{\textbackslash{}\textbackslash{}}\StringTok{(|}\CharTok{\textbackslash{}\textbackslash{}}\StringTok{)"}\NormalTok{, }\StringTok{""}\NormalTok{, unb.pub.df}\OperatorTok{$}\StringTok{`}\DataTypeTok{autores-endogeno}\StringTok{`}\NormalTok{)}
\KeywordTok{glimpse}\NormalTok{(unb.pub.df)}
\end{Highlighting}
\end{Shaded}

\begin{verbatim}
## Observations: 24,456
## Variables: 10
## $ natureza           <chr> "COMPLETO", "COMPLETO", "COMPLETO", "COMPLE...
## $ titulo             <chr> "An unusual presentation of pediatric Cushi...
## $ periodico          <chr> "Journal of Pediatric Endocrinology & Metab...
## $ ano                <chr> "2010", "2010", "2010", "2010", "2010", "20...
## $ volume             <chr> "23", "5", "78", "32", "13", "7", "259", "2...
## $ issn               <chr> "0334018X", "17446651", "00099163", "180611...
## $ paginas            <chr> "607 - 612", "697 - 709", "457 - 463", "1 -...
## $ doi                <chr> "", "10.1586/eem.10.47", "10.1111/j.1399-00...
## $ autores            <chr> "AZEVEDO, M. F.;Azevedo, M;AZEVEDO, M F;AZE...
## $ `autores-endogeno` <chr> "0017467628165816", "0017467628165816", "00...
\end{verbatim}

\subsubsection{Arquivo Orientação}\label{arquivo-orientacao}

\begin{Shaded}
\begin{Highlighting}[]
\CommentTok{#Orientação}
\CommentTok{#Visualizar a estrutura do json no painel Viewer}
\CommentTok{#jsonedit(unb.adv)}
\CommentTok{#Reunir todos os anos e orientações concluidas em um mesmo data-frame}
\NormalTok{unb.adv.tipo.df <-}\StringTok{ }\KeywordTok{data.frame}\NormalTok{(); unb.adv.df <-}\StringTok{ }\KeywordTok{data.frame}\NormalTok{()}
\ControlFlowTok{for}\NormalTok{ (i }\ControlFlowTok{in} \DecValTok{1}\OperatorTok{:}\KeywordTok{length}\NormalTok{(unb.adv[[}\DecValTok{1}\NormalTok{]]))}
\NormalTok{  unb.adv.tipo.df <-}\StringTok{ }\KeywordTok{rbind}\NormalTok{(unb.adv.tipo.df, unb.adv}\OperatorTok{$}\NormalTok{ORIENTACAO_CONCLUIDA_POS_DOUTORADO[[i]])}
\NormalTok{unb.adv.df <-}\StringTok{ }\KeywordTok{rbind}\NormalTok{(unb.adv.df, unb.adv.tipo.df); unb.adv.tipo.df <-}\StringTok{ }\KeywordTok{data.frame}\NormalTok{()}
\ControlFlowTok{for}\NormalTok{ (i }\ControlFlowTok{in} \DecValTok{1}\OperatorTok{:}\KeywordTok{length}\NormalTok{(unb.adv[[}\DecValTok{1}\NormalTok{]]))}
\NormalTok{  unb.adv.tipo.df <-}\StringTok{ }\KeywordTok{rbind}\NormalTok{(unb.adv.tipo.df, unb.adv}\OperatorTok{$}\NormalTok{ORIENTACAO_CONCLUIDA_DOUTORADO[[i]])}
\NormalTok{unb.adv.df <-}\StringTok{ }\KeywordTok{rbind}\NormalTok{(unb.adv.df, unb.adv.tipo.df); unb.adv.tipo.df <-}\StringTok{ }\KeywordTok{data.frame}\NormalTok{()}
\ControlFlowTok{for}\NormalTok{ (i }\ControlFlowTok{in} \DecValTok{1}\OperatorTok{:}\KeywordTok{length}\NormalTok{(unb.adv[[}\DecValTok{1}\NormalTok{]]))}
\NormalTok{  unb.adv.tipo.df <-}\StringTok{ }\KeywordTok{rbind}\NormalTok{(unb.adv.tipo.df, unb.adv}\OperatorTok{$}\NormalTok{ORIENTACAO_CONCLUIDA_MESTRADO[[i]])}
\NormalTok{unb.adv.df <-}\StringTok{ }\KeywordTok{rbind}\NormalTok{(unb.adv.df, unb.adv.tipo.df)}
\KeywordTok{glimpse}\NormalTok{(unb.adv.df)}
\end{Highlighting}
\end{Shaded}

\begin{verbatim}
## Observations: 15,109
## Variables: 13
## $ natureza                    <chr> "Supervisão de pós-doutorado", "Su...
## $ titulo                      <chr> "A Interculturalidade na sala de a...
## $ ano                         <chr> "2010", "2010", "2010", "2010", "2...
## $ id_lattes_aluno             <chr> "", "", "", "", "", "", "", "", ""...
## $ nome_aluno                  <chr> "Lucielena Mendonça de Lima", "Iov...
## $ instituicao                 <chr> "Universidad de Brasília", "Univer...
## $ curso                       <chr> "", "", "", "", "", "", "", "", ""...
## $ codigo_do_curso             <chr> "", "", "", "", "", "", "", "", ""...
## $ bolsa                       <chr> "NAO", "SIM", "SIM", "SIM", "SIM",...
## $ agencia_financiadora        <chr> "", "Fundação de Ciência e Tecnolo...
## $ codigo_agencia_financiadora <chr> "", "005100000992", "000700000992"...
## $ nome_orientadores           <list> ["Maria Luisa Ortíz Alvarez", "Ma...
## $ id_lattes_orientadores      <list> ["0562632464695581", "05626324646...
\end{verbatim}

\begin{Shaded}
\begin{Highlighting}[]
\CommentTok{#Transformar as colunas de listas em caracteres eliminando c("")}
\NormalTok{unb.adv.df}\OperatorTok{$}\NormalTok{nome_orientadores <-}\StringTok{ }\KeywordTok{gsub}\NormalTok{(}\StringTok{"}\CharTok{\textbackslash{}"}\StringTok{|c}\CharTok{\textbackslash{}\textbackslash{}}\StringTok{(|}\CharTok{\textbackslash{}\textbackslash{}}\StringTok{)"}\NormalTok{, }\StringTok{""}\NormalTok{, unb.adv.df}\OperatorTok{$}\NormalTok{nome_orientadores)}
\NormalTok{unb.adv.df}\OperatorTok{$}\NormalTok{id_lattes_orientadores <-}\StringTok{ }\KeywordTok{gsub}\NormalTok{(}\StringTok{"}\CharTok{\textbackslash{}"}\StringTok{|c}\CharTok{\textbackslash{}\textbackslash{}}\StringTok{(|}\CharTok{\textbackslash{}\textbackslash{}}\StringTok{)"}\NormalTok{, }\StringTok{""}\NormalTok{, unb.adv.df}\OperatorTok{$}\NormalTok{id_lattes_orientadores)}
\CommentTok{#Separar as colunas com dois orientadores}
\NormalTok{unb.adv.df <-}\StringTok{ }\KeywordTok{separate}\NormalTok{(unb.adv.df, nome_orientadores, }\DataTypeTok{into =} \KeywordTok{c}\NormalTok{(}\StringTok{"ori1"}\NormalTok{, }\StringTok{"ori2"}\NormalTok{), }\DataTypeTok{sep =} \StringTok{","}\NormalTok{)}
\end{Highlighting}
\end{Shaded}

\begin{verbatim}
## Warning: Expected 2 pieces. Additional pieces discarded in 6 rows [34, 35,
## 36, 2771, 5282, 5283].
\end{verbatim}

\begin{verbatim}
## Warning: Expected 2 pieces. Missing pieces filled with `NA` in 14710
## rows [1, 2, 3, 4, 5, 6, 7, 8, 9, 10, 11, 12, 13, 14, 15, 16, 17, 18, 19,
## 21, ...].
\end{verbatim}

\begin{Shaded}
\begin{Highlighting}[]
\NormalTok{unb.adv.df <-}\StringTok{ }\KeywordTok{separate}\NormalTok{(unb.adv.df, id_lattes_orientadores, }\DataTypeTok{into =} \KeywordTok{c}\NormalTok{(}\StringTok{"idLattes1"}\NormalTok{, }\StringTok{"idLattes2"}\NormalTok{), }\DataTypeTok{sep =} \StringTok{","}\NormalTok{)}
\end{Highlighting}
\end{Shaded}

\begin{verbatim}
## Warning: Expected 2 pieces. Additional pieces discarded in 6 rows [34, 35,
## 36, 2771, 5282, 5283].

## Warning: Expected 2 pieces. Missing pieces filled with `NA` in 14710
## rows [1, 2, 3, 4, 5, 6, 7, 8, 9, 10, 11, 12, 13, 14, 15, 16, 17, 18, 19,
## 21, ...].
\end{verbatim}

\begin{Shaded}
\begin{Highlighting}[]
\CommentTok{#Numero de orientacoes por ano}
\KeywordTok{table}\NormalTok{(unb.adv.df}\OperatorTok{$}\NormalTok{ano)}
\end{Highlighting}
\end{Shaded}

\begin{verbatim}
## 
## 2010 2011 2012 2013 2014 2015 2016 2017 
## 1301 1556 1749 2150 2237 2117 2166 1833
\end{verbatim}

\begin{Shaded}
\begin{Highlighting}[]
\CommentTok{#Tabela com nome de professor e numero de orientacoes}
\KeywordTok{head}\NormalTok{(}\KeywordTok{sort}\NormalTok{(}\KeywordTok{table}\NormalTok{(}\KeywordTok{rbind}\NormalTok{(unb.adv.df}\OperatorTok{$}\NormalTok{ori1, unb.adv.df}\OperatorTok{$}\NormalTok{ori2)), }\DataTypeTok{decreasing =} \OtherTok{TRUE}\NormalTok{), }\DecValTok{20}\NormalTok{)}
\end{Highlighting}
\end{Shaded}

\begin{verbatim}
## 
##                        Octavio Luiz Franco 
##                                         76 
##              Célia Maria de Almeida Soares 
##                                         61 
##                  Maria Fatima Grossi de Sa 
##                                         59 
##                     Jorge Madeira Nogueira 
##                                         53 
##         Concepta Margaret McManus Pimentel 
##                                         51 
##                    Juliana de Fátima Sales 
##                                         44 
##                Ana Maria Resende Junqueira 
##                                         40 
##                               Debora Diniz 
##                                         39 
##            Ana Suelly Arruda Câmara Cabral 
##                                         36 
##                          Gabriele Cornelli 
##                                         36 
##                         Helena Eri Shimizu 
##                                         36 
##                         Nivaldo dos Santos 
##                                         36 
##                         Lucio França Teles 
##                                         35 
##                      Edson Silva de Farias 
##                                         34 
##                      Ileno Izídio da Costa 
##                                         34 
##                  Ricardo Bentes de Azevedo 
##                                         34 
## Stella Maris Bortoni de Figueiredo Ricardo 
##                                         34 
##                  Anderson de Rezende Rocha 
##                                         33 
##                   Aparecido Divino da Cruz 
##                                         33 
##                     André Pacheco de Assis 
##                                         32
\end{verbatim}

\subsection{CRISP-DM Fase.Atividade 2.4 - Verificação da qualidade dos
dados.}\label{crisp-dm-fase.atividade-2.4---verificacao-da-qualidade-dos-dados.}

Como já informado, a verificação da qualidade dos dados envolve
responder se os dados disponíveis estão realmente completos.

As informações disponíveis são suficientes para o trabalho proposto?

Neste projeto, a verificação da qualidade dos dados é responsabilidade
dos alunos.

\section{\texorpdfstring{CRISP-DM Fase 3 - \textbf{Preparação dos
Dados}}{CRISP-DM Fase 3 - Preparação dos Dados}}\label{crisp-dm-fase-3---preparacao-dos-dados}

Como já informado, na fase de \textbf{Preparação dos Dados} os
\emph{datasets} que serão utilizados em todo o trabalho são construídos
a partir dos dados brutos. Aqui os dados são ``filtrados'' retirando-se
partes que não interessam e selecionando-se os ``campos'' necessários
para o trabalho de mineração.

São 5 as atividades genéricas nesta fase de preparação dos dados, a
seguir divididas em subseções

\subsection{CRISP-DM Fase.Atividade 3.1 - Seleção dos
dados.}\label{crisp-dm-fase.atividade-3.1---selecao-dos-dados.}

Como já informado, a seleção dos dados envolve identificar quais dados,
da nossa ``montanha de dados'', serão realmente utilizados.

Quais variáveis dos dados brutos serão convertidas para o
\emph{dataset}?

Não é raro cometer o erro de selecionar dados para um modelo preditivo
com base em uma falsa ideia de que aqueles dados contém a resposta para
o modelo que se quer construir. Surge o cuidado de se separar o sinal do
ruído (Silver, Nate. The Signal and the Noise: Why so many predictions
fail --- but some don't. USA: The Penguin Press HC, 2012.).

\subsection{CRISP-DM Fase.Atividade 3.2 - Limpeza dos
dados}\label{crisp-dm-fase.atividade-3.2---limpeza-dos-dados}

\subsection{CRISP-DM Fase.Atividade 3.3 - Construção dos
dados}\label{crisp-dm-fase.atividade-3.3---construcao-dos-dados}

Como já informado, a construção dos dados envolve a criação de novas
variáveis a partir de outras presentes nos \emph{datasets}.

\begin{Shaded}
\begin{Highlighting}[]
\CommentTok{# Funcoes }

\CommentTok{# converte as colunas de um dataframe tipo lista em tipo character}
\NormalTok{cv_tplista2tpchar <-}\StringTok{ }\ControlFlowTok{function}\NormalTok{( df  ) \{ }
  \ControlFlowTok{for}\NormalTok{( variavel }\ControlFlowTok{in} \KeywordTok{names}\NormalTok{(df)) \{}
    \ControlFlowTok{if}\NormalTok{ (}\KeywordTok{class}\NormalTok{(df[[variavel]]) }\OperatorTok{==}\StringTok{ "list"}\NormalTok{ ) \{}
\NormalTok{      df[[variavel]] <-}\StringTok{ }\KeywordTok{lapply}\NormalTok{(df[[variavel]] ,   }\ControlFlowTok{function}\NormalTok{(x)   }\KeywordTok{lista2texto}\NormalTok{( x  ) ) }
\NormalTok{      df[[variavel]] <-}\StringTok{ }\KeywordTok{as.character}\NormalTok{( df[[variavel]] )}
\NormalTok{    \}}
\NormalTok{  \}}
  \KeywordTok{return}\NormalTok{(df)}
\NormalTok{\}}
\NormalTok{###}


\CommentTok{# converte o conteudo de lista em array de characters}
\NormalTok{lista2texto <-}\StringTok{ }\ControlFlowTok{function}\NormalTok{( lista  ) \{}
  \ControlFlowTok{if}\NormalTok{(}\KeywordTok{is.null}\NormalTok{(lista)) \{}
    \KeywordTok{return}\NormalTok{ ( }\OtherTok{NULL}\NormalTok{ )}
\NormalTok{  \}}
\NormalTok{  saida <-}\StringTok{ ""}
  \ControlFlowTok{for}\NormalTok{( j }\ControlFlowTok{in} \DecValTok{1}\OperatorTok{:}\KeywordTok{length}\NormalTok{(lista)) \{ }
    \ControlFlowTok{for}\NormalTok{( i }\ControlFlowTok{in} \DecValTok{1}\OperatorTok{:}\KeywordTok{length}\NormalTok{(lista[[j]]) ) \{}
\NormalTok{      elemento <-}\StringTok{ }\NormalTok{lista[[j]][i] }
      \ControlFlowTok{if}\NormalTok{( }\OperatorTok{!}\KeywordTok{is.null}\NormalTok{(elemento)) \{ }
        \ControlFlowTok{if}\NormalTok{( i }\OperatorTok{==}\StringTok{ }\KeywordTok{length}\NormalTok{(lista[[j]]) }\OperatorTok{&}\StringTok{ }\NormalTok{j }\OperatorTok{==}\StringTok{ }\KeywordTok{length}\NormalTok{(lista)  ) \{ }
          \CommentTok{# se for o ultimo elemento nao coloque o ponto e virgula no final            }
\NormalTok{          saida <-}\StringTok{ }\KeywordTok{paste0}\NormalTok{( saida , elemento  )}
\NormalTok{        \} }\ControlFlowTok{else}\NormalTok{ \{}
          \CommentTok{# enquanto nao for o ultimo coloque ; separando os elementos concatenados }
\NormalTok{          saida <-}\StringTok{ }\KeywordTok{paste0}\NormalTok{( saida , elemento , }\DataTypeTok{sep =} \StringTok{" ; "}\NormalTok{)}
\NormalTok{        \}}
\NormalTok{      \}  }
\NormalTok{    \}}
\NormalTok{  \}}
  \KeywordTok{return}\NormalTok{( saida )}
\NormalTok{\}}

\CommentTok{# Converte producao elattes separada por anos em um unico dataframe }
\NormalTok{converte_producao2dataframe<-}\StringTok{ }\ControlFlowTok{function}\NormalTok{( lista_producao ) \{}
\NormalTok{  df_saida <-}\StringTok{ }\OtherTok{NULL} 
  
  \ControlFlowTok{for}\NormalTok{( ano }\ControlFlowTok{in} \KeywordTok{names}\NormalTok{(lista_producao)) \{}
\NormalTok{    df_saida <-}\StringTok{ }\KeywordTok{rbind}\NormalTok{(df_saida , lista_producao[[ano]])}
\NormalTok{  \}}
  
  \CommentTok{# converte tipo lista em array de character }
\NormalTok{  df_saida <-}\StringTok{ }\KeywordTok{cv_tplista2tpchar}\NormalTok{(df_saida)}
  \KeywordTok{return}\NormalTok{(df_saida)}
  

\NormalTok{\}}

\CommentTok{#concatena dois dataframes com  colunas diferentes }
\NormalTok{concatenadf <-}\StringTok{ }\ControlFlowTok{function}\NormalTok{( df1, df2) \{ }
  \CommentTok{#cria colunas de df1 que faltam em df2}
  \ControlFlowTok{for}\NormalTok{( coluna }\ControlFlowTok{in} \KeywordTok{names}\NormalTok{(df1 ) ) \{}
    \ControlFlowTok{if}\NormalTok{( }\OperatorTok{!}\KeywordTok{is.element}\NormalTok{(coluna, }\KeywordTok{names}\NormalTok{(df2) )) \{}
\NormalTok{      df2[coluna] <-}\StringTok{ }\OtherTok{NA}
\NormalTok{    \}}
\NormalTok{  \}}
  
  \CommentTok{#cria colunas de df2 que faltam em df1  }
  \ControlFlowTok{for}\NormalTok{( coluna }\ControlFlowTok{in} \KeywordTok{names}\NormalTok{(df2 ) ) \{}
    
    \ControlFlowTok{if}\NormalTok{( }\OperatorTok{!}\KeywordTok{is.element}\NormalTok{(coluna, }\KeywordTok{names}\NormalTok{(df1) )) \{}
\NormalTok{      df1[coluna] <-}\StringTok{ }\OtherTok{NA}
\NormalTok{    \}}
\NormalTok{  \}}
  
  
  \CommentTok{#faz o rbind dos dois dataframes }
\NormalTok{  df_final <-}\StringTok{ }\KeywordTok{rbind}\NormalTok{(df1 , df2)}
  \KeywordTok{return}\NormalTok{(df_final)}
  
\NormalTok{\}}

\CommentTok{# Extracao dos perfis dos professores }

\NormalTok{extrai_1perfil <-}\StringTok{ }\ControlFlowTok{function}\NormalTok{( professor ) \{}
\NormalTok{  idLattes <-}\StringTok{ }\KeywordTok{names}\NormalTok{(professor)}
\NormalTok{  nome <-}\StringTok{ }\NormalTok{professor[[idLattes]]}\OperatorTok{$}\NormalTok{nome   }
\NormalTok{  resumo_cv <-}\StringTok{ }\NormalTok{professor[[idLattes]]}\OperatorTok{$}\NormalTok{resumo_cv }
\NormalTok{  endereco_profissional <-}\StringTok{ }\NormalTok{professor[[idLattes]]}\OperatorTok{$}\NormalTok{endereco_profissional }\CommentTok{#list }
\NormalTok{  instituicao <-}\StringTok{ }\NormalTok{endereco_profissional}\OperatorTok{$}\NormalTok{instituicao}
\NormalTok{  orgao <-}\StringTok{ }\NormalTok{endereco_profissional}\OperatorTok{$}\NormalTok{orgao}
\NormalTok{  unidade <-}\StringTok{ }\NormalTok{endereco_profissional}\OperatorTok{$}\NormalTok{unidade}
\NormalTok{  DDD <-}\StringTok{ }\NormalTok{endereco_profissional}\OperatorTok{$}\NormalTok{DDD}
\NormalTok{  telefone <-}\StringTok{ }\NormalTok{endereco_profissional}\OperatorTok{$}\NormalTok{telefone}
\NormalTok{  bairro <-}\StringTok{ }\NormalTok{endereco_profissional}\OperatorTok{$}\NormalTok{bairro}
\NormalTok{  cep <-}\StringTok{ }\NormalTok{endereco_profissional}\OperatorTok{$}\NormalTok{cep}
\NormalTok{  cidade <-}\StringTok{ }\NormalTok{endereco_profissional}\OperatorTok{$}\NormalTok{cidade}
\NormalTok{  senioridade <-}\StringTok{ }\NormalTok{professor[[idLattes]]}\OperatorTok{$}\NormalTok{senioridade  }
\NormalTok{  df_1perfil <-}\StringTok{ }\KeywordTok{data.frame}\NormalTok{( idLattes , nome, resumo_cv ,instituicao , }
\NormalTok{                           orgao, unidade , DDD, telefone, bairro,cep,cidade , senioridade,}
                           \DataTypeTok{stringsAsFactors =} \OtherTok{FALSE}\NormalTok{)}
  
  \KeywordTok{return}\NormalTok{(df_1perfil)  }
\NormalTok{\}}

\NormalTok{extrai_perfis <-}\StringTok{ }\ControlFlowTok{function}\NormalTok{(jsonProfessores) \{}
\NormalTok{  df_saida <-}\StringTok{ }\KeywordTok{data.frame}\NormalTok{()}
  \ControlFlowTok{for}\NormalTok{( i }\ControlFlowTok{in} \DecValTok{1}\OperatorTok{:}\KeywordTok{length}\NormalTok{(jsonProfessores)) \{}
\NormalTok{    jsonProfessor <-}\StringTok{ }\NormalTok{jsonProfessores[i]}
\NormalTok{    df_professor <-}\StringTok{ }\KeywordTok{extrai_1perfil}\NormalTok{(jsonProfessor)}
    \ControlFlowTok{if}\NormalTok{( }\KeywordTok{nrow}\NormalTok{(df_saida) }\OperatorTok{>}\StringTok{ }\DecValTok{0}\NormalTok{ ) \{}
\NormalTok{      df_saida <-}\StringTok{ }\KeywordTok{rbind}\NormalTok{(df_saida , df_professor)}
\NormalTok{    \} }\ControlFlowTok{else}\NormalTok{ \{}
\NormalTok{      df_saida <-}\StringTok{ }\NormalTok{df_professor }
\NormalTok{    \}}
\NormalTok{  \}}
   
  \KeywordTok{return}\NormalTok{(df_saida)}
\NormalTok{\}}

\CommentTok{# Extracao da producao bibliografica dos professores }

\NormalTok{extrai_1producao <-}\StringTok{ }\ControlFlowTok{function}\NormalTok{(professor) \{}
\NormalTok{  idLattes <-}\StringTok{ }\KeywordTok{names}\NormalTok{(professor)}
\NormalTok{  df_1producao <<-}\StringTok{ }\OtherTok{NULL} 
\NormalTok{  producao_bibliografica <-}\StringTok{ }\NormalTok{professor[[idLattes]]}\OperatorTok{$}\NormalTok{producao_bibiografica  }\CommentTok{#list}
  \ControlFlowTok{for}\NormalTok{( tipo_producao }\ControlFlowTok{in} \KeywordTok{names}\NormalTok{(producao_bibliografica)) \{ }
\NormalTok{    df_temporario <-}\StringTok{ }\KeywordTok{cv_tplista2tpchar}\NormalTok{ ( producao_bibliografica[[tipo_producao]]) }
\NormalTok{    df_temporario}\OperatorTok{$}\NormalTok{tipo_producao <-}\StringTok{  }\NormalTok{tipo_producao }
\NormalTok{    df_temporario}\OperatorTok{$}\NormalTok{idLattes <-}\StringTok{  }\NormalTok{idLattes}
\NormalTok{    df_1producao <-}\StringTok{ }\KeywordTok{concatenadf}\NormalTok{( df_1producao , df_temporario  )}
\NormalTok{  \}  }
  \KeywordTok{return}\NormalTok{(df_1producao)}
\NormalTok{\}}

\NormalTok{extrai_producoes <-}\StringTok{ }\ControlFlowTok{function}\NormalTok{( jsonProfessores) \{}
\NormalTok{  df_saida <-}\StringTok{ }\KeywordTok{data.frame}\NormalTok{()}
  \ControlFlowTok{for}\NormalTok{( i }\ControlFlowTok{in} \DecValTok{1}\OperatorTok{:}\KeywordTok{length}\NormalTok{(jsonProfessores)) \{}
\NormalTok{    jsonProfessor <-}\StringTok{ }\NormalTok{jsonProfessores[i]}
\NormalTok{    df_producao <-}\StringTok{ }\KeywordTok{extrai_1producao}\NormalTok{(jsonProfessor)}
    \ControlFlowTok{if}\NormalTok{( }\KeywordTok{nrow}\NormalTok{(df_saida) }\OperatorTok{>}\StringTok{ }\DecValTok{0}\NormalTok{ ) \{}
\NormalTok{      df_saida <-}\StringTok{ }\KeywordTok{concatenadf}\NormalTok{(df_saida , df_producao)}
\NormalTok{    \} }\ControlFlowTok{else}\NormalTok{ \{}
\NormalTok{      df_saida <-}\StringTok{ }\NormalTok{df_producao }
\NormalTok{    \}}
\NormalTok{  \}}
\NormalTok{  df_saida <-}\StringTok{ }\NormalTok{df_saida }\OperatorTok\StringTok{ }\KeywordTok{filter}\NormalTok{( }\OperatorTok{!}\KeywordTok{is.na}\NormalTok{(tipo_producao))}
  \KeywordTok{return}\NormalTok{(df_saida)  }
\NormalTok{\}}

\CommentTok{# Extracao das orientacoes dos professores }

\NormalTok{extrai_1orientacao <-}\StringTok{ }\ControlFlowTok{function}\NormalTok{(professor) \{}
\NormalTok{  idLattes <-}\StringTok{ }\KeywordTok{names}\NormalTok{(professor)}
\NormalTok{  df_1orientacao <-}\StringTok{ }\OtherTok{NULL}
\NormalTok{  orientacoes_academicas  <-}\StringTok{ }\NormalTok{professor[[idLattes]]}\OperatorTok{$}\NormalTok{orientacoes_academicas  }\CommentTok{#list}
  \ControlFlowTok{for}\NormalTok{( orientacao }\ControlFlowTok{in} \KeywordTok{names}\NormalTok{(orientacoes_academicas )) \{ }
\NormalTok{    df_temporario <-}\StringTok{ }\KeywordTok{cv_tplista2tpchar}\NormalTok{ ( orientacoes_academicas[[orientacao]])}
\NormalTok{    df_temporario}\OperatorTok{$}\NormalTok{orientacao <-}\StringTok{  }\NormalTok{orientacao }
\NormalTok{    df_temporario}\OperatorTok{$}\NormalTok{idLattes <-}\StringTok{  }\NormalTok{idLattes}
\NormalTok{    df_1orientacao <-}\StringTok{ }\KeywordTok{concatenadf}\NormalTok{( df_1orientacao , df_temporario  )}
\NormalTok{  \}  }
  \KeywordTok{return}\NormalTok{(df_1orientacao) }
\NormalTok{\}}

\NormalTok{extrai_orientacoes <-}\StringTok{ }\ControlFlowTok{function}\NormalTok{(jsonProfessores) \{}
\NormalTok{  df_saida <-}\StringTok{ }\KeywordTok{data.frame}\NormalTok{()}
  \ControlFlowTok{for}\NormalTok{( i }\ControlFlowTok{in} \DecValTok{1}\OperatorTok{:}\KeywordTok{length}\NormalTok{(jsonProfessores)) \{}
\NormalTok{    jsonProfessor <-}\StringTok{ }\NormalTok{jsonProfessores[i]}
\NormalTok{    df_orientacao <-}\StringTok{ }\KeywordTok{extrai_1orientacao}\NormalTok{(jsonProfessor)}
    \ControlFlowTok{if}\NormalTok{( }\KeywordTok{nrow}\NormalTok{(df_saida) }\OperatorTok{>}\StringTok{ }\DecValTok{0}\NormalTok{ ) \{}
\NormalTok{      df_saida <-}\StringTok{ }\KeywordTok{concatenadf}\NormalTok{(df_saida , df_orientacao)}
\NormalTok{    \} }\ControlFlowTok{else}\NormalTok{ \{}
\NormalTok{      df_saida <-}\StringTok{ }\NormalTok{df_orientacao}
\NormalTok{    \}}
\NormalTok{  \}}
\NormalTok{  df_saida <-}\StringTok{ }\NormalTok{df_saida }\OperatorTok\StringTok{ }\KeywordTok{filter}\NormalTok{(}\OperatorTok{!}\KeywordTok{is.na}\NormalTok{(idLattes))}
  \KeywordTok{return}\NormalTok{(df_saida)  }
\NormalTok{\}}

\CommentTok{# Extracao das areas de atuacao dos professores }

\NormalTok{extrai_1area_de_atuacao <-}\StringTok{ }\ControlFlowTok{function}\NormalTok{(professor)\{}
\NormalTok{  idLattes <-}\StringTok{ }\KeywordTok{names}\NormalTok{(professor)}
\NormalTok{  df_1area <-}\StringTok{  }\NormalTok{professor[[idLattes]]}\OperatorTok{$}\NormalTok{areas_de_atuacao}
\NormalTok{  df_1area}\OperatorTok{$}\NormalTok{idLattes <-}\StringTok{ }\NormalTok{idLattes}
  \KeywordTok{return}\NormalTok{(df_1area)}
\NormalTok{\}}

\NormalTok{extrai_areas_atuacao <-}\StringTok{ }\ControlFlowTok{function}\NormalTok{(jsonProfessores)\{}
\NormalTok{  df_saida <-}\StringTok{ }\KeywordTok{data.frame}\NormalTok{()}
  \ControlFlowTok{for}\NormalTok{( i }\ControlFlowTok{in} \DecValTok{1}\OperatorTok{:}\KeywordTok{length}\NormalTok{(jsonProfessores)) \{}
\NormalTok{    jsonProfessor <-}\StringTok{ }\NormalTok{jsonProfessores[i]}
\NormalTok{    df_area_atuacao <-}\StringTok{ }\KeywordTok{extrai_1area_de_atuacao}\NormalTok{(jsonProfessor)}
    \ControlFlowTok{if}\NormalTok{( }\KeywordTok{nrow}\NormalTok{(df_saida) }\OperatorTok{>}\StringTok{ }\DecValTok{0}\NormalTok{ ) \{}
\NormalTok{      df_saida <-}\StringTok{ }\KeywordTok{concatenadf}\NormalTok{(df_saida , df_area_atuacao)}
\NormalTok{    \} }\ControlFlowTok{else}\NormalTok{ \{}
\NormalTok{      df_saida <-}\StringTok{ }\NormalTok{df_area_atuacao}
\NormalTok{    \}}
\NormalTok{  \}}
\NormalTok{  df_saida <-}\StringTok{ }\NormalTok{df_saida }\OperatorTok\StringTok{ }\KeywordTok{filter}\NormalTok{( }\OperatorTok{!}\KeywordTok{is.na}\NormalTok{(idLattes))}
  \KeywordTok{return}\NormalTok{(df_saida)   }
\NormalTok{\}}
\NormalTok{########################### Inicio }

\CommentTok{# colocar o diretorio onde está o arquivo json de perfis a serem lidos }
\NormalTok{unb.prof.json <-}\StringTok{ }\KeywordTok{read_file}\NormalTok{(}\StringTok{"./unbpos/unbpos.profile.json"}\NormalTok{)}
\NormalTok{unb.prof.df.capes <-}\StringTok{ }\KeywordTok{read.csv}\NormalTok{(}\StringTok{"./unbpos/PesqPosCapes.csv"}\NormalTok{, }
                              \DataTypeTok{sep =} \StringTok{";"}\NormalTok{, }\DataTypeTok{header =} \OtherTok{TRUE}\NormalTok{, }\DataTypeTok{colClasses =} \StringTok{"character"}\NormalTok{)}
\NormalTok{unb.prof <-}\StringTok{ }\KeywordTok{fromJSON}\NormalTok{(unb.prof.json)}
\KeywordTok{length}\NormalTok{(unb.prof)}
\end{Highlighting}
\end{Shaded}

\begin{verbatim}
## [1] 1764
\end{verbatim}

\begin{Shaded}
\begin{Highlighting}[]
\CommentTok{# extrai perfis dos professores }
\NormalTok{unb.prof.df.professores <-}\StringTok{ }\KeywordTok{extrai_perfis}\NormalTok{(unb.prof)}

\CommentTok{# extrai producao bibliografica de todos os professores }
\NormalTok{unb.prof.df.publicacoes <-}\StringTok{ }\KeywordTok{extrai_producoes}\NormalTok{(unb.prof)}

\CommentTok{#extrai orientacoes }
\NormalTok{unb.prof.df.orientacoes <-}\StringTok{ }\KeywordTok{extrai_orientacoes}\NormalTok{(unb.prof)}

\CommentTok{#extrai areas de atuacao }
\NormalTok{unb.prof.df.areas.de.atuacao <-}\StringTok{ }\KeywordTok{extrai_areas_atuacao}\NormalTok{(unb.prof)}

\CommentTok{#salva os daframes }
\KeywordTok{save}\NormalTok{(unb.prof.df.professores, unb.prof.df.publicacoes,}
\NormalTok{     unb.prof.df.orientacoes, unb.prof.df.areas.de.atuacao, }\DataTypeTok{file =} \StringTok{"dataframes.Rda"}\NormalTok{)}

\CommentTok{#cria arquivo para análise}
\NormalTok{unb.prof.df <-}\StringTok{ }\KeywordTok{data.frame}\NormalTok{()}
\NormalTok{unb.prof.df <-}\StringTok{ }\NormalTok{unb.prof.df.professores }\OperatorTok\StringTok{ }
\StringTok{  }\KeywordTok{select}\NormalTok{(idLattes, nome, resumo_cv, senioridade) }\OperatorTok\StringTok{ }
\StringTok{  }\KeywordTok{left_join}\NormalTok{(}
\NormalTok{    unb.prof.df.orientacoes }\OperatorTok\StringTok{ }
\StringTok{      }\KeywordTok{select}\NormalTok{(orientacao, idLattes) }\OperatorTok\StringTok{ }
\StringTok{      }\KeywordTok{filter}\NormalTok{(}\OperatorTok{!}\KeywordTok{grepl}\NormalTok{(}\StringTok{"EM_ANDAMENTO"}\NormalTok{, orientacao)) }\OperatorTok\StringTok{ }
\StringTok{      }\KeywordTok{group_by}\NormalTok{(idLattes) }\OperatorTok\StringTok{ }
\StringTok{      }\KeywordTok{count}\NormalTok{(orientacao) }\OperatorTok\StringTok{ }
\StringTok{      }\KeywordTok{spread}\NormalTok{(}\DataTypeTok{key =}\NormalTok{ orientacao, }\DataTypeTok{value =}\NormalTok{ n), }
    \DataTypeTok{by =} \StringTok{"idLattes"}\NormalTok{) }\OperatorTok\StringTok{ }
\StringTok{  }\KeywordTok{left_join}\NormalTok{(}
\NormalTok{    unb.prof.df.publicacoes }\OperatorTok\StringTok{ }
\StringTok{      }\KeywordTok{select}\NormalTok{(tipo_producao, idLattes) }\OperatorTok\StringTok{ }
\StringTok{      }\KeywordTok{filter}\NormalTok{(}\OperatorTok{!}\KeywordTok{grepl}\NormalTok{(}\StringTok{"ARTIGO_ACEITO"}\NormalTok{, tipo_producao)) }\OperatorTok\StringTok{ }
\StringTok{      }\KeywordTok{group_by}\NormalTok{(idLattes) }\OperatorTok\StringTok{ }
\StringTok{      }\KeywordTok{count}\NormalTok{(tipo_producao) }\OperatorTok\StringTok{ }
\StringTok{      }\KeywordTok{spread}\NormalTok{(}\DataTypeTok{key =}\NormalTok{ tipo_producao, }\DataTypeTok{value =}\NormalTok{ n), }
    \DataTypeTok{by =} \StringTok{"idLattes"}\NormalTok{) }\OperatorTok\StringTok{ }
\StringTok{  }\KeywordTok{left_join}\NormalTok{(}
\NormalTok{    unb.prof.df.areas.de.atuacao }\OperatorTok\StringTok{ }
\StringTok{      }\KeywordTok{select}\NormalTok{(area, idLattes) }\OperatorTok\StringTok{ }
\StringTok{      }\KeywordTok{group_by}\NormalTok{(idLattes) }\OperatorTok\StringTok{ }
\StringTok{      }\KeywordTok{summarise}\NormalTok{(}\KeywordTok{n_distinct}\NormalTok{(area)), }
    \DataTypeTok{by =} \StringTok{"idLattes"}\NormalTok{) }\OperatorTok\StringTok{ }
\StringTok{  }\KeywordTok{left_join}\NormalTok{(}
\NormalTok{    unb.prof.df.capes }\OperatorTok\StringTok{ }
\StringTok{      }\KeywordTok{select}\NormalTok{(AreaPos, idLattes) }\OperatorTok\StringTok{ }
\StringTok{      }\KeywordTok{group_by}\NormalTok{(idLattes) }\OperatorTok\StringTok{ }
\StringTok{      }\KeywordTok{summarise}\NormalTok{(}\KeywordTok{n_distinct}\NormalTok{(AreaPos)), }
    \DataTypeTok{by =} \StringTok{"idLattes"}\NormalTok{)}

\KeywordTok{glimpse}\NormalTok{(unb.prof.df)}
\end{Highlighting}
\end{Shaded}

\begin{verbatim}
## Observations: 1,764
## Variables: 16
## $ idLattes                               <chr> "0000507838194708", "00...
## $ nome                                   <chr> "Norai Romeu Rocco", "A...
## $ resumo_cv                              <chr> "Possui graduação em Ma...
## $ senioridade                            <chr> "8", "9", "7", "8", "9"...
## $ ORIENTACAO_CONCLUIDA_DOUTORADO         <int> 3, 3, NA, NA, NA, NA, N...
## $ ORIENTACAO_CONCLUIDA_MESTRADO          <int> 3, 14, 1, 5, 2, 3, NA, ...
## $ ORIENTACAO_CONCLUIDA_POS_DOUTORADO     <int> NA, NA, NA, NA, NA, NA,...
## $ OUTRAS_ORIENTACOES_CONCLUIDAS          <int> NA, 6, 11, 7, 5, 14, 10...
## $ CAPITULO_DE_LIVRO                      <int> NA, 3, 1, 5, 1, NA, 3, ...
## $ DEMAIS_TIPOS_DE_PRODUCAO_BIBLIOGRAFICA <int> 7, 10, NA, NA, NA, NA, ...
## $ EVENTO                                 <int> 1, 8, 25, 17, 9, 1, 26,...
## $ LIVRO                                  <int> NA, 2, NA, 2, NA, NA, 1...
## $ PERIODICO                              <int> 6, 27, 3, 6, 27, 2, 14,...
## $ TEXTO_EM_JORNAIS                       <int> NA, NA, NA, 1, NA, NA, ...
## $ `n_distinct(area)`                     <int> 2, 1, 2, 2, 1, 1, 1, 4,...
## $ `n_distinct(AreaPos)`                  <int> 1, 1, 1, 2, 1, 1, 1, 2,...
\end{verbatim}

\subsection{CRISP-DM Fase.Atividade 3.4 - Integração dos
dados}\label{crisp-dm-fase.atividade-3.4---integracao-dos-dados}

Como já informado, a integração dos dados envolve a união (merge) de
diferentes tabelas para criar um único \emph{dataset} para ser utilizado
no R, por exemplo.

\subsection{CRISP-DM Fase.Atividade 3.5 - Formatação dos
dados}\label{crisp-dm-fase.atividade-3.5---formatacao-dos-dados}

Como já informado, a formatação de dados envolve a realização de
pequenas alterações na estrutura dos dados, como a ordem das variáveis,
para permitir a execução de determinado método de data mining.

\section{\texorpdfstring{CRISP-DM Fase 4 -
\textbf{Modelagem}}{CRISP-DM Fase 4 - Modelagem}}\label{crisp-dm-fase-4---modelagem}

Como já informado, na fase de \textbf{Modelagem} no CRISP-DM ocorre a
construção e avaliação de modelos estatísticos ou computacionais,
podendo ser realizada em quatro atividades genéricas, a seguir
organizadas na forma de seções

\subsection{CRISP-DM Fase.Atividade 4.1 - Seleção das técnicas de
modelagem}\label{crisp-dm-fase.atividade-4.1---selecao-das-tecnicas-de-modelagem}

\subsection{CRISP-DM Fase.Atividade 4.2 - Realização de testes de
modelagem}\label{crisp-dm-fase.atividade-4.2---realizacao-de-testes-de-modelagem}

Como já informado, na realização de testes de modelagem diferentes
modelos estatísticos ou computacionais são previamente testados e
avaliados. Pode-se dividir o \emph{dataset} criado na etapa anterior
para se ter uma base de treino na construção de modelos, e outra pequena
parte para validar e avaliar a eficiência de cada modelo criado até se
chegar ao mais ``eficiente''.

\subsection{CRISP-DM Fase.Atividade 4.3 - Construção do modelo
definitivo}\label{crisp-dm-fase.atividade-4.3---construcao-do-modelo-definitivo}

Como já informado, a construçao do modelo definitivo é realizada com
base na melhor experiência do passo anterior.

\subsection{CRISP-DM Fase.Atividade 4.4 - Avaliação do
modelo}\label{crisp-dm-fase.atividade-4.4---avaliacao-do-modelo}

\section{\texorpdfstring{CRISP-DM Fase 5 -
\textbf{Avaliação}}{CRISP-DM Fase 5 - Avaliação}}\label{crisp-dm-fase-5---avaliacao}

Como já informado, na fase de \textbf{Avaliação} do CRISP-DM os
resultados não são apenas avaliados, mas se verifica se existem questões
relacionadas à organização que não foram suficientemente abordadas.
Deve-se refletir se o uso arepetido do modelo criado pode trazer algum
``efeito colateral'' para a organização.

Como já informado, nesta fase, pode-se trabalhar com 3 atividades
genéricas, a seguir distribuídas em seções.

\subsection{CRISP-DM Fase.Atividade 5.1 - Avaliação dos
resultados}\label{crisp-dm-fase.atividade-5.1---avaliacao-dos-resultados}

\subsection{CRISP-DM Fase.Atividade 5.2 - Revisão do
processo}\label{crisp-dm-fase.atividade-5.2---revisao-do-processo}

Como já informado, durante a revisão do processo verifica-se se o modelo
foi construído adequadamente. As variáveis (passadas) para construir o
modelo estarão disponíveis no futuro?

\subsection{CRISP-DM Fase.Atividade 5.3 - Determinação dos etapas
seguintes}\label{crisp-dm-fase.atividade-5.3---determinacao-dos-etapas-seguintes}

Como já informado, pode ser necessário decidir-se por finalizar o
projeto, passar à etapa de desenvolvimento, ou rever algumas fases
anteriores para a melhoria do projeto.

\section{\texorpdfstring{CRISP-DM Fase 6 - \textbf{Implantação}
(\emph{deployment})}{CRISP-DM Fase 6 - Implantação (deployment)}}\label{crisp-dm-fase-6---implantacao-deployment}

Como já informado, na fase de \textbf{Implantação} (\emph{deployment})
se realiza o planejamento de implantação dos produtos desenvolvidos
(scripts, no caso do executado nesta disciplina) para o ambiente
operacional, para seu uso repetitivo, envolvendo atividades de
monitoramento e manutenção do sistema (script) desenvolvido. A fase de
implantação concluir com a produção e apresentação do relatório final
com os resultados do projeto.

Como já informado, são seis as atividades genéricas na fase de
\textbf{implantação}, a seguir apresentadas na forma de seções.

\subsection{CRISP-DM Fase.Atividade 6.1 - Planejamento da
transição}\label{crisp-dm-fase.atividade-6.1---planejamento-da-transicao}

De que forma os produtos desenvolvidos pelo grupo poderiam ser colocados
em uso prático regular, na organização cliente?

\subsection{CRISP-DM Fase.Atividade 6.2 - Planejamento do monitoramento
dos
produtos}\label{crisp-dm-fase.atividade-6.2---planejamento-do-monitoramento-dos-produtos}

De que forma seria possível realizar o monitoramento do funcionamento
dos produtos em utilização no ambiente operacional?

\subsection{CRISP-DM Fase.Atividade 6.3 - Planejamento de
manuteção}\label{crisp-dm-fase.atividade-6.3---planejamento-de-manutecao}

que manutenções, ajustes, mudanças, poderia ter que ser eventualmente
realizadas no produto (scripts), quando em uso no ambiente operacional
do cliente?

\subsection{CRISP-DM Fase.Atividade 6.4 - Produção do relatório
final}\label{crisp-dm-fase.atividade-6.4---producao-do-relatorio-final}

A entrega do relatório do grupo, tomando como base este aqui, reflete a
execução desta etapa.

\subsection{CRISP-DM Fase.Atividade 6.5 - Apresentação do relatório
final}\label{crisp-dm-fase.atividade-6.5---apresentacao-do-relatorio-final}

Como já informado, não será feita apresentação do relatório, mas
esperamos que publicações científicas possam ser geradas com pelo seu
grupo, com o apoio dos professores da disciplina.

\subsection{CRISP-DM Fase.Atividade 6.6 - Revisão sobre a execução do
projeto}\label{crisp-dm-fase.atividade-6.6---revisao-sobre-a-execucao-do-projeto}

Deve-se fazer aqui o registro de lições aprendidas, bem como traçadas
perspectivas futuras de aprimoramento deste trabalho, da disciplina de
Ciência de Dados para Todos etc.

\section{Referências}\label{referencias}

\begin{itemize}
\tightlist
\item
  Azevedo, Mário Luiz Neves de, João Ferreira de Oliveira, e Afrânio
  Mendes Catani. ``O Sistema Nacional de Pós-Graduação (SNPG) e o Plano
  Nacional de Educação (PNE 2014-2024): regulação, avaliação e
  financiamento''. Revista Brasileira de Política e Administração da
  Educação 32, nº 3 (2016).
  \url{http://dx.doi.org/10.21573/vol32n32016.68576}.
\item
  Can, Fazli, Tansel Özyer, e Faruk Polat, orgs. State of the Art
  Applications of Social Network Analysis. Lecture Notes in Social
  Networks. Switzerland: Springer International Publishing, 2014.
\item
  CAPES. ``Documentos de Área''. CAPES.gov.br. Acessado 12 de junho de
  2018.
  \url{http://avaliacaoquadrienal.capes.gov.br/documentos-de-area}.
\item
  ---------. ``Plano Nacional de Pós-Graduação - PNPG 2011/2020 Vol.
  1''. Brasília - DF, dezembro de 2010.
  \url{http://www.capes.gov.br/images/stories/download/Livros-PNPG-Volume-I-Mont.pdf}.
\item
  ---------. ``Plano Nacional de Pós-Graduação - PNPG 2011/2020 Vol.
  2''. Brasília - DF, dezembro de 2010.
  \url{http://www.capes.gov.br/images/stories/download/PNPG_Miolo_V2.pdf}.
\item
  ---------. ``Sucupira: coleta de dados, docentes de pós-graduação
  stricto sensu no Brasil 2015''. CAPES - Banco de Metadados, 16 de
  março de 2016.
  \url{http://metadados.capes.gov.br/index.php/catalog/63}.
\item
  Chapman, Pete, Julian Clinton, Randy Kerber, Thomas Khabaza, Thomas
  Reinartz, Colin Shearer, e Rüdiger Wirth. ``CRISP-DM 1.0: Step-by-Step
  Data Mining Guide''. USA: CRISP-DM Consortium, 2000.
  \url{https://www.the-modeling-agency.com/crisp-dm.pdf}.
\item
  Datacamp. ``Machine Learning with R (Skill Track)''. Datacamp, 2018.
  \url{https://www.datacamp.com/tracks/machine-learning}.
\item
  Fernandes, Jorge H C, e Ricardo Barros Sampaio. ``DataScienceForAll''.
  Zotero, 13 de junho de 2018.
  \url{https://www.zotero.org/groups/2197167/datascienceforall}.
\item
  ---------. ``Especificação do Trabalho Final da Disciplina de Ciência
  de Dados para Todos 2017.2: Estudo sobre a visibilidade internacional
  da produção científica das pós-graduações vinculadas às áreas de
  conhecimento da CAPES, na Universidade de Brasília (Comunicação
  Interna)''. Disciplina 116297 - Tópicos Avançados em Computadores,
  turma D, do semestre 2017.2, do Departamento de Ciência da Computação
  do Instituto de Ciências Exatas da Universidade de Brasília, 28 de
  novembro de 2017.
  \url{https://aprender.ead.unb.br/pluginfile.php/474549/mod_resource/content/1/Estudo\%20da\%20Cie\%CC\%82ncia.pdf}.
\item
  Fernandes, Jorge H C, Ricardo Barros Sampaio, e João Ribas de Moura.
  ``Ciência de Dados para Todos (Data Science For All) - 2018.1 -
  Análise da Produção Científica e Acadêmica da Universidade de Brasília
  - Modelo de Relatório Final da Disciplina - Departamento de Ciência da
  Computação da UnB''. Disciplina 116297 - Tópicos Avançados em
  Computadores, turma D, do semestre 2018.1, do Departamento de Ciência
  da Computação do Instituto de Ciências Exatas da Universidade de
  Brasília, 13 de junho de 2018.
\item
  Frickel, Scott, e Kelly Moore. The New Political Sociology of Science:
  Institutions, Networks, and Power. Science and technology in society.
  USA: The University of Wisconsin Press, 2006.
\item
  Graduate Prospects Ltd. ``Job profile: Higher education lecturer'',
  2018.
  \url{https://www.prospects.ac.uk/job-profiles/higher-education-lecturer}.
\item
  Kalpazidou Schmidt, Evanthia, e Ebbe Krogh Graversen. ``Persistent
  factors facilitating excellence in research environments''. Higher
  Education 75, nº 2 (1º de fevereiro de 2018): 341--63.
  \url{https://doi.org/10.1007/s10734-017-0142-0}.
\item
  Kilduff, Martin, e Wenpin Tsai. Social Networks and Organizations. UK:
  Sage Publications, 2003.
\item
  Kolaczyk, Eric D., e Gábor Csárdi. Statistical Analysis of Network
  Data with R. USA: Springer, 2014.
\item
  Kuhn, Max, Jed Wing, Steve Weston, Andre Williams, Chris Keefer, Allan
  Engelhardt, Tony Cooper, et al. ``Package `Caret' - Classification and
  Regression Training'', 27 de maio de 2018.
  \url{https://cran.r-project.org/web/packages/caret/caret.pdf}.
\item
  Leite, Fernando César Lima. ``Considerações básicas sobre a Avaliação
  do Sistema Nacional de Pós-Graduação''. Comunicação Pessoal (slides).
  Universidade de Brasília, abril de 2018.
  \url{https://aprender.ead.unb.br/pluginfile.php/502250/mod_resource/content/1/Considera\%C3\%A7\%C3\%B5es\%20b\%C3\%A1sicas\%20sobre\%20a\%20Avalia\%C3\%A7\%C3\%A3o\%20do\%20Sistema\%20Nacional.pdf}.
\item
  Lusher, Dean, Johan Koskinen, e Garry Robins, orgs. Exponential Random
  Graph Models for Social Networks: Theory, methods, and applications.
  Structural Analysis in the Social Sciences. USA: Cambridge University
  Press, 2013.
\item
  Mariscal, Gonzalo, Óscar Marbán, e Covadonga Fernández. ``A survey of
  data mining and knowledge discovery process models and
  methodologies''. The Knowledge Engineering Review 25, nº 2 (2010):
  137--66. \url{https://doi.org/10.1017/S0269888910000032}.
\item
  Nery, Guilherme, Ana Paula Bragaglia, Flávia Clemente, e Suzana
  Barbosa. ``Nem tudo parece o que é: Entenda o que é plágio''.
  Instituto de Arte e Comunicação Social da UFF, 2009.
  \url{http://www.noticias.uff.br/arquivos/cartilha-sobre-plagio-academico.pdf}.
\item
  Nooy, Wouter de, Andrej Mrvar, e Vladimir Batagelj. Exploratory Social
  Network Analysis with Pajek. Structural Analysis in the Social
  Sciences. USA: Routldge, 2005.
\item
  Pátaro, Cristina Saitê de Oliveira, e Frank Antonio Mezzomo. ``Sistema
  Nacional de Pós-Graduação no Brasil: estrutura, resultados e desafios
  para política de Estado - Lívio Amaral''. Revista Educação e
  Linguagens 2, nº 3 (julho de 2013): 11--17.
\item
  Schwartzman, Simon. ``A Ciência da Ciência''. Ciência Hoje 2, nº 11
  (março de 1984): 54--59.
\item
  Silver, Nate. The Signal and the Noise: Why so many predictions fail
  --- but some don't. USA: The Penguin Press HC, 2012.
\item
  Vicari, Donatella, Akinori Okada, Giancarlo Ragozini, e Claus Wiehs.
  Analysis and Modeling of Complex Data in Behavioral and Social
  Sciences. Studies in Classifi cation, Data Analysis, and Knowledge
  Organization. Switzerland: Springer, 2014.
\item
  Wickham, Hadley, e Garrett Grolemund. R for Data Science: Import,
  Tidy, Transform, Visualize, and Model Data. USA: O'Reilly, 2016.
\end{itemize}


\end{document}
